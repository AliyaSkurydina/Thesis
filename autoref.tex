\documentclass[%
autoref,     % тип документа
href,        % использовать пакет hyperref для создания гиперссылок
facsimile,   % отображать факсимиле диссертанта и ученого секретаря
colorlinks,  % цветные гиперссылки
%fixint,     % отключить прямые знаки интегралов
%times,      % шрифт Times как основной
%classified, % гриф секретности
]{disser}

\usepackage[
  a4paper, mag=1000,
  left=2.5cm, right=1cm, top=2cm, bottom=2cm, headsep=0.7cm, footskip=1cm
]{geometry}
\usepackage{amsmath, amsthm, amssymb}
\usepackage[T1,T2A]{fontenc}
\usepackage[utf8]{inputenc}
\usepackage[english,russian]{babel}
\usepackage{tabularx}
\usepackage{csquotes}
\ifpdf\usepackage{epstopdf}\fi
\usepackage{lastpage}

\usepackage[style=gost-numeric,
  backend=biber,
  language=auto,
  hyperref=auto,
  autolang=other,
  defernumbers=true,
  maxbibnames=4,
  sorting=none,
  movenames=false
]{biblatex}

\addbibresource{thesis.bib}

% Номера страниц снизу и по центру
\pagestyle{footcenter}
\chapterpagestyle{footcenter}

% Точка с запятой в качестве разделителя между номерами цитирований
%\setcitestyle{semicolon}

% Путь к файлам с иллюстрациями
\graphicspath{{fig/}}

\newtheorem{theorem}{Теорема} 
\newtheorem{cor}{Следствие} 
\newtheorem{lemma}{Лемма}
\newtheorem{fact}{Утверждение}
\newtheorem{remark}{Замечание}

\begin{document}
% Включение файла с общим текстом диссертации и автореферата
% (текст титульного листа и характеристика работы).
% Общие поля титульного листа диссертации и автореферата
\institution{Название организации}

\topic{Тема диссертации}

\author{Скурыдина Алия Фиргатовна}

\specnum{01.01.07}
\spec{Вычислительная математика}
%\specsndnum{01.04.07}
%\specsnd{Физика конденсированного состояния}

\sa{Акимова Елена Николаевна}
\sastatus{д.~ф.-м.~н., доц.}
%\sasnd{ФИО второго руководителя}
%\sasndstatus{к.~ф.-м.~н., проф.}

%\scon{ФИО консультанта}
%\sconstatus{д.~ф.-м.~н., проф.}
%\sconsnd{ФИО второго консультанта}
%\sconsndstatus{д.~ф.-м.~н., проф.}

\city{Екатеринбург}
\date{\number\year}

% Общие разделы автореферата и диссертации
\mkcommonsect{actuality}{Актуальность темы исследования.}{%
Построение итеративно регуляризованных алгоритмов востребовано для решения широкого круга прикладных задач. Так, решение структурных обратных задач гравиметрии и магнитометрии сводится к решению нелинейных интегральных уравнений Урысона первого рода.
}

\mkcommonsect{development}{Степень разработанности темы исследования.}{
Текст о степени разработанности темы.
}

\mkcommonsect{objective}{Цели и задачи диссертационной работы:}{%
построить новые методы решения нелинейных операторных уравнений, исследовать их сходимость.

Для достижения поставленных целей были решены следующие задачи:
\begin{itemize}
	\item для нелинейного уравнения с монотонным оператором доказаны теоремы сходимости для регуляризованного метода Гаусса---Ньютона, доказана сильная фейеровость итерационных процессов;
	\item построены регуляризованные методы градиентного типа, названные нелинейными аналогами $\alpha$-процессов, для нелинейного уравнения с монотонным оператором доказаны теоремы сходимости для них, доказана сильная фейеровость итерационных процессов;
	\item для задачи с немонотонным оператором с производной, имеющей неотрицательный спектр, доказаны теоремы сходимости методов Ньютона, нелинейных $\alpha$-процессов и их модифицированных вариантов;
	\item предложена вычислительная оптимизация метода Ньютона и его модифицированного варианта при решении задач с матрицей производной, близкой к ленточной;
	\item lля решения систем нелинейных интегральных уравнений  с ядром оператора структурной обратной задачи гравиметрии в двуслойной среде предложен покомпонентный метод, основанный на методе Ньютона;
	\item для решения систем нелинейных уравнений  структурных обратных задач гравиметрии в многослойной среде предложен подход на основе метода Левенберга-Марквардта – покомпонентный метод типа Левенберга-Марквардта;
	\item проведены численные эксперименты, интерпретированы результаты.
\end{itemize}
}

\mkcommonsect{novelty}{Научная новизна.}{%
Результаты, полученные в диссертационной работе, являются новыми и состоят в следующем:

	в рамках двухэтапного метода построения регуляризующего алгоритма обоснованы сходимость метод Ньютона и нелинейные аналоги альфа-процессов: метод минимальной ошибки (ММО), метод наискорейшего спуска (МНС) и метод минимальных невязок (ММН). Также установлена сходимость модифицированных вариантов методов ММО, МНС, ММН, когда производная оператора вычисляется в начальной точке итераций. Рассмотрены два случая: оператор задачи является монотонным, либо оператор является конечномерным и его производная имеет неотрицательный спектр.
	
	Для решения систем нелинейных интегральных уравнений  с ядром оператора структурной обратной задачи гравиметрии в двуслойной среде предложен покомпонентный метод, основанный на методе Ньютона. 
	Предложена вычислительная оптимизация метода Ньютона и его модифицированного варианта в виде перехода от плотно заполненной матрицы производной оператора к ленточной в силу особенности строения ядер интегральных операторов задач грави- магнитометрии.
	Для решения систем нелинейных уравнений  структурных обратных задач гравиметрии в многослойной среде предложен подход на основе метода Левенберга-Марквардта – покомпонентный метод типа Левенберга-Марквардта.

}

\mkcommonsect{value}{Теоретическая и практическая значимость.}{%
Результаты, изложенные в диссертации, могут быть использованы для решения нелинейных операторных уравнений, в частности, задач гравиметрии и магнитометрии.
}

\mkcommonsect{methods}{Методология и методы исследования.}{%
Текст о методах исследования.
}

\mkcommonsect{results}{Положения, выносимые на защиту:}{%
1. Сформулированы и доказаны теоремы, устанавливающие сильную фейеровость оператора шага итераций методов:
	\begin{itemize}
		\item 	метод Ньютона;
		\item	метод минимальной ошибки и его модифицированный вариант;
		\item	метод наискорейшего спуска и его модифицированный вариант;
		\item	метод минимальных невязок и его модифицированный вариант.
	\end{itemize}
	
	Доказана сильная фейеровость оператора шага итераций данных методов в случае монотонного оператора задачи и в случае конечномерного оператора с производной, имеющей неотрицательный спектр. Доказывается линейная скорость сходимости итерационных процессов.
	
	2. Предложена вычислительная оптимизация метода Ньютона, которая в задачах гравиметрии и магнитометрии обеспечивает более высокую точность численного решения, а также уменьшает время счета программ.
	
	3. Предложены покомпонентные методы:
		\begin{itemize}
			\item покомпонентный основанный на методе Ньютона для решения нелинейного интегрального уравнения в задаче гравиметрии в двухслойной среде;
			\item покомпонентный метод типа Левенберга-Марквардта для решения систем нелинейных уравнений  структурных обратных задач гравиметрии в многослойной среде.
		\end{itemize} Данные методы обладает меньшей вычислительной сложностью в отличие от классических методов Ньютона и Левенберга-Марквардта.
		
Вычислительные эксперименты показывают, что предложенные метод позволяют существенно уменьшить вычислительную сложность задачи и являются экономичными по потреблению памяти ЭВМ.
		
		4. Проведены численные эксперименты для модельных и квазиреальных геофизических данных, разработан комплекс параллельных программ для многоядерных и графических процессоров с использованием технологий OpenMP, CUDA. 
}

\mkcommonsect{approbation}{Степень достоверности и апробация результатов.}{%
Основные результаты по материалам диссертационной работы докладывались на конференциях:

1. XIV и XV Уральская молодежная научная школа по геофизике (Пермь, 2013 г., Екатеринбург 2014 г.);

2. Международная коференция "Параллельные вычислительные технологии" (Ростов-на-Дону, 2014 г., Екатеринбург, 2015 г., Казань, 2017 г.);

3. Международная конференция «Геоинформатика: теоретические и прикладные аспекты» (Киев 2014, 2015, 2016 г.)

4. Международная конференция "Актуальные проблемы вычислительной и прикладной математики" (Новосибирск, 2014 г.)

5. Международный научный семинар по обратным и некорректно поставленным задачам (Москва, 2015 г.)
}

\mkcommonsect{pub}{Публикации.}{%
Материалы диссертации опубликованы в $N$ печатных работах, из них $n_1$
статей в рецензируемых журналах~\cite{Ivanov_1999_Journal_17_173,
Petrov_2001_Journal_23_12321,Sidorov_2002_Journal_32_1531}, $n_2$ статей в
сборниках трудов конференций и $n_3$ тезисов докладов.
}

\mkcommonsect{contrib}{Личный вклад автора.}{%
Содержание диссертации и основные положения, выносимые на защиту, отражают персональный вклад автора в опубликованные работы.
Подготовка к публикации полученных результатов проводилась совместно с соавторами, причем вклад диссертанта был определяющим. Все представленные в диссертации результаты получены лично автором.
}

\mkcommonsect{struct}{Структура и объем диссертации.}{%
Диссертация состоит из введения, обзора литературы, $n$ глав, заключения и библиографии.
Общий объем диссертации $P$ страниц, из них $p_1$ страницы текста, включая $f$ рисунков.
Библиография включает $B$ наименований на $p_2$ страницах.
}


% номер копии для грифа секретности
%\copynum{1}
% класс доступа
%\classlabel{Для служебного пользования}

\title{АВТОРЕФЕРАТ\\
диссертации на соискание ученой степени\\
кандидата физико-математических наук}

\maketitle

% Внутренняя сторона обложки
\thispagestyle{empty}
\vspace*{-2cm}
\noindent
\small
\begin{center}
Работа выполнена в \emph{Федеральном государственном бюджетном учреждении науки Институт математики и механики им. Н. Н. Красовского Уральского отделения Российской академии наук}.
\end{center}
\vskip1ex
\begin{tabularx}{\linewidth}{lp{1cm}X}
Научный руководитель:  & & \emph{доктор физико-математических наук}, \\
                       & & \emph{доцент Акимова Елена Николаевна}
\\
Официальные оппоненты: & & \emph{Танана Виталий Павлович}\\
					& & \emph{доктор физико-математических наук}, \\
                       & & \emph{профессор, главный научный сотрудник кафедры Системного программирования ФГАОУ ВО <<Южно-Уральский государственный
                       	университет (национальный исследовательский
                       	университет)>> (г. Челябинск)}, \\
                       & & \emph{Ягола Анатолий Григорьевич} \\
                       & & \emph{доктор физико-математических наук}, \\
                       & & \emph{профессор, профессор кафедры математики физического факультета ФГБОУ ВО <<Московский государственный университет имени М. В. Ломоносова>>}, \\
                       
Ведущая организация:   & & \emph{ФГАОУ ВО <<Казанский (Приволжский) федеральный университет>>}\\
\end{tabularx}

\vskip2ex\noindent
Защита состоится \datefield{} в \rule[0pt]{1cm}{0.5pt} часов
на заседании диссертационного совета \emph{Д 004.006.04} при \emph{ФГБУН Институт математики и механики им. Н. Н. Красовского УрО РАН} по адресу:
\emph{620990, Екатеринбург, ул. Софьи Ковалевской, 16, актовый зал}

\vskip1ex\noindent
С диссертацией можно ознакомиться в библиотеке
\emph{ФГБУН Институт математики и механики им. Н. Н. Красовского УрО РАН}.

\vskip1ex\noindent
Автореферат разослан \datefield{}

\vskip2ex\noindent
Отзывы и замечания по автореферату в двух экземплярах, заверенные
печатью, просьба высылать по вышеуказанному адресу на имя ученого секретаря
диссертационного совета.

\vfill\noindent
\begin{minipage}[b]{0.4\linewidth}
  Ученый секретарь\\
  диссертационного совета,\\
  \emph{доктор физ.-мат. наук}, \emph{с.н.с.}
\end{minipage}
\hfill
% вставка файла, содержащего факсимиле ученого секретаря
\makeatletter
\ifDis@facsimile
  \includegraphics[width=3cm]{sec-facsimile}\hfill
\fi%
\makeatother%
\emph{В. Д. Скарин}

\clearpage

\normalsize
\nsection{1. Общая характеристика работы}
\vskip -1mm
% Актуальность работы
\actualitysection
\actualitytext

	%Теорию решения некорректных задач развивали А.~Н.~Тихонов, М.~М.~Лаврентьев, В.~К.~Иванов, А.~Б.~Бакушинский, Б.~Т.~Поляк, А.~В.~Гончарский, В.~В.~Васин, А.~Л.~Агеев, В.~П.~Танана, А.~Г.~Ягола, A.~Neubauer, O.~Scherzer, B.~Kaltenbacher, U.~Tautenhahn и др.
	
	%Для решения систем нелинейных уравнений проводились исследования в работах Л.~В.~Канторовича, А.~Б.~Бакушинского, М.~Ю.~Кокуриным, Б.~Т.~Поляка, J.~M.~Ortega и W.~C.~Rheinboldt,	A.~Neubauer, M.~J.~D.~Powell, 	J.~C.~Gilbert, J.~Nocedal, S.~J.~Wright.
	%L.~Landweber, M.~Hanke. %Градиентные методы с применением метода Ландвебера исследовались М.~Ю.~Кокуриным.
	%Методы решения структурных обратных задач гравиметрии и магнитометрии предложены в работах В.~Б.~Гласко, В.~Н.~Страхова.
	
	Основы теории некорректно поставленных задач были заложены в 50--60 годы прошлого века в работах выдающихся российских математиков А.~Н.~Тихонова, В.~К.~Иванова, М.~М.~Лаврентьева и дальнейшее ее развитие было продолжено в работах их последователей и учеников. В работах А.~Б.~Бакушинского$^1$ сформулирован принцип итеративной регуляризации. 
	
%	В последние три десятилетия получили развитие устойчивые (регулярные) методы решения нелинейных некорректных задач на основе принципа итеративной регуляризации и иных подходах. Эти исследования связаны с именами Б.~Т.~Поляка, А.~В.~Гончарского, М.~Ю.~Кокурина, В.~В.~Васина, В.~Г.~Романова, С.~И.~Кабанихина, Ф.~П.~Васильева, В.~А.~Морозова, А.~Г.~Яголы, А.~С.~Леонова, А.~Л.~Агеева, В.~П.~Тананы, В.~И.~Максимова, А.~И.~Короткого, А.~Б.~Смирновой, A.~Neubauer, B.Kaltenbacher, H.~W.~Engl, M.~Hanke, C.~Boeckmann.

Устойчивые методы решения нелинейных некорректных задач 
строились и исследовались в работах А.~Л.~Агеева, В.~В.~Васина, А.~В.~Гончарского, С.~И.~Кабанихина, М.~Ю.~Кокурина, А.~С.~Леонова, В.~А.~Морозова, Б.~Т.~Поляка, В.~П.~Тананы, А.~Г.~Яголы, H.~W.~Engl, M.~Hanke, A.~Neubauer, B.~Kaltenbacher.
	
%	Структурные задачи гравиметрии и магнитометрии --- важный класс нелинейных некорректных задач. Если исходные данные о геофизических полях измеряются на большой площади, то это приводит к необходимости решать системы нелинейных уравнений большой размерности с использованием многопроцессорных вычислителей и технологий распараллеливания. Для их решения широко использовались регуляризованные методы Ньютона, Левенберга --Марквардта и процессы градиентного типа (А.~Б.~Бакушинский, М.~Ю.~Кокурин, В.~В.~Васин$^1$, Е.~Н.~Акимова$^2$, Л.~Ю.~Тимерханова, Г.~Я.~Пересторонина, В.~Е.~Мисилов), а также экономичный метод локальных поправок (П.~С.~Мартышко, И.~Л.~Пруткин$^3$).
%\vskip 5pt

В Екатеринбурге в ИГФ УрО РАН разработана оригинальная методика решения обратных задач гравиметрии и магнитометрии с использованием идей регуляризации, построены алгоритмы на основе метода локальных поправок (П.С.~Мартышко$^2$,  И.Л.~Пруткин, Н.В.~Федорова, А.Л.~Рублев и др.).%, И.В.~Ладовский, А.Г.~Цидаев, Д.Д.~Бызов). 

В ИММ УрО РАН разработаны и исследованы параллельные алгоритмы на основе регуляризованных методов Ньютона, Левенберга -- Марквардта и процессов градиентного типа (В.~В.~Васин$^3$, Е.~Н.~Акимова$^4$, Г.~Я.~Пересторонина, Л.~Ю.~Тимерханова, В.~Е.~Мисилов).
 
{\scriptsize	
	\let\thefootnote\relax\let\thefootnote\relax\footnotetext{ 1. A. Bakushinsky, A. Goncharsky. Ill-Posed Problems: Theory and Applications. Berlin; Boston; London: Kluwer Academic Publishers, 1994. 258~p.}
		\let\thefootnote\relax\let\thefootnote\relax\footnotetext{\footnotesize 2. П.С. Мартышко, И.В. Ладовский, Н.В. Федорова, Д.Д. Бызов, А.Г. Цидаев. Теория и методы комплексной итерпретации геофизических данных. Екатеринбург: УрО РАН, 2016. 94 с.}
		\let\thefootnote\relax\let\thefootnote\relax\footnotetext{\footnotesize 3. В. В. Васин, Г. Я. Пересторонина, И. Л. Пруткин, Л. Ю. Тимерханова. Решение трехмерных обратных задач гравиметрии и магнитометрии для трехслойной среды // Мат. мод. 2003. Т. 15, №2. С. 69--76.}
		\let\thefootnote\relax\let\thefootnote\relax\footnotetext{\footnotesize 4. Е. Н. Акимова. Параллельные алгоритмы решения обратных задач гравиметрии и магнитометрии на МВС-1000 // Вестник ННГУ. 2009. №4. С. 181--189.}
}

При решении обратных задач гравиметрии и магнитометрии на больших сетках используются параллельные алгоритмы и многопроцессорные системы.
	
	%Алгоритмы решения обратных задач математической физики на основе регуляризованных методов Ньютона, Левенберга -- Марквардта и градиентных методов разрабатывались в ИММ УрО РАН В.~В.~Васиным$^1$, Е.~Н.~Акимовой$^2$, 
	%Л.~Ю.~Тимерхановой, Г.~Я.~Пересторониной, В.~Е.~Мисиловым.
%	{\scriptsize	
%	\let\thefootnote\relax\let\thefootnote\relax\footnotetext{ 1. A. Bakushinsky, A. Goncharsky. Ill-Posed Problems: Theory and Applications. Berlin; Boston; London: Kluwer Academic Publishers, 1994. 258~p.}
	%\let\thefootnote\relax\let\thefootnote\relax\footnotetext{\footnotesize 2. Akimova E. N., Vasin V. V. Stable Parallel algorythms for solving the inverse gravimetry and magnetometry problems // Intern. J. Engineering Modelling. 2004. Vol. 17. № 1-2, P. 13--19.}
%	\let\thefootnote\relax\let\thefootnote\relax\footnotetext{\footnotesize 3. И. Л. Пруткин. О решении трехмерной обратной задачи гравиметрии в классе контактных поверхностей методом локальных поправок // Изв. АН СССР. Физика Земли. 1986. №1. С. 67--77.}
%	\let\thefootnote\relax\let\thefootnote\relax\footnotetext{\footnotesize 3. В. В. Васин, Г. Я. Пересторонина, И. Л. Пруткин, Л. Ю. Тимерханова. Решение трехмерных обратных задач гравиметрии и магнитометрии для трехслойной среды // Мат. мод. 2003. Т. 15, №2. С. 69--76.}
%	\let\thefootnote\relax\let\thefootnote\relax\footnotetext{\footnotesize 4. Е. Н. Акимова. Параллельные алгоритмы решения обратных задач гравиметрии и магнитометрии на МВС-1000 // Вестник ННГУ. 2009. №4. С. 181--189.}
	%\let\thefootnote\relax\let\thefootnote\relax\footnotetext{\footnotesize 3. П. С. Мартышко, И. В. Ладовский, А. Г. Цидаев. Построение региональных геофизических моделей на основе комплексной интерпретации гравитационных и сейсмических данных // Физика Земли. 2010. №11. С. 23--35.}	
%	} 
	
%	Алгоритмы решения геофизических задач на основе метода локальных поправок разрабатывались в ИГФ УрО РАН 
%	П.~С.~Мартышко$^3$, И.~Л.~Пруткиным. 
%	{\scriptsize
%	\let\thefootnote\relax\let\thefootnote\relax\footnotetext{\footnotesize 4. П. С. Мартышко, И. В. Ладовский, А. Г. Цидаев. Построение региональных геофизических моделей на основе комплексной интерпретации гравитационных и сейсмических данных // Физика Земли. 2010. №11. С. 23--35.}
%	}

% Степень разработанности темы исследования
%\developmentsection
%\developmenttext

% Цели и задачи диссертационной работы
\objectivesection
\objectivetext

% Методология и методы исследования
\methodssection
\methodstext

% Научная новизна
\noveltysection
\noveltytext

% Теоретическая и практическая значимость
\valuesection
\valuetext

% Положения, выносимые на защиту
%\resultssection
%\resultstext

% Степень достоверности и апробация результатов
\approbationsection
\approbationtext

% Публикации
\pubsection
\pubtext

% Личный вклад автора
\contribsection
\contribtext

% Структура и объем диссертации
\structsection
\structtext

Исследования по теме диссертации выполнены в период с 2013 по 2017 годы в отделе некорректных задач анализа и приложений Института математики и механики УрО РАН.

Автор выражает глубокую благодарность своему научному руководителю доктору физико-математических наук, ведущему научному сотруднику ИММ УрО РАН Елене Николаевне Акимовой.

Автор выражает искреннюю признательность за постановку ряда проблем и внимание к работе члену-корреспонденту РАН, главному научному сотруднику ИММ УрО РАН Владимиру Васильевичу Васину.

\nsection{2. Содержание работы}

Во \textbf{введении} обосновывается актуальность темы исследований и выполнен краткий обзор публикаций по теме диссертации, сформулирована цель работы, показаны научная новизна и практическая значимость полученных результатов.

\textbf{В первой главе} рассматриваются методы решения некорректных задач с монотонным оператором. Обоснован двухэтапный метод на основе  регуляризованного метода Ньютона. Построены методы минимальной ошибки, наискорейшего спуска и минимальных невязок решения нелинейных уравнений и доказывается их сходимость к регуляризованному решению.

Рассматривается нелинейное уравнение $I$ рода
\begin{equation}\label{equ1}A(u)=f\end{equation}
в гильбертовом пространстве $H$ с монотонным непрерывно дифференцируемым по Фреше оператором $A$, для которого обратные операторы $A'(u)^{-1}$, $A^{-1}$ разрывны в окрестности решения, что влечет некорректность задачи $\eqref{equ1}$. Используется двухэтапный метод, в котором на первом этапе используется регуляризация по схеме Лаврентьева
\begin{equation}\label{equ2}A(u)+\alpha(u-u^0)-f_\delta=0,\end{equation}
где $\|f-f_\delta\|\leqslant\delta$, $u^0$ --- некоторое приближение к решению; а на втором этапе для аппроксимации регуляризованного решения $u_\alpha$ применяется либо регуляризованный метод Ньютона (РМН) (А. Б. Бакушинский$^5$ ($\gamma=1$, $\bar{\alpha}=\alpha=\alpha_k$)):
\vskip -10 mm
%либо модифицированный регуляризованный метод Ньютона,
{\scriptsize
	\let\thefootnote\relax\let\thefootnote\relax\footnotetext{\footnotesize 5. А. Б. Бакушинский. Регуляризующий алгоритм на основе метода Ньютона -- Канторовича для решения вариационных неравенств // ЖВМиМФ, 16:6 (1976). С. 1397--1604.}
	%\let\thefootnote\relax\let\thefootnote\relax\footnotetext{\footnotesize 5. В. В. Васин, Е. Н. Акимова, А. Ф. Миниахметова. Итерационные алгоритмы ньютоновского типа и их приложения к обратной задаче гравиметрии // Вестник ЮУрГУ, 6:3 (2013). С. 26--37.}
}
\begin{equation}\label{equ_rmn}
u^{k+1}=u^k-\gamma(A'(u^k)+\bar\alpha I)^{-1}(A(u^k)+\alpha(u^k-u^0)-f_\delta)\equiv{T(u^k)},
\end{equation}
либо нелинейные аналоги $\alpha$-процессов
\begin{equation}\label{equ_alphaproc}
u^{k+1}=u^k-\gamma\frac{\langle (A'(u^k)+\bar\alpha I)^{\varkappa}S_\alpha(u^k), S_\alpha(u^k)\rangle }{\langle(A'(u^k)+\bar\alpha I)^{\varkappa+1}S_\alpha(u^k), S_\alpha(u^k)\rangle }S_\alpha(u^k)\equiv{T(u^k)}
\end{equation}
при $\varkappa=-1,0,1$. Здесь $\alpha>0$, $\bar\alpha>0$ --- параметры регуляризации, $\gamma>0$ --- демпфирующий множитель, $S_\alpha(u)=A(u)+\alpha(u-u^0)-f_\delta$.
\begin{remark}
	Формула $\eqref{equ_alphaproc}$ при $\varkappa=1$ справедлива лишь для самосопряженного оператора $A'(u)$. В общем случае знаменатель дроби при $\varkappa=1$ следует заменить на $\|(A'(u)+\alpha I)S_\alpha (u)\|^2$.
\end{remark}
Итеративно регуляризованный метод Ньютона ($\gamma=1$, $ \bar\alpha=\alpha={\alpha}_k$) был предложен и исследован ранее в работах А. Б. Бакушинского, где априори выбирается последовательность ${\alpha}_{k(\delta)}$ и при более строгих условиях доказывается сходимость итераций к решению уравнения $\eqref{equ1}$ без оценки погрешности регуляризованного решения.

Итерационные $\alpha$-процессы были предложены в работах М.~А.~Красносель-\\ского$^6$ и др. для решения линейного уравнения с ограниченным самосопряженным положительно определенным оператором. Нелинейные аналоги модифицированных $\alpha$-процессов для решения некорректных задач были предложены и исследованы в работе В.~В.~Васина$^7$.

{\scriptsize
	\let\thefootnote\relax\let\thefootnote\relax\footnotetext{\footnotesize 6. М. А. Красносельский, Г. М. Забрейко, П. П. Забрейко и др. Приближенное решение операторных уравнений // М.: Наука, 1969.}
	\let\thefootnote\relax\let\thefootnote\relax\footnotetext{\footnotesize 7. В. В. Васин. Регуляризованные модифицированные $\alpha$-процессы для нелинейных уравнений с монотонным оператором // ДАН. 2016. Т.94, №1. С.13--16.}
}

Так как оператор $A$ --- монотонный, то его производная $A'(u^k)$ --- неотрицательно определенный оператор. Операторы $(A'(u^k)+\bar\alpha I)^{-1}$ существуют и ограничены, следовательно, процессы $\eqref{equ_rmn}$, $\eqref{equ_alphaproc}$ определены корректно. 

В данной главе в предположении, что производная $A'(u)$ удовлетворяет условию Липшица, устанавливается линейная скорость сходимости методов $\eqref{equ_rmn}$,  $\eqref{equ_alphaproc}$ и сильная фейеровость итераций. При истокообразной представимости решения асимптотическое правило останова итераций $k(\delta)$ определяется из равенства оценок погрешности для итераций и регуляризованного решения $u_\alpha$.

Пусть имеются следующие условия:
\begin{equation}\label{cond1.1}
\forall u, v \in S(u^0;R)\quad\|A(u)-A(v)\|\leqslant N_1\|u-v\|, \quad
\|A'(u)-A'(v)\|\leqslant N_2\|u-v\|
\end{equation}
где шар $S(u^0;R)$ содержит решения уравнений $\eqref{equ1}$, $\eqref{equ2}$ и известна оценка нормы производной в начальном приближении $u^0$
\begin{equation}\label{cond1.3}
\|A'(u^0)\| \leqslant N_1.
\end{equation}

\begin{theorem}\label{teo2.1} Пусть $A$ --- монотонный оператор, для которого выполнены условия $\eqref{cond1.1}$, $0<\alpha \leqslant \bar\alpha$, $\|u^0-u_\alpha\| \leqslant r$, $r\leqslant \alpha/N_2$. 
	
	Тогда для процесса $\eqref{equ_rmn}$ c $\gamma=1$ имеет место линейная скорость сходимости метода при аппроксимации единственного решения $u_\alpha$ регуляризованного уравнения $\eqref{equ2}$
	\begin{equation}\label{nwt_conv}
	\| u^k-u_\alpha \| \leqslant q^kr, \quad q=(1-\frac{\alpha}{2\bar\alpha}).
	\end{equation}
\end{theorem}

Усиленное свойство Фейера$^8$ для оператора $T\colon H\to H$ означает, что для некоторого $\nu>0$ выполнено соотношение
\begin{equation}\label{fejer_prop_uni}
\forall u\in H\quad \forall z\in Fix(T)\quad {\|T(u)-z\|}^2\leqslant{\|u-z\|}^2-\nu{\|u-T(u)\|}^2,
\end{equation}
где $Fix(T)$ --- множество неподвижных точек оператора $T$. Это влечет для итерационных точек $u^k$, порождаемых процессом $u^{k+1}=T(u^k)$, выполнение неравенства для всех $k\in\mathbb{N}$ и всех $z\in Fix(T)$
\begin{equation}\label{fejer_prop_it}
{\|u^{k+1}-z\|}^2\leqslant{\|u^k-z\|}^2-\nu{\|u^k-u^{k+1}\|}^2.
\end{equation}
{\scriptsize
\let\thefootnote\relax\let\thefootnote\relax\footnotetext{\footnotesize 8. В. В. Васин, И. И. Еремин. Операторы и итерационные процессы фейеровского типа. Теория и приложения. Екатеринбург: УрО РАН, 2005.}}
Множество фейеровских операторов является замкнутым относительно операций произведения и взятия выпуклой суммы, что позволяет строить гибридные итерационные процессы, а также учитывать априорные ограничения на решение в виде систем неравенств.

\begin{theorem} \label{teo2.3}
	Пусть выполнены условия $\eqref{cond1.1}$--$\eqref{cond1.3}$, $A'(u^0)$ --- самосопряженный оператор, 
$0<\alpha\leqslant\bar\alpha$, $\bar\alpha\geqslant 4N_1$, $\|u_\alpha-u^0\|\leqslant r$, $r\leqslant\alpha/8N_2$. Тогда при
	$\gamma<\frac{\alpha\bar\alpha}{2(N_1+\alpha)^2}$
	оператор шага $T$ процесса $\eqref{equ_rmn}$ при
	$$\nu=\frac{\alpha\bar\alpha}{2\gamma(N_1+\alpha)^2}-1$$
	удовлетворяет неравенству $\eqref{fejer_prop_uni}$, для итераций $u^k$ справедливо соотношение $\eqref{fejer_prop_it}$ и имеет место сходимость
	$\lim_{k\to\infty}\|u^k-u_\alpha\|=0.$
	Если параметр $\gamma$ принимает значение ${\gamma}^{opt}=\frac{\alpha\bar\alpha}{4(N_1+\alpha)^2},$ то справедлива оценка $$\|u^k-u_\alpha\|\leqslant q^k r, \quad q=\sqrt{1-\frac{{\alpha}^2}  {16(N_1+\alpha)^2}}.$$
\end{theorem}

Приводится оценка скорости сходимости итераций ММО, МНС и ММН.
\begin{theorem}\label{teo3.2}
	Пусть выполнены условия $\eqref{cond1.1}$--$\eqref{cond1.3}$, $A'(u^0)$ --- самосопряженный оператор, $0<\alpha \leqslant \bar\alpha$, $\bar\alpha \geqslant N_1$, для ММО $\|u_\alpha-u^0\|\leqslant r$, $r\leqslant \alpha/8N_2$.  Тогда при
	$$\gamma<\frac{2}{\mu _\varkappa}\quad (\varkappa=-1,0,1),$$
	где $\mu_\varkappa$ вычисляется для каждого из трех методов, для последовательности $\{u^k\}$, порождаемой $\alpha$-процессом, имеет место сходимость $\lim_{k\to\infty}\|u^k-u_\alpha\|=0, $ а при 
	$\gamma^{opt}=\frac{1}{\mu_\varkappa}$
	справедлива оценка $\|u^k-u_\alpha\|\leqslant q{_\varkappa^k}r,$ где
	$$
	q_{-1}=\sqrt{1-\frac{\alpha^2}{16(N_1+\alpha)^2}}, \quad q_0=\sqrt{1-\frac{\alpha^2\bar\alpha^2}{(N_1+\alpha)^2(N_1+\bar\alpha)^2}}, \quad $$$$q_1=\sqrt{1-\frac{\alpha^2\bar\alpha^4}{(N_1+\bar\alpha)^4}}.
	$$
\end{theorem}

%Результаты первой главы опубликованы в работе~\cite{VasSkur2017,}.

\textbf{Во второй главе} обоснована сходимость к регуляризованному решению итераций РМН, ММО, МНС, ММН в~конечномерном случае без требования монотонности оператора $A$ исходного уравнения. Представлены результаты численных экспериментов.

Пусть собственные значения $\lambda _i$ матрицы $A'(u)$ $n\times n$ различны между собой и неотрицательны. Тогда при $\bar\alpha>0$ матрица имеет представление $A'(u)+\bar\alpha I =S(u)\Lambda S^{-1}(u)$ и справедлива оценка
\begin{equation}\label{est4.1}
\|(A'(u)+\bar\alpha I)^{-1}\|\leqslant \frac{\mu (S(u))}{\bar\alpha+\lambda_{min}} \leqslant \frac{\mu(S(u))}{\bar\alpha},
\end{equation}
где столбцы матрицы $S(u)$ составлены из собственных векторов матрицы $A'(u)+\bar\alpha I$, $\Lambda$ --- диагональная матрица, ее элементы --- собственные значения матрицы $A'(u)+\bar\alpha I$, $\mu(S(u))=\|S(u)\|\cdot\|S^{-1}(u)\|$.

Рассмотрим теперь вариант теорем 2, 3, когда оператор $A\colon \mathbb{R}^n \to \mathbb{R}^n$ и его производная имеет неотрицательный спектр. С оценкой $\eqref{est4.1}$ доказывается теорема для регуляризованного метода Ньютона
\begin{theorem}\label{conv_rate_nemonot_nwt}
	Пусть выполнены условия $\eqref{cond1.1}$--$\eqref{cond1.3}$, а также: $\sup\{\mu(S(u)): u\in S(u_\alpha;r)\}\leqslant\bar S <\infty,$ 
	$A'(u^0)$ --- симметричная матрица, $0<\alpha\leqslant\bar\alpha$, $\bar\alpha\geqslant 4N_1$, $\|u_\alpha-u^0\|\leqslant r$, $r\leqslant\alpha/8N_2\bar S$.

	Тогда для метода $\eqref{equ_rmn}$ справедливо заключение теоремы (1.3), где
	$$\gamma<\frac{\alpha\bar\alpha}{2(N_1+\alpha)^2\bar S^2},
	\quad
	{\gamma}^{opt}=\frac{\alpha\bar\alpha}{4(N_1+\alpha)^2\bar S^2},$$ 
	$$\|u^k-u_\alpha\|\leqslant q^k r, \quad q=\sqrt{1-\frac{\alpha ^2}{16(N_1+\alpha)^2\bar S^2}}.$$
\end{theorem}
Аналогично докажем теорему о сходимости ММО, МНС и ММН.
\begin{theorem}\label{conv_rate_nemonot_alpha}
Пусть выполнены условия теоремы 4. 
Тогда при $\gamma<2/\mu _\varkappa$, \\$\varkappa=-1,0,1$, с соответствующими $\mu _\varkappa$ для каждого процесса, последовательности ${u^k}$, порождаемые процессом $\eqref{equ_alphaproc}$ при $\varkappa=-1,0,1$, сходятся к $u_\alpha$, т.е., $\lim_{k\to\infty}\|u^k-u_\alpha\|=0,$ а при $
\gamma^{opt}=1/\mu_\varkappa$
справедлива оценка $\|u^{k+1}-u_\alpha\|\leqslant q{_\varkappa^k}r,$ где
$$q_{-1}=\sqrt{1-\frac{\alpha^2}{64\bar S^2(N_1+\alpha)^2}}, \quad q_0=\sqrt{1-\frac{\alpha^2\bar\alpha^2}{36(N_1+\alpha)^2(N_1+\bar\alpha)^2}},$$
$$q_1=\sqrt{1-\frac{\alpha^2\bar\alpha^6}{36(N_1+\alpha)^2(N_1+\bar\alpha)^6}}.$$
\end{theorem}
Для модифицированных ММО, МНС и ММН, где производная $A'(u)$ вычисляется в начальном приближении $u^0$, также обосновывается сходимость.	
\begin{theorem}\label{conv_rate_nemonot_alpha_mod}
	Пусть выполнены условия $\eqref{cond1.1}$--$\eqref{cond1.3}$ $A'(u^0)$ --- самосопряженный оператор, спектр которого состоит из неотрицательных различных собственных значений, 
	$0<\alpha\leqslant\bar{\alpha}$, $\bar{\alpha}\geqslant N_1$, $\|u_\alpha-u^0\|\leqslant r$, $r\leqslant\alpha/6N_2$. Тогда при
	$\gamma <2/\mu _\varkappa \quad (\varkappa=-1,0,1)$
	для последовательности $\{u^k\}$, порождаемой модифицированным $\alpha$-процессом при соответствующем $\varkappa$, имеет место сходимость $\lim_{k\to\infty}\|u^k-u_\alpha\|=0, $ а при 
	$\gamma^{opt}=\frac{1}{\mu_\varkappa}$
	справедлива оценка $\|u^k-u_\alpha\|\leqslant q^k r,$ где
	$q=\sqrt{1-\frac{9\alpha^2}{64(N_1+\alpha)^2}}$.
\end{theorem}
\begin{remark}
	Предложенный подход к получению оценок скорости сходимости итерационных процессов полностью переносится на случай, когда спектр матрицы $A'(u^k)$, состоящий из различных вещественных значений, содержит набор малых по абсолютной величине отрицательных собственных значений. Пусть $\lambda _1$ --- отрицательное собственное значение с наименьшим модулем $|\lambda_1|$ и $\bar\alpha -|\lambda _1|=\bar\alpha _1<\alpha^*$. Тогда оценка $\eqref{est4.1}$ 
	трансформируется в неравенство
	$$\|(A'(u^k)+\bar\alpha I)^{-1}\|\leqslant\frac{\mu(S(u^k))}{\bar\alpha^*}\leqslant\frac{\bar S}{\bar\alpha^*}.$$
	Все теоремы остаются справедливыми при замене $\bar\alpha$ на $\bar\alpha^*$.
\end{remark}

Приводится оптимальная по порядку оценка погрешности двухэтапного метода с монотонным оператором на классе истокообразно представимых решений:
\begin{equation*}
\|u_{\alpha(\delta)}^{\delta, k}-\hat{u}\|\leqslant 4\sqrt{k_0 \delta},
\end{equation*}
где $u_{\alpha(\delta)}^{\delta, k}$ --- $k$-е приближение, $\hat{u}$ --- решение уравнения (\ref{equ1}), $k_0=(1+N_2\|v\|/2)\|v\|$ ($u^0-\hat{u}=A'(\hat{u})v$ --- истокообразная представимость решения). Для получения оценки используется результат U. Tautenhahn$^{9}$ для регуляризованного решения. В конечномерном случае для оператора $A'(u)$ с положительным спектром установлена оценка для невязки --- основной характеристики точности метода при решении задачи с реальными данными.
$$\|A(u_{\alpha(\delta)}^{\delta,k})-f_\delta\|\leqslant 2m\delta^p,$$
где для $\alpha(\delta)$ ограничена величина $\|u_{\alpha(\delta)}^{\delta}-u^0\|\leqslant m <\infty$, $\alpha(\delta)=\delta^p$. 
{\scriptsize
%\let\thefootnote\relax\let\thefootnote\relax\footnotetext{\footnotesize 9. В. В. Васин, А. Ф. Скурыдина. Двухэтапный метод построения регуляризующих алгоритмов для нелинейных некорректных задач // Труды ИММ УрО РАН. Т.23 В.1 (2017), С. 57–74.}
\let\thefootnote\relax\let\thefootnote\relax\footnotetext{\footnotesize 9. U. Tautenhahn. On the method of Lavrentiev regurarization for nonlinear ill-posed problems // Inverse Problem. 2002. Vol. 91, №1. P. 191–207.}
}

 Методы РМН, ММО, МНС, ММН и их модифицированные варианты использованы при решении обратной структурной задачи магнитометрии
\begin{equation*}\begin{aligned}
\left[A(u)\right](x,y)=\frac{\mu_0}{4\pi}\Delta J  \bigg\{&\iint_{D} \frac{H}{[(x-x')^2+(y-y')^2+H^2]^{3/2}}dx'dy' \notag\\
- &\iint_{D} \frac{u(x',y')}{[(x-x')^2+(y-y')^2+u^2(x',y')]^{3/2}}dx'dy' \bigg\}= B_z(x,y,0),
\end{aligned} \end{equation*}
где $\mu_0/{4\pi}=10^{-7}$ Гн/м --- магнитная постоянная, $\Delta J$ --- скачок $z$-компоненты вектора намагниченности, $z=H$ --- асимптотическая плоскость, $ B_z(x,y,0)$ --- функция, описывающая  аномальное поле, $D=\{c\leqslant x' \leqslant d, a\leqslant y' \leqslant b\}$, $z=u(x,y)$ --- искомая функция.

Число обусловленности $cond(A'_n(u_n^k))\approx 1.8\cdot 10^7$, спектр является неотрицательным и состоит из различных собственных значений, $\bar\alpha=10^{-2}$, $\alpha = 10^{-4}$, $\gamma=1$, $\varepsilon =\|u^k-\hat{u}\|/\|\hat{u}\| < 10^{-2}$. Итерационные методы достигают точности $\varepsilon$ за 4--5 итераций, у модифицированных методов меньше время счета.

%Результаты второй главы опубликованы в работах~\cite{VasSkur2017,VasSkur2015}.

\textbf{В третьей главе} предложены покомпонентные методы типа Ньютона и типа Левенберга -- Марквардта, а также вычислительная оптимизация метода Ньютона. Параллельные алгоритмы реализованы в виде комплекса программ на многоядерных и графических процессорах.

1. Задача гравиметрии о нахождении поверхности раздела в декартовой системе координат с осью $z$, направленной вниз, имеет вид
\vskip -2 mm
\begin{equation}\label{equ_grav_2l}
\begin{aligned}
A(u)=g\Delta\sigma \bigg\{ &\iint_{D} \frac{1}{[(x-x')^2+(y-y')^2+H^2]^{1/2}}dxdy \\
- &\iint_{D} \frac{1}{[(x-x')^2+(y-y')^2+u^2(x,y)]^{1/2}}dxdy\bigg\}=\Delta f(x',y',0),
\end{aligned} 
\end{equation}
	где $g$ --- гравитационная постоянная, равная $6.67\cdot10^{-8}$ см$^3/$г$\cdot c^2$, $\Delta\sigma=\sigma_2-\sigma_1$ --- скачок плотности на поверхности раздела сред $u(x,y)$, $D=\{c\leqslant x \leqslant d, a\leqslant y \leqslant b\}$, $\Delta f(x',y',0)$ --- аномальное гравитационное поле.%, вызванное отклонением поверхности от асимптотической плоскости $z=H$ для искомого решения $u(x,y)$.
	
	Итерации в методе Ньютона строятся по схеме
$$A'(u^k)(\Delta u^k)=-\big[A(u^k)-f\big],$$ где $A(u)=\int_{a}^{b}\int_{c}^{d}K(x,y, x',y',u^k(x,y))dxdy$ --- интегральный оператор задачи гравиметрии, $\Delta u^k=u^{k+1}-u^k$.
Т.е., для задачи гравиметрии
\begin{equation}\label{newton_integral_equ}
	f\Delta\sigma\int_{a}^{b}\int_{c}^{d}K'_u(x,y, x',y',u^k(x,y))\Delta u^k(x,y) dxdy=-\big[A(u(x',y'))-f(x',y')\big].
\end{equation}

%\begin{remark} (И. Л. Пруткин)
%	На значение гравитационного поля в точке $(x',y')$ наибольшее влияние оказывает глубина залегания поверхности в точке.
%\end{remark}  
Заменяя $\Delta u^k(x,y)$ в $\eqref{newton_integral_equ}$ на $\Delta u^k(x',y')=\textnormal{const}$ относительно переменных интегрирования, перейдем к приближенному соотношению
$$f\Delta\sigma(\Delta u^k(x',y'))\int_{a}^{b}\int_{c}^{d}K'_u(x,y, x',y',u^k(x,y)) dxdy\approx -\big[A(u(x',y'))-f(x',y')\big].$$

%Регуляризованный метод Ньютона имеет вид$^4$:
%$$ u^{k+1}=u^k-\gamma(A'(u^k)+\bar\alpha I)^{-1}(A(u^k)+\alpha(u^k-u^0)-f_\delta).$$

Покомпонентный метод типа Ньютона (ПМН) имеет вид$^{10}$:
$$u^{k+1}(x',y')=u^k(x',y')-\gamma\frac{1}{\varPsi(x',y')}\big([A(u^k)](x',y')-f(x',y')\big),$$
где $\varPsi(x',y')=f\Delta\sigma\int_{a}^{b}\int_{c}^{d}K'_u(x,y, x',y',u^k(x,y)) dxdy.$

Регуляризованный покомпонентный метод типа Ньютона имеет вид:
$$u^{k+1}(x',y')=u^k(x',y')-\gamma\frac{1}{\varPsi(x',y')+\bar{\alpha}}([A(u^k)](x',y')+$$ 
$$+\alpha (u^k(x',y')-u^0(x',y'))-f_\delta(x',y')),$$
где $\gamma$ --- демпфирующий множитель, $\alpha>0$, $\bar{\alpha} >0$ --- параметры регуляризации.

В дискретной записи итерационный процесс запишется
$$u_{m,l}^{k+1}=u_{m,l}^k-\frac{1}{\psi_{m,l}^k+\bar\alpha}([A_n(u^k)]_{m,l} + \alpha(u^k-u^0) -f_{m,l}),\quad 1\le m \le M, \quad 1\le l \le N,$$
{\scriptsize
	\let\thefootnote\relax\let\thefootnote\relax\footnotetext{\footnotesize 10. Akimova E., Skurydina A. A componentwise Newton type method for solving the structural inverse gravity problem // EAGE Geoinformatics 2015}
}
где $\psi_{m,l}^k=f\Delta\sigma\sum\limits_{i=1}^{M}\sum\limits_{j=1}^{N}
\Delta x\Delta y\frac{u_{ij}}{[(x_k-x'_j)^2+(y_l-y'_i)^2+(u_{ij})^2]^{3/2}}$, $n=M\cdot N$.
Сумма $\psi_{m,l}^k$ --- сумма элементов $(m\times M + l)$-й строки матрицы производной $A'_n(u_n^k)$. Вычислительная сложность ПМН для решения системы $n$ уравнений без учета сложности алгоритма вычисления $A_n(u_n^k)$ составляет $O(n)$, а в РМН сложность алгоритма составляет $O(n^2)$ при обращении $A'_n(u_n^k)$ итерационными методами.

2. Предложена вычислительная оптимизация метода Ньютона. Прозводная оператора $A$ в точке $u^k$ определяется формулами в задаче гравиметрии
%\begin{itemize}
%	\item в задаче гравиметрии
%	$$ [A'(u)]h=\iint_{D} \frac{u^k(x',y')h(x',y')}{[(x-x')^2+(y-y')^2+(u^k(x',y'))^2]^{3/2}}dx'dy',$$
%	\item в задаче магнитометрии
%	$$ [A'(u^k)]h=\iint_{D} \frac{(x-x')^2+(y-y')^2-2(u^k(x',y'))^2}{[(x-x')^2+(y-y')^2+(u^k(x',y'))^2]^{5/2}}h(x',y')dx'dy'.$$
%\end{itemize}
$$ [A'(u^k)]h=\iint_{D} \frac{u^k(x,y)h}{[(x-x')^2+(y-y')^2+(u^k(x',y'))^2]^{3/2}}dxdy,$$
и в задаче магнитометрии
$$ [A'(u^k)]h=\iint_{D} \frac{(x-x')^2+(y-y')^2-2(u^k(x,y))^2}{[(x-x')^2+(y-y')^2+(u^k(x,y))^2]^{5/2}}h dxdy.$$
После дискретизации интегрального оператора и его производной, получаем матрицу $A'_n(u^k)$ с диагональным преобладанием$^{11}$.
Без существенной потери точности учитываются только значения элементов $a_{ij}$, где $i+j\in[i-\beta;i+\beta]$.
{\scriptsize
	\let\thefootnote\relax\let\thefootnote\relax\footnotetext{\footnotesize 11. Акимова Е. Н., Миниахметова А. Ф., Мартышко М. П. Оптимизация и распараллеливание методов типа  Ньютона для решения структурных обратных задач гравиметрии и магнитометрии // EAGE Geoinformatics 2014.}
	}

3. Задача гравиметрии о нахождении нескольких поверхностей раздела имеет вид (суммарное поле получаем сложением полей от каждой поверхности)
\begin{equation}\label{equ_grav_multi}
		\begin{aligned}
		& A(u)=\sum_{l=1}^{L}g\Delta\sigma_l\frac{1}{4\pi}\times 
		\iint_D\bigg\{\frac{1}{[(x-x')^2+(y-y')^2+u_l^2(x,y)]^{1/2}} \\
		&-\frac{1}{[(x-x')^2+(y-y')^2+H_l^2]^{1/2}}\bigg\}dxdy=\Delta f(x',y',0),
		\end{aligned}
\end{equation}		
где $L$~--- число границ раздела, $g=6.67\cdot10^{-8}$ см$^3/$г$\cdot c^2$, $\Delta\sigma=\sigma_2-\sigma_1$ --- скачок плотности на поверхности раздела сред $u(x,y)$, $\Delta f(x',y',0)$ --- аномальное гравитационное поле.

Регуляризованный метод Левенберга -- Марквардта имеет вид 
$$u^{k+1}=u^k-\gamma[A'(u^k)^T A'(u^k)+\alpha I]^{-1} [A'(u^k)^*(A(u^k)-f_\delta)].$$

По аналогии с ПМН, получим покомпонентный метод типа Левенберга -- Марквардта$^{12}$:
$$ u_l^{k+1}=u_l^k-\gamma\frac{1}{\varphi_l+\bar{\alpha}}\Lambda[ A'(u_l^k)^T(A(u^k)-f_\delta)],$$
где $l$ --- номер границы раздела, $l=1,..,L$, $\Lambda$ --- диагональный весовой оператор, 
%\begin{equation*}
%\begin{aligned}
%\varphi_l=\bigg[ f\Delta\sigma\int_{a}^{b}\int_{c}^{d}
%K'_u(x, y, x', y', u_l^k(x',y'))dxdy\bigg] \notag \\ \times\bigg[f\Delta\sigma\int_{a}^{b}\int_{c}^{d}K'_u(x,y, x',y',u_l^k(x,y))dxdy\bigg], 
%\end{aligned}
%\end{equation*} 
$\varphi_l=\bigg[ f\Delta\sigma\int_{a}^{b}\int_{c}^{d}
K'_u(x', y', x, y, u_l^k(x',y'))dxdy\bigg] \notag \\ \times\bigg[f\Delta\sigma\int_{a}^{b}\int_{c}^{d}K'_u(x,y, x',y',u_l^k(x,y))dxdy\bigg], $
где $K'_u(x',y', x, y, u_l^k(x',y'))$ --- ядро интегрального оператора.
% Величина $\varphi_l$ зависит от $u_l^k$.
{\scriptsize
\let\thefootnote\relax\let\thefootnote\relax\footnotetext{ 12. Skurydina A. F. Regularized Levenberg –- Marquardt Type Method Applied to the Structural Inverse Gravity Problem in a Multilayer Medium and its Parallel Realization // Bulletin of South Ural State University. 2017. V.6, N.3. pp. 5--15}}
В дискретной форме
\begin{equation*}\label{comp_lm_meth_disc}
u_{l,i}^{k+1}=u_{l,i}^k-\gamma\frac{1}{\varphi_{l,i}+\bar{\alpha}}w_{l,i}\bigg[ \{A'(u_l^k)^T(A(u^k)-f_\delta)\}_i\bigg],
\end{equation*}
где $w_{l,i}$ --- $i$-й весовой множитель, зависящий от $l$-й границы раздела,
%\begin{equation*}
%\begin{aligned}
$
\varphi_{l,i}=\bigg[ f\Delta\sigma\sum\limits_{k=1}^{N}
\sum\limits_{m=1}^{M}
K'_u(x'_k,y'_m, \{x, y\}_i, u_l^k(x'_k,y'_m)) \Delta x' \Delta y'\bigg] \notag \\ \times\bigg[f\Delta\sigma\sum\limits_{k=1}^{N}
\sum\limits_{m=1}^{M}K'_u(x_k,y_m, \{x',y'\}_i,u_l^k(x_k,y_m))\Delta x' \Delta y'\bigg]. 
$
%\end{aligned}
%\end{equation*}
{\scriptsize
\let\thefootnote\relax\let\thefootnote\relax\footnotetext{\footnotesize 13. Е. Н. Акимова, В. Е. Мисилов, А. Ф. Скурыдина, А. И. Третьяков. Градиентные методы решения структурных обратных задач гравиметрии и магнитометрии на суперкомпьютере “Уран” // Вычислительные методы и программирование, 2015. Т. 16, 155–164.}}
Весовые множители выбираются по формулам из работы$^{13}$. 

По сравнению с методом Левенберга -- Марквадта, где вычислительная сложность алгоритмов достигает $O(n^3)$ в силу умножения матриц $[A'(u^k)^T A'(u^k)]$, вычислительная сложность покомпонентного метода типа Левенберга -- Марквадта составляет $O(n^2)$.

4. Параллельные алгоритмы реализованы в виде комплекса программ на многоядерных и графических процессорах с помощью технологий OpenMP и CUDA для решения задач гравиметрии и магнитометрии на сетках большого размера. 
Результаты решения модельных структурных обратных задач гравиметрии на сетках размера $512\times 512$ и $1000\times 1000$ продемонстрировали, что покомпонентный метод типа Ньютона работает в три раза быстрее метода Ньютона, а покомпонентный метод типа Левенберга -- Марквардта --- в десять раз быстрее метода Левенберга -- Марквардта. Время решения задачи на GPU для модели двухслойной среды методом ПМН уменьшилось в 100 раз, а время решения методом ПМЛ для модели многослойной среды уменьшилось в 22 раза. Предлагаемые автором покомпонентные методы являются экономичными по затратам оперативной памяти и по количеству вычислительных операций.
\vskip -5 mm
\nsection{3. Основные результаты диссертации}
\vskip -5 mm
1. Для нелинейного уравнения с монотонным оператором дано обоснование двухэтапного метода на основе регуляризованного метода Ньютона. 
Выполнены теоретические иследования методов минимальной ошибки, наискорейшего спуска и минимальных невязок. Доказаны теоремы сходимости и сильная фейеровость итерационных процессов при аппроксимации регуляризованного решения. Для задачи с немонотонным оператором и неотрицательным спектром его производной обоснована сходимость метода  Ньютона и нелинейных аналогов $\alpha$-процессов к регуляризованному решению.

2. Для решения нелинейных интегральных уравнений обратных задач гравиметрии предложены экономичные покомпонентные методы типа Ньютона и типа Левенберга – Марквардта. Предложена вычислительная оптимизация метода Ньютона для задач, где матрица производной имеет диагональное преобладание. 

3. Разработан комплекс параллельных программ для многоядерных и графических процессоров (видеокарт) решения обратных задач гравиметрии и магнитометрии на сетках большой размерности методами ньютоновского типа и покомпонентными методами.
%\vskip -1 mm 
%В дальнейшей научной работе автора предполагается исследование на сходимость покомпонентных методов типа Ньютона и Левенберга -- Марквардта.
% ----------------------------------------------------------------

\begingroup
\let\clearpage\relax
\printbibheading[title={Основные публикации по теме диссертации}]
\vskip -8 mm
%Если не работает - коммент-анкоммент, все будет чики-чики
\printbibliography[
heading=subbibliography,
keyword=hac,
title={Статьи в изданиях из перечня ВАК, SCOPUS}
]
\endgroup

\printbibliography[
heading=subbibliography,
keyword=nohac,
title={Другие публикации}
]
%\printbibliography[notkeyword=own,title={Цитированная литература}]

% ----------------------------------------------------------------
% Выходные данные
%\clearpage
%\thispagestyle{empty}
%\normalfont\selectfont
%\vspace*{2cm}
%\begin{center}
%\textit{Научное издание}\\
%\vskip 5cm
%\makeatletter
%\@author
%\vskip 1.5cm
%\@title{} на тему:\\
%\@topic\\
%\makeatother
%\end{center}
%\vfill

%Подписано в печать ... Формат $60\times 90$/16. Тираж 100 экз. Заказ 256.
%\noindent
%Санкт-Петербургская издательская фирма <<Наука>> РАН.
%199034, Санкт-Петербург, Менделеевская линия, 1,
%\href{http://www.naukaspb.spb.ru}{http://www.naukaspb.spb.ru}

\end{document}
