\documentclass[%
autoref,     % тип документа
href,        % использовать пакет hyperref для создания гиперссылок
facsimile,   % отображать факсимиле диссертанта и ученого секретаря
colorlinks,  % цветные гиперссылки
%fixint,     % отключить прямые знаки интегралов
%times,      % шрифт Times как основной
%classified, % гриф секретности
]{disser}

\usepackage[
  a4paper, mag=1000,
  left=2.5cm, right=1cm, top=2cm, bottom=2cm, headsep=0.7cm, footskip=1cm
]{geometry}
\usepackage[T2A]{fontenc}
\usepackage[utf8]{inputenc}
\usepackage[english,russian]{babel}
\usepackage{tabularx}
\usepackage{csquotes}
\ifpdf\usepackage{epstopdf}\fi

\usepackage[style=gost-numeric,
  backend=biber,
  language=auto,
  hyperref=auto,
  autolang=other,
  defernumbers=true,
  sorting=nyt
]{biblatex}

\addbibresource{thesis.bib}

% Номера страниц снизу и по центру
%\pagestyle{footcenter}
%\chapterpagestyle{footcenter}

% Точка с запятой в качестве разделителя между номерами цитирований
%\setcitestyle{semicolon}

% Путь к файлам с иллюстрациями
\graphicspath{{fig/}}

\begin{document}
% Включение файла с общим текстом диссертации и автореферата
% (текст титульного листа и характеристика работы).
% Общие поля титульного листа диссертации и автореферата
\institution{Название организации}

\topic{Тема диссертации}

\author{Скурыдина Алия Фиргатовна}

\specnum{01.01.07}
\spec{Вычислительная математика}
%\specsndnum{01.04.07}
%\specsnd{Физика конденсированного состояния}

\sa{Акимова Елена Николаевна}
\sastatus{д.~ф.-м.~н., доц.}
%\sasnd{ФИО второго руководителя}
%\sasndstatus{к.~ф.-м.~н., проф.}

%\scon{ФИО консультанта}
%\sconstatus{д.~ф.-м.~н., проф.}
%\sconsnd{ФИО второго консультанта}
%\sconsndstatus{д.~ф.-м.~н., проф.}

\city{Екатеринбург}
\date{\number\year}

% Общие разделы автореферата и диссертации
\mkcommonsect{actuality}{Актуальность темы исследования.}{%
Построение итеративно регуляризованных алгоритмов востребовано для решения широкого круга прикладных задач. Так, решение структурных обратных задач гравиметрии и магнитометрии сводится к решению нелинейных интегральных уравнений Урысона первого рода.
}

\mkcommonsect{development}{Степень разработанности темы исследования.}{
Текст о степени разработанности темы.
}

\mkcommonsect{objective}{Цели и задачи диссертационной работы:}{%
построить новые методы решения нелинейных операторных уравнений, исследовать их сходимость.

Для достижения поставленных целей были решены следующие задачи:
\begin{itemize}
	\item для нелинейного уравнения с монотонным оператором доказаны теоремы сходимости для регуляризованного метода Гаусса---Ньютона, доказана сильная фейеровость итерационных процессов;
	\item построены регуляризованные методы градиентного типа, названные нелинейными аналогами $\alpha$-процессов, для нелинейного уравнения с монотонным оператором доказаны теоремы сходимости для них, доказана сильная фейеровость итерационных процессов;
	\item для задачи с немонотонным оператором с производной, имеющей неотрицательный спектр, доказаны теоремы сходимости методов Ньютона, нелинейных $\alpha$-процессов и их модифицированных вариантов;
	\item предложена вычислительная оптимизация метода Ньютона и его модифицированного варианта при решении задач с матрицей производной, близкой к ленточной;
	\item lля решения систем нелинейных интегральных уравнений  с ядром оператора структурной обратной задачи гравиметрии в двуслойной среде предложен покомпонентный метод, основанный на методе Ньютона;
	\item для решения систем нелинейных уравнений  структурных обратных задач гравиметрии в многослойной среде предложен подход на основе метода Левенберга-Марквардта – покомпонентный метод типа Левенберга-Марквардта;
	\item проведены численные эксперименты, интерпретированы результаты.
\end{itemize}
}

\mkcommonsect{novelty}{Научная новизна.}{%
Результаты, полученные в диссертационной работе, являются новыми и состоят в следующем:

	в рамках двухэтапного метода построения регуляризующего алгоритма обоснованы сходимость метод Ньютона и нелинейные аналоги альфа-процессов: метод минимальной ошибки (ММО), метод наискорейшего спуска (МНС) и метод минимальных невязок (ММН). Также установлена сходимость модифицированных вариантов методов ММО, МНС, ММН, когда производная оператора вычисляется в начальной точке итераций. Рассмотрены два случая: оператор задачи является монотонным, либо оператор является конечномерным и его производная имеет неотрицательный спектр.
	
	Для решения систем нелинейных интегральных уравнений  с ядром оператора структурной обратной задачи гравиметрии в двуслойной среде предложен покомпонентный метод, основанный на методе Ньютона. 
	Предложена вычислительная оптимизация метода Ньютона и его модифицированного варианта в виде перехода от плотно заполненной матрицы производной оператора к ленточной в силу особенности строения ядер интегральных операторов задач грави- магнитометрии.
	Для решения систем нелинейных уравнений  структурных обратных задач гравиметрии в многослойной среде предложен подход на основе метода Левенберга-Марквардта – покомпонентный метод типа Левенберга-Марквардта.

}

\mkcommonsect{value}{Теоретическая и практическая значимость.}{%
Результаты, изложенные в диссертации, могут быть использованы для решения нелинейных операторных уравнений, в частности, задач гравиметрии и магнитометрии.
}

\mkcommonsect{methods}{Методология и методы исследования.}{%
Текст о методах исследования.
}

\mkcommonsect{results}{Положения, выносимые на защиту:}{%
1. Сформулированы и доказаны теоремы, устанавливающие сильную фейеровость оператора шага итераций методов:
	\begin{itemize}
		\item 	метод Ньютона;
		\item	метод минимальной ошибки и его модифицированный вариант;
		\item	метод наискорейшего спуска и его модифицированный вариант;
		\item	метод минимальных невязок и его модифицированный вариант.
	\end{itemize}
	
	Доказана сильная фейеровость оператора шага итераций данных методов в случае монотонного оператора задачи и в случае конечномерного оператора с производной, имеющей неотрицательный спектр. Доказывается линейная скорость сходимости итерационных процессов.
	
	2. Предложена вычислительная оптимизация метода Ньютона, которая в задачах гравиметрии и магнитометрии обеспечивает более высокую точность численного решения, а также уменьшает время счета программ.
	
	3. Предложены покомпонентные методы:
		\begin{itemize}
			\item покомпонентный основанный на методе Ньютона для решения нелинейного интегрального уравнения в задаче гравиметрии в двухслойной среде;
			\item покомпонентный метод типа Левенберга-Марквардта для решения систем нелинейных уравнений  структурных обратных задач гравиметрии в многослойной среде.
		\end{itemize} Данные методы обладает меньшей вычислительной сложностью в отличие от классических методов Ньютона и Левенберга-Марквардта.
		
Вычислительные эксперименты показывают, что предложенные метод позволяют существенно уменьшить вычислительную сложность задачи и являются экономичными по потреблению памяти ЭВМ.
		
		4. Проведены численные эксперименты для модельных и квазиреальных геофизических данных, разработан комплекс параллельных программ для многоядерных и графических процессоров с использованием технологий OpenMP, CUDA. 
}

\mkcommonsect{approbation}{Степень достоверности и апробация результатов.}{%
Основные результаты по материалам диссертационной работы докладывались на конференциях:

1. XIV и XV Уральская молодежная научная школа по геофизике (Пермь, 2013 г., Екатеринбург 2014 г.);

2. Международная коференция "Параллельные вычислительные технологии" (Ростов-на-Дону, 2014 г., Екатеринбург, 2015 г., Казань, 2017 г.);

3. Международная конференция «Геоинформатика: теоретические и прикладные аспекты» (Киев 2014, 2015, 2016 г.)

4. Международная конференция "Актуальные проблемы вычислительной и прикладной математики" (Новосибирск, 2014 г.)

5. Международный научный семинар по обратным и некорректно поставленным задачам (Москва, 2015 г.)
}

\mkcommonsect{pub}{Публикации.}{%
Материалы диссертации опубликованы в $N$ печатных работах, из них $n_1$
статей в рецензируемых журналах~\cite{Ivanov_1999_Journal_17_173,
Petrov_2001_Journal_23_12321,Sidorov_2002_Journal_32_1531}, $n_2$ статей в
сборниках трудов конференций и $n_3$ тезисов докладов.
}

\mkcommonsect{contrib}{Личный вклад автора.}{%
Содержание диссертации и основные положения, выносимые на защиту, отражают персональный вклад автора в опубликованные работы.
Подготовка к публикации полученных результатов проводилась совместно с соавторами, причем вклад диссертанта был определяющим. Все представленные в диссертации результаты получены лично автором.
}

\mkcommonsect{struct}{Структура и объем диссертации.}{%
Диссертация состоит из введения, обзора литературы, $n$ глав, заключения и библиографии.
Общий объем диссертации $P$ страниц, из них $p_1$ страницы текста, включая $f$ рисунков.
Библиография включает $B$ наименований на $p_2$ страницах.
}


% номер копии для грифа секретности
%\copynum{1}
% класс доступа
%\classlabel{Для служебного пользования}

\title{АВТОРЕФЕРАТ\\
диссертации на соискание ученой степени\\
кандидата физико-математических наук}

\maketitle

% Внутренняя сторона обложки
\thispagestyle{empty}
\vspace*{-2cm}
\noindent
\small
\begin{center}
Работа выполнена в \emph{Федеральном государственном бюджетном учреждении науки Институт математики и механики им. Н. Н. Красовского Уральского отделения Российской академии наук}.
\end{center}
\vskip1ex
\begin{tabularx}{\linewidth}{lp{1cm}X}
Научный руководитель:  & & \emph{доктор физико-математических наук}, \\
                       & & \emph{доцент Акимова Елена Николаевна}
\\
Официальные оппоненты: & & \emph{доктор физико-математических наук}, \\
                       & & \emph{профессор, главный научный сотрудник кафедры Системного программирования ФГАОУ ВО <<Южно-Уральский государственный
                       	университет (национальный исследовательский
                       	университет)>> (г. Челябинск)}, \\
                       & & \emph{Танана Виталий Павлович}\\
                       & & \emph{доктор физико-математических наук}, \\
                       & & \emph{профессор, профессор кафедры математики физического факультета ФГБОУ ВО <<Московский государственный университет имени М. В. Ломоносова>>}, \\
                       & & \emph{Ягола Анатолий Григорьевич}
\\
Ведущая организация:   & & \emph{ФГАОУ ВО <<Казанский (Приволжский) федеральный университет>>}\\
\end{tabularx}

\vskip2ex\noindent
Защита состоится \datefield{} в \rule[0pt]{1cm}{0.5pt} часов
на заседании диссертационного совета \emph{Д 004.006.04} при \emph{ФГБУН Институт математики и механики им. Н. Н. Красовского УрО РАН} по адресу:
\emph{620990, Екатеринбург, ул. Софьи Ковалевской, 16, актовый зал}

\vskip1ex\noindent
С диссертацией можно ознакомиться в библиотеке
\emph{ФГБУН Институт математики и механики им. Н. Н. Красовского УрО РАН}.

\vskip1ex\noindent
Автореферат разослан \datefield{}

\vskip2ex\noindent
Отзывы и замечания по автореферату в двух экземплярах, заверенные
печатью, просьба высылать по вышеуказанному адресу на имя ученого секретаря
диссертационного совета.

\vfill\noindent
\begin{minipage}[b]{0.4\linewidth}
  Ученый секретарь\\
  диссертационного совета,\\
  \emph{доктор физ.-мат. наук}, \emph{с.н.с.}
\end{minipage}
\hfill
% вставка файла, содержащего факсимиле ученого секретаря
\makeatletter
\ifDis@facsimile
  \includegraphics[width=3cm]{sec-facsimile}\hfill
\fi%
\makeatother%
\emph{Скарин В. Д.}

\clearpage

\normalsize
\nsection{Общая характеристика работы}

% Актуальность работы
\actualitysection
\actualitytext

% Степень разработанности темы исследования
%\developmentsection
%\developmenttext

% Цели и задачи диссертационной работы
\objectivesection
\objectivetext

% Научная новизна
\noveltysection
\noveltytext

% Теоретическая и практическая значимость
\valuesection
\valuetext

% Методология и методы исследования
%\methodssection
%\methodstext

% Положения, выносимые на защиту
\resultssection
\resultstext

% Степень достоверности и апробация результатов
\approbationsection
\approbationtext

% Публикации
\pubsection
\pubtext

% Личный вклад автора
\contribsection
\contribtext

% Структура и объем диссертации
\structsection
\structtext

\nsection{Содержание работы}

\textbf{Во Введении} обоснована актуальность диссертационной работы, сформулирована цель и аргументирована научная новизна исследований, показана практическая значимость полученных результатов, представлены выносимые на защиту научные положения.

\textbf{В первой главе} ...

Содержание первой главы.

Результаты первой главы опубликованы в работе~\cite{VasSkur2017}.

\textbf{Во второй главе} ...

Содержание второй главы.

Результаты второй главы опубликованы в работе~\cite{VasSkur2017}.

\textbf{В третьей главе} ...

Содержание третьей главы.

Результаты третьей главы опубликованы в работах~\cite{AkSkur2014}, \cite{AkSkur2015}, \cite{AkSkur2016}, \cite{Skur2017_2}.

\textbf{В Заключении}

\nsection{Основные результаты диссертации}

1. Для нелинейного уравнения с монотонным оператором доказаны теоремы сходимости для регуляризованного метода Ньютона, построены регуляризованные методы градиентного типа для решения нелинейного уравнения с монотонным оператором --- метод минимальной ошибки, метод наискорейшего спуска, метод минимальных невязок, названные нелинейными аналогами $\alpha$-процессов, доказаны теоремы сходимости для них, доказана сильная фейеровость итерационных процессов.

Для задачи с немонотонным оператором, производная которого имеет неотрицательный спектр, доказаны теоремы сходимости методов Ньютона, нелинейных $\alpha$-процессов и их модифицированных вариантов. 

2. Для решения систем нелинейных интегральных уравнений  с ядром оператора структурной обратной задачи гравиметрии для модели двухслойной среды предложен покомпонентный метод, основанный на методе Ньютона. Предложена вычислительная оптимизация метода Ньютона и его модифицированного варианта при решении задач с матрицей производной, близкой к ленточной; на примере решения обратной задачи гравиметрии продемонстрирована вычислительная экономичность модификации. Для решения систем нелинейных уравнений  структурных обратных задач гравиметрии для моделей двухслойной и многослойной сред предложен подход на основе метода Левенберга -- Марквардта --- покомпонентный метод типа Левенберга -- Марквардта.

3. Разработан комплекс параллельных программ, с использованием многоядерных процессоров для всех предложенных методов и с вычислением на графических процессорах (видеокартах) для покомпонентных методов и метода Ньютона и модифицированного варианта.

В дальнейшей научной работе автора предполагается исследование на сходимость покомпонентных методов типа Ньютона и Левенберга -- Марквардта.

% ----------------------------------------------------------------
\printbibheading[title={Основные публикации по теме диссертации}]
\printbibliography[heading=subbibliography,keyword=hac,title={Статьи в изданиях из перечня ВАК, SCOPUS}]
\printbibliography[heading=subbibliography,keyword=nohac,title={Другие публикации}]
%\printbibliography[notkeyword=own,title={Цитированная литература}]

% ----------------------------------------------------------------
% Выходные данные
\clearpage
\thispagestyle{empty}
\normalfont\selectfont
\vspace*{2cm}
\begin{center}
\textit{Научное издание}\\
\vskip 2cm
\makeatletter
\@author
\vskip 1.5cm
\@title{} на тему:\\
\@topic\\
\makeatother
\end{center}
\vfill
Подписано в печать~25.01.2011.
Формат~$60 \times 90$~1/16.
Тираж~100~экз.
Заказ~256.\\[2ex]
\noindent
Санкт-Петербургская издательская фирма <<Наука>> РАН.
199034, Санкт-Петербург, Менделеевская линия, 1,
\href{http://www.naukaspb.spb.ru}{http://www.naukaspb.spb.ru}

\end{document}
