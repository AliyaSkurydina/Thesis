\conclusion

Приведены основные результаты диссертационной работы.

1. Для нелинейного уравнения с монотонным оператором доказаны теоремы сходимости для регуляризованного метода Гаусса -- Ньютона, построены регуляризованные методы градиентного типа, названные нелинейными аналогами $\alpha$-процессов, для нелинейного уравнения с монотонным оператором доказаны теоремы сходимости для них, доказана сильная фейеровость итерационных процессов.

2. Для задачи с немонотонным оператором, производная которого имеет неотрицательный спектр, доказаны теоремы сходимости методов Ньютона, нелинейных $\alpha$-процессов и их модифицированных вариантов.

3. Предложена вычислительная оптимизация метода Ньютона и его модифицированного варианта при решении задач с матрицей производной, близкой к ленточной; на примере решения обратной задачи гравиметрии продемонстрирована вычислительная экономичность модификации. 

Для решения систем нелинейных интегральных уравнений  с ядром оператора структурной обратной задачи гравиметрии в двуслойной среде предложен покомпонентный метод, основанный на методе Ньютона. Для решения систем нелинейных уравнений  структурных обратных задач гравиметрии в многослойной среде предложен подход на основе метода Левенберга -- Марквардта --- покомпонентный метод типа Левенберга -- Марквардта.

Программные реализации предложенных методов поддерживают использование многоядерных процессоров, для покомпонентных методов и метода Ньютона и модифицированного варианта реализованы также программы для вычислений на графических процессорах.

В дальнейшей научной работе автора предполагается исследование на сходимость покомпонентных методов типа Ньютона и Левенберга -- Марквардта.
