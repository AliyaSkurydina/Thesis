\conclusion

Приведем основные результаты диссертационной работы.

1. Для нелинейного уравнения с монотонным оператором дано обоснование двухэтапного метода на основе регуляризованного метода Ньютона. 
Построены регуляризованные градиентные методы для решения нелинейного уравнения с монотонным оператором: метод минимальной ошибки, метод наискорейшего спуска, метод минимальных невязок. Доказаны теоремы сходимости и сильная фейеровость итерационных процессов при аппроксимации регуляризованного решения. Для задачи с немонотонным оператором и неотрицательным спектром его производной обоснована сходимость метода  Ньютона и нелинейных $\alpha$-процессов с модифицированными вариантами к регуляризованному решению.

2. Для решения нелинейных интегральных уравнений обратных задач гравиметрии предложены экономичные покомпонентные методы 
типа Ньютона и типа Левенберга – Марквардта. Предложена вычислительная оптимизация метода 
Ньютона для задач, где матрица производной имеет диагональное преобладание. 

3. Разработан комплекс параллельных программ для многоядерных и графических процессоров (видеокарт) решения обратных задач гравиметрии и магнитометрии на сетках большой размерности методами ньютоновского типа и покомпонентными методами.

В дальнейшей научной работе автора предполагается исследование на сходимость покомпонентных методов типа Ньютона и Левенберга -- Марквардта.
