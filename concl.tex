\conclusion

Приведены основные результаты диссертационной работы.

1. Для нелинейного уравнения с монотонным оператором доказаны теоремы о сходимости регуляризованного метода Ньютона. 
Построены нелинейные аналоги $\alpha$-процессов:  регуляризованные методы градиентного типа для решения нелинейного уравнения с монотонным оператором: метод минимальной ошибки, метод наискорейшего спуска, метод минимальных невязок. Доказаны теоремы сходимости и сильная фейеровость итерационных процессов. Для задачи с немонотонным оператором, производная которого имеет неотрицательный спектр, доказаны теоремы сходимости для метода  Ньютона, нелинейных $\alpha$-процессов и их модифицированных вариантов.

2. Для решения систем нелинейных интегральных уравнений с 
ядром оператора структурной обратной задачи гравиметрии для 
модели двухслойной среды предложен покомпонентный метод 
типа Ньютона. Предложена вычислительная оптимизация метода 
Ньютона и его модифицированного варианта при решении задач 
с матрицей производной, близкой к ленточной. Для решения систем 
нелинейных уравнений структурных обратных задач гравиметрии 
для модели многослойной среды предложен покомпонентный
метод типа Левенберга – Марквардта с весовыми множителями.
При решении модельных обратных задач гравиметри на больших 
сетках продемонстрирована экономичность предложенных методов
по вычислениям и затратам оперативной памяти.  

3. Разработан комплекс параллельных программ для многоядерных и графических процессоров (видеокарт) решения обратных задач гравиметрии и магнитометрии на сетках большой размерности методом Ньютона, модифицированным методом Ньютона,  методом Левенберга-Марквардта, покомпонентными методами типа Ньютона и типа Левенберга-Марквардта.

В дальнейшей научной работе автора предполагается исследование на сходимость покомпонентных методов типа Ньютона и Левенберга -- Марквардта.
