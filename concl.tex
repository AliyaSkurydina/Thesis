\conclusion

Приведены основные результаты диссертационной работы.

1. Для нелинейного уравнения с монотонным оператором доказаны теоремы сходимости для регуляризованного метода Ньютона, построены регуляризованные методы градиентного типа для решения нелинейного уравнения с монотонным оператором --- метод минимальной ошибки, метод наискорейшего спуска, метод минимальных невязок, названные нелинейными аналогами $\alpha$-процессов, доказаны теоремы сходимости для них, доказана сильная фейеровость итерационных процессов.

 Для задачи с немонотонным оператором, производная которого имеет неотрицательный спектр, доказаны теоремы сходимости методов Ньютона, нелинейных $\alpha$-процессов и их модифицированных вариантов. 

2. Для решения систем нелинейных интегральных уравнений  с ядром оператора структурной обратной задачи гравиметрии для модели двухслойной среды предложен покомпонентный метод, основанный на методе Ньютона. Предложена вычислительная оптимизация метода Ньютона и его модифицированного варианта при решении задач с матрицей производной, близкой к ленточной; на примере решения обратной задачи гравиметрии продемонстрирована вычислительная экономичность модификации. Для решения систем нелинейных уравнений  структурных обратных задач гравиметрии для моделей двухслойной и многослойной сред предложен подход на основе метода Левенберга -- Марквардта --- покомпонентный метод типа Левенберга -- Марквардта.

3. Разработан комплекс параллельных программ, с использованием многоядерных процессоров для всех предложенных методов и с вычислением на графических процессорах (видеокартах) для покомпонентных методов и метода Ньютона и модифицированного варианта.

В дальнейшей научной работе автора предполагается исследование на сходимость покомпонентных методов типа Ньютона и Левенберга -- Марквардта.
