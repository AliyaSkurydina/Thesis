\intro

%
% Используемые далее команды определяются в файле common.tex.
%

% Актуальность работы
\actualitysection
\actualitytext

Решение практических задач требует обработки больших объемов данных. Для уменьшения времени счета используются параллельные алгоритмы и многопроцессорные вычислители.

%На практике такие задачи нередко имеют большой объем входных данных, что приводит к необходимости решать системы нелинейных уравнений большой размерности. 

% Степень разработанности темы исследования
\developmentsection
\developmenttext

% Цель диссертационной работы
\objectivesection
\objectivetext

% Методология и методы исследования
\methodssection
\methodstext

% Научная новизна
\noveltysection
\noveltytext

% Практическая ценность
\valuesection
\valuetext

% Результаты и положения, выносимые на защиту
%\resultssection
%\resultstext

% Апробация работы
\approbationsection
\approbationtext

% Публикации
\pubsection
\pubtext

% Личный вклад автора
\contribsection
\contribtext

% Структура и объем диссертации
\structsection
\structtext


Исследования по теме диссертации выполнены в период с 2013 по 2017 в Отделе некорректных задач анализа и приложений Института математики и механики УрО РАН.

\textbf{Краткое содержание диссертации}

Во введении обосновывается актуальность темы проведенных исследований и дан обзор публикаций, близких к теме диссертации.

Во введении также сформулированы цели данной работы, научная новизна и значимость результатов, кратко изложено содержание работы.

В первой главе рассматриваются методы решения некорректных задач с нелинейным монотонным оператором. В работе используется двухэтапный подход, где на первом этапе происходит регуляризация по Лаврентьеву, на втором этапе решения задачи применяются итерационные алгоритмы решения регулярной задачи. Первый раздел содержит определения и постановку задачи. Второй раздел главы посвящен вопросам сходимости регуляризованного метода Ньютона. В третьем разделе построены итерационные процессы градиентного типа~--- нелинейные $\alpha$-процессы (метод минимальной ошибки, метод наискорейшего спуска и метод минимальных невязок) и доказывается их сходимость к регуляризованному решению. В четвертом разделе приводится оценка погрешности двухэтапного метода. В пятом разделе приводится пример решения модельного нелинейного интегрального уравнения рассмотренными в данной главе итерационными методами.

Во второй главе рассматривается конечномерный случай, где оператор исходного уравнения является немонотонным, но производная оператора имеет неотрицательный спектр, состоящий из различных собственных значений. Монотонность оператора $A$ исходного уравнения --- сильное требование, которое не выполняется во многих прикладных задачах, например, в задачах гравиметрии и магнитометрии. В данной главе ослабляется условие монотонности и обосновывается сходимость итераций метода Ньютона и нелинейных аналогов $\alpha$-процессов. Для немонотонного оператора в первом разделе доказаны теоремы сходимости метода Ньютона с регуляризацией, во втором разделе доказаны теоремы сходимости для нелинейных аналогов $\alpha$-процессов, в третьем разделе представлены следствия для модифицированных аналогов $\alpha$-процессов и оценка невязки двухэтапного метода, в~четвертом разделе приведен пример решения нелинейного уравнения с монотонным оператором рассмотренными методами.

В третьей главе предложены покомпонентные методы типа Ньютона и типа Левенберга -- Марквардта для решения обратной задачи гравиметрии, а также вычислительная оптимизация методов типа Ньютона. Покомпонентный метод типа Ньютона предлагается для решения обратных задач гравиметрии в случае модели двухслойной среды, а метод типа Левенберга -- Марквардта --- для решения задачи гравиметриив случае модели многослойной среды. В первом разделе предложен покомпонентный метод типа Ньютона и предложена вычислительная оптимизация метода Ньютона с учетом структуры матрицы производной исходного оператора. Во втором разделе предложен покомпонентный метод типа Левенберга -- Марквардта с весовыми множителями. Третий раздел посвящен описанию инструментов параллельного программирования OpenMP для многоядерных процессоров и CUDA для видеокарт. В четвертом разделе приводятся результаты решения модельных структурных обратных задач гравиметрии на сетках размера $512\times 512$ и $1000\times 1000$. Эксперименты показали, что время решения задачи гравиметрии покомпонентным методом типа Ньютона в три раза меньше времени решения задачи обычным методом Ньютона. Покомпонентный метод типа Левенберга -- Марквардта решает задачу гравиметрии в десять раз быстрее метода Левенберга -- Марквардта. Для задач, имеющих большой размер данных, приводятся результаты расчетов с использованием параллельных вычислений на многоядерных процессорах и графических ускорителях. В пятом разделе приводится описание комплекса параллельных программ для выполнения на многоядерных процессорах и графических ускорителях NVIDIA.

Автор глубоко благодарен своему научному руководителю доктору физико-математических наук, ведущему научному сотруднику ИММ УрО РАН Елене Николаевне Акимовой.

Автор выражает искреннюю признательность за постановку ряда проблем, внимание к работе, полезные замечания и обсуждения член-корреспонденту РАН, главному научному сотруднику ИММ УрО РАН Владимиру Васильевичу Васину.

Автор благодарен своим коллегам из ИММ УрО РАН за помощь и поддержку: заведующему Отделом некорректных задач, анализа и приложений д.ф.-м.н. А.Л. Агееву, к.ф.-м.н.~В.Е.~Мисилову, к.ф.-м.н.~П.А.~Чистякову, \\к.ф.-м.н.~Г.Г.~Скорику.

Отдельные слова благодарности автор выражает своей семье за любовь и постоянную поддержку.

Глубокую признательность автор выражает своей первой учительнице математики Майсаре Сафиевне Хафизовой, которая во многом повлияла на выбор профессионального пути.


