\intro

%
% Используемые далее команды определяются в файле common.tex.
%

% Актуальность работы
\actualitysection
\actualitytext

На практике такие задачи нередко имеют большой объем входных данных, что приводит к необходимости решать системы нелинейных уравнений большой размерности с использованием многопроцессорных вычислителей и технологий распараллеливания.


% Степень разработанности темы исследования
\developmentsection
\developmenttext

% Цель диссертационной работы
\objectivesection
\objectivetext

% Методология и методы исследования
\methodssection
\methodstext

% Научная новизна
\noveltysection
\noveltytext

% Практическая ценность
\valuesection
\valuetext

% Результаты и положения, выносимые на защиту
%\resultssection
%\resultstext

% Апробация работы
\approbationsection
\approbationtext

% Публикации
\pubsection
\pubtext

% Личный вклад автора
\contribsection
\contribtext

% Структура и объем диссертации
\structsection
\structtext

Автор искренне признателен своему научному руководителю доктору физико-математических наук, ведущему научному сотруднику ИММ УрО РАН Елене Николаевне Акимовой.

Автор выражает глубокую благодарность за постановку ряда проблем, внимание к работе, полезные замечания и обсуждения член-корреспонденту РАН, главному научному сотруднику ИММ УрО РАН Владимиру Васильевичу Васину.

Автор также благодарен своим коллегам из ИММ УрО РАН за постоянную помощь и поддержку: зав. Отделом некорректных задач, анализа и приложений доктору физико-математических наук А.Л. Агееву, В.Е. Мисилову, П.А. Чистякову, Г.Г. Скорику.

Отдельные слова благодарности автор выражает своей семье за любовь, постоянную и всестороннюю поддержку.

Глубокую признательность автор выражает своей первой учительнице математики --- Майсаре Сафиевне Хафизовой, которая во многом повлияла на выбор профессионального пути.


