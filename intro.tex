\intro

%
% Используемые далее команды определяются в файле common.tex.
%

% Актуальность работы
\actualitysection
\actualitytext

Решение практических задач требует обработки больших объемов данных. Для уменьшения времени счета используются параллельные алгоритмы и многопроцессорные вычислители.

%На практике такие задачи нередко имеют большой объем входных данных, что приводит к необходимости решать системы нелинейных уравнений большой размерности. 

% Степень разработанности темы исследования
\developmentsection
\developmenttext

% Цель диссертационной работы
\objectivesection
\objectivetext

% Методология и методы исследования
\methodssection
\methodstext

% Научная новизна
\noveltysection
\noveltytext

% Практическая ценность
\valuesection
\valuetext

% Результаты и положения, выносимые на защиту
%\resultssection
%\resultstext

% Апробация работы
\approbationsection
\approbationtext

% Публикации
\pubsection
\pubtext

% Личный вклад автора
\contribsection
\contribtext

% Структура и объем диссертации
\structsection
\structtext
\newpage

Автор глубоко благодарен своему научному руководителю доктору физико-математических наук, ведущему научному сотруднику ИММ УрО РАН Елене Николаевне Акимовой.

Автор выражает искреннюю признательность за постановку ряда проблем, внимание к работе, полезные замечания и обсуждения член-корреспонденту РАН, главному научному сотруднику ИММ УрО РАН Владимиру Васильевичу Васину.

Автор благодарен своим коллегам из ИММ УрО РАН за помощь и поддержку: заведующему Отделом некорректных задач, анализа и приложений д.ф.-м.н. А.Л. Агееву, к.ф.-м.н.~В.Е.~Мисилову, к.ф.-м.н.~П.А.~Чистякову, \\к.ф.-м.н.~Г.Г.~Скорику.

Отдельные слова благодарности автор выражает своей семье за любовь и постоянную поддержку.

Глубокую признательность автор выражает своей первой учительнице математики Майсаре Сафиевне Хафизовой, которая во многом повлияла на выбор профессионального пути.


