\documentclass[10pt,pdf, mathserif, hyperref={unicode}]{beamer}

% \documentclass[aspectratio=43]{beamer}
% \documentclass[aspectratio=1610]{beamer}
% \documentclass[aspectratio=169]{beamer}

\usepackage{lmodern}

% подключаем кириллицу 
\usepackage[T2A]{fontenc}
\usepackage[utf8]{inputenc}

% отключить клавиши навигации
\setbeamertemplate{navigation symbols}{}

% тема оформления
\usetheme{Berlin}

% цветовая схема
\usecolortheme{seahorse}
% сглаживаем углы
\useinnertheme{rounded}

\useoutertheme{shadow}

\title{The title of presentation}   
\author{proft} 
%\date{\today} 

\begin{document}	
	\small
	\footnotesize
	
\title[\hspace*{55mm}{\insertpagenumber /\pageref{lastpage}}]{{\small\textbf{Регуляризованные методы  на основе схем Гаусса-Ньютона и нелинейных альфа-процессов в приложении к задачам гравиметрии и магнитометрии}}}
\author[\insertlogo{\em{А.Ф. Скурыдина}}%
\hspace*{60mm}]{\textbf{\color{blue}А.Ф. Скурыдина}}
\institute[ИММ УрО РАН, УрФУ]{\textbf{Институт математики и механики УрО РАН\\[1em] Уральский федеральный университет\\[1em]
		620990, Екатеринбург, Россия\\ [1em]  e-mail: afinapal@gmail.com}\\ [1em]
\flushleft{05.13.18, Математическое моделирование, численные методы и комплексы программ}\\ [1em]
{Научный руководитель: д. ф.-м. н., вед. н.с. Е.Н. Акимова}}
\frame{\titlepage}

\begin{frame}{Общая характеристика работы}
	Доклад посвящен регуляризованным итерационным методам решения нелинейных некорректных операторных уравнений.
	
	{\color{blue}Актуальность темы.} Построение итеративно регуляризованных алгоритмов востребовано для решения широкого круга прикладных задач. Так, решение структурных обратных задач гравиметрии и магнитометрии сводится к решению нелинейных интегральных уравнений Урысона первого рода.
	
	{\color{blue}Цель.} Построить новые методы решения нелинейных операторных уравнений, исследовать их сходимость. Реализовать параллельные алгоритмы, провести численные эксперименты.

\end{frame}

%\begin{frame}{Научная новизна}
%	В рамках двухэтапного метода построения регуляризующего алгоритма исследуются на сходимость метод Ньютона, либо нелинейные аналоги альфа-процессов: метод минимальной ошибки (ММО), метод наискорейшего спуска (МНС) и метод минимальных невязок (ММН). Исследуются на сходимость модифицированные варианты методов ММО, МНС, ММН, когда производная оператора вычисляется в начальной точке итераций. %Рассмотрены два случая: оператор задачи является монотонным, либо оператор является конечномерным и его производная имеет неотрицательный спектр.
%	
%	Для решения систем нелинейных интегральных уравнений  с ядром оператора структурной обратной задачи гравиметрии в двуслойной среде предложен покомпонентный метод, основанный на методе Ньютона. 
%	Предложена вычислительная оптимизация метода Ньютона и его модифицированного варианта в виде перехода от плотно заполненной матрицы производной оператора к ленточной в силу особенности строения ядер интегральных операторов задач грави- магнитометрии.
%	Для решения систем нелинейных уравнений  структурных обратных задач гравиметрии в многослойной среде предложен подход на основе метода Левенберга-Марквардта – покомпонентный метод типа Левенберга-Марквардта.
%\end{frame}
\begin{frame}{1.1. Общая постановка задачи}
		Рассматривается нелинейное уравнение с неизвестной функцией $u$
		$$A(u)=f \eqno (1.1)$$
		с нелинейным непрерывно дифференцируемым по Фреше оператором $A$, действующим на паре гильбертовых пространств, для которого обратные операторы $A'(u)^{-1}$, $A^{-1}$ в общем случае разрывны в окрестности решения, что влечет некорректность задачи (1.1).
\end{frame}
\begin{frame}{1.2. Двухэтапный метод аппроксимации решения}
	Рассматривается двухэтапный метод построения регуляризующего алгоритма (РА):
	
	1. Регуляризация по схеме Лаврентьева
	$$A(u)+\alpha(u-u^0)-f_\delta=0, \eqno (1.2)$$
	где $\alpha >0$ --- параметр регуляризации, $\|f-f_\delta\|\le\delta$, $u_0$ --- некоторое приближение к $u_\alpha$;
	
	\smallskip
	2. Применение итерационных алгоритмов аппроксимации регуляризованного решения $u_\alpha$. Рассматриваются методы  на основе метода Гаусса-Ньютона и методы градиентного типа на основе нелинейных $\alpha$-процессов.
\end{frame}
\begin{frame}{\small $\alpha$-процессы (Красносельский и др., 1969)}
	Рассматривается линейное уравнение в гильбертовом пространстве
	$$Ax=y,$$
	с ограниченным, самосопряженным, положительно полуопределенным оператором $A$. %, где $A^{-1}$ может не существовать или быть неограниченным.
	
	\smallskip
	Пусть $\alpha \in [-1, \infty)$ --- некоторое фиксированное вещественное число. Определяется итеррационная последовательность
	$$x^{k+1}=x^k-\frac{\langle A^\alpha\Delta^k, \Delta^k \rangle}{\langle A^{\alpha +1}\Delta^k, \Delta^k\rangle}\Delta^k,$$
	где $\Delta^k=Ax^k-y$.
	\begin{itemize}
		\item при $\alpha=1$ --- метод минимальных невязок (Красносельский и др., 1969),
		\item при $\alpha=0$ --- метод наискорейшего спуска (например, Канторович, Акилов, 1959),
		\item при $\alpha=-1$ --- метод минимальной ошибки.
	\end{itemize}
\end{frame}
\begin{frame}{2.1. Регуляризованный метод Гаусса-Ньютона}
		Итерации производятся по формуле:
		$$ u^{k+1}=u^k-\gamma(A'(u^k)+\bar\alpha I)^{-1}(A(u^k)+\alpha(u^k-u^0)-f_\delta)\equiv{T(u^k)}, \eqno (2.1)$$
		где $A'(u^k)$ --- производная по Фреше оператора $A$ уравнения (1.1), $\gamma$ --- демпфирующий множитель, $\bar{\alpha} \ge \alpha >0 $ --- параметры регуляризации, $T$ --- оператор шага.
		
		\bigskip
		Модифицированный вариант:
		$$ u^{k+1}=u^k-\gamma(A'(u^0)+\bar\alpha I)^{-1}(A(u^k)+\alpha(u^k-u^0)-f_\delta)\equiv{T(u^k)}, \eqno (2.2)$$
		где $A'(u^0)$ --- производная в начальной точке итераций.
\end{frame}
\begin{frame}{2.2. Принципы построения $\alpha$-процессов для нелинейного оператора}
	В литературе {\textbf{\color{red}(например, Васин, Еремин, 2009)}} изложена схема построения итерационных $\alpha$-процессов в случае линейного оператора. В случае нелинейного оператора требуется использовать линеаризацию оператора в точке $u^k$ по формуле Тейлора.
	
	\smallskip
	Итерационный процесс запишем в виде:$$u^{k+1}=u^k-\beta_k(A(u^k)-f_{\delta}),$$ и в случае метода минимальной ошибки (ММО) найдем параметр $\beta$ из условия
	$$\min_{\beta_k}{\|u^k-\beta(A(u^k)-f_{\delta})-z\|^2},$$ где $z$ --- решение уравнения $A'(u^k)z=y^k$, $y^k=f_{\delta}+A'(u^k)u^k-A(u^k)$ (используем разложение Тейлора в точке $u^k$).
	
	{\color{blue} Регуляризованный} метод минимальной ошибки
	$$u^{k+1} =u^k - \frac{\langle B^{-1}(u^k)S_\alpha(u^k), S_\alpha (u^k)\rangle}{\|S_\alpha(u^k)\|^2}S_\alpha(u^k),$$ где $B(u^k)=A'(u^k)+\bar{\alpha}I, \quad S_\alpha(u^k)=A(u^k)+\alpha(u^k-u^0)-f_\delta$.
\end{frame}
\begin{frame}
	Если использовать экстремальные принципы
	$$\min_{\beta}\{\langle A'(u^k)u^{k+1},u^{k+1}\rangle-2\langle u^{k+1},y^k\rangle\},$$
	либо
	$$\min_{\beta}\{\|A'(u^k)(u^k-\beta(A(u^k)-f_{\delta})-y^k\|^2\},$$
	то получаем нелинейные регуляризованные методы наискорейшего спуска (МНС)
	$$u^{k+1} =u^k - \frac{\langle S_\alpha(u^k), S_\alpha (u^k)\rangle}{\langle B(u^k)S_\alpha(u^k), S_\alpha(u^k)\rangle}(A(u^k)+\alpha(u^k-u^0)-f_\delta)$$
	и минимальных невязок (ММН)
	$$u^{k+1} =u^k - \frac{\langle B(u^k)S_\alpha(u^k), S_\alpha (u^k)\rangle}{\|B(u^k)S_\alpha(u^k)\|^2}(A(u^k)+\alpha(u^k-u^0)-f_\delta).$$
	
	\smallskip
	В общем виде итерационную последовательность обозначим
	$$ u^{k+1}=u^k-\gamma\frac{\langle(A'(u^k)+\bar\alpha I)^{\varkappa}S_\alpha(u^k), S_\alpha(u^k)\rangle}{\langle(A'(u^k)+\bar\alpha I)^{\varkappa+1}S_\alpha(u^k), S_\alpha(u^k)\rangle}S_\alpha(u^k)\equiv{T(u^k)}\eqno (2.3)$$
	при соответстующем $\varkappa=-1,0,1$, $\gamma$ --- демпфирующий множитель.
\end{frame}
\begin{frame}{2.3. Модифицированные варианты на основе $\alpha$-процессов}
	Рассматривается случай модификации итерационных методов, когда производная оператора задачи вычисляется в начальной точке итераций:
	%$$ u^{k+1}=u^k-\gamma\frac{<(A'(u^0)+\bar\alpha I)^{\varkappa}S_\alpha(u^k), S_\alpha(u^k)>}{<(A'(u^0)+\bar\alpha I)^{\varkappa+1}S_\alpha(u^k), S_\alpha(u^k)>}S_\alpha(u^k)\equiv{T(u^k)},\eqno (3.3)$$
	%$\varkappa=-1,0,1$.
	\begin{itemize}
		\item ММО: $$u^{k+1}=u^k-\gamma\frac{\langle(A'(u^0)+\bar{\alpha}I)^{-1}S_\alpha(u^k),S_\alpha(u^k)\rangle}{\|S_\alpha(u^k)\|^2}S_\alpha(u^k),\eqno (2.4)$$
		\item МНС:
		$$u^{k+1}=u^k-\gamma\frac{\|S_\alpha(u^k)\|^2}{\langle(A'(u^0))S_\alpha(u^k), S_\alpha(u^k)\rangle}S_\alpha(u^k),\eqno (2.5)$$
		\item ММН:
		$$u^{k+1}=u^k-\gamma\frac{\langle(A'(u^0))S_\alpha(u^k), S_\alpha(u^k)\rangle}{\|(A'(u^0))S_\alpha(u^k)\|^2}S_\alpha(u^k).\eqno (2.6)$$
	\end{itemize}
	
	\smallskip
	Данный прием является более экономичным с точки зрения численной реализации, так как позволяет снизить вычислительные затраты по времени.
\end{frame}
\begin{frame}{}
	\begin{block}{\bf Определение} Усиленное свойство Фейера {\textbf{\color{red}(Васин, Еремин, 2009)}} для оператора $T$ означает, что для некоторого $\nu>0$ выполнено соотношение
	$$\|T(u)-z\|^2\le\|u-z\|^2-\nu\|u-T(u)\|^2,\eqno (2.7)$$
	где $z\in Fix(T)$---множество неподвижных точек оператора $T$. Это влечет для итерационных точек $u^k$, порождаемых процессом $u^{k+1}=T(u^k)$, выполнение неравенства
	$${\|u^{k+1}-z\|}^2\le{\|u^k-z\|}^2-\nu{\|u^k-u^{k+1}\|}^2.\eqno (2.8)$$
	\end{block}
\end{frame}

\begin{frame}{\small Модифицированный метод Ньютона}
	\begin{block}{Васин, 2013}
		Пусть $A: U \to U$ --- непрерывно дифференцируемый оператор с условиями
		$$\|A'(u)\|\le N_1, \quad \|A'(u) - A'(v)\|\le N_2\|u-v\|, \quad \forall u, v \in U,$$
		$A'(u^0)$ --- самосопряженный неотрицательно определенный оператор. Если начальное приближение $u^0$ и параметры $\alpha, \bar{\alpha}, r, N_1, N_2$ удовлетворяют условиям:
		$$\|u^0-u_\alpha\|\le r, \quad 0<\alpha\le \bar{\alpha}, \quad r=\alpha/(6N_2), \quad \bar{\alpha}\ge 3N_1,$$
		то при $\gamma<\alpha\bar{\alpha}/(\alpha+N_1^2)^2$ последовательность $u^k$, порождаемая итерационным процессом, сходится сильно к решению $u_\alpha$ регуляризованного уравнения и выполнено соотношение
		$${\|u^{k+1}-z\|}^2\le{\|u^k-z\|}^2-\nu{\|u^k-u^{k+1}\|}^2.$$
	\end{block}
	\let\thefootnote\relax\let\thefootnote\relax\footnotetext{\footnotesize В. В. Васин, Е. Н. Акимова, А. Ф. Миниахметова. Итерационные алгоритмы ньютоновского типа и их приложения к обратной задаче гравиметрии // Вестник ЮУрГУ, 6:3 (2013). С. 26–37.}
\end{frame}

\begin{frame}{3. Сходимость итерационных процессов с немонотонным оператором уравнения}
	%Условие монотонности оператора $A$ --- очень сильное требование, которое не выполняется во многих важных прикладных задачах, например, в задачах гравиметрии и магнитометрии.
	
	\smallskip
	Цель --- обосновать сходимость рассматриваемых процессов в конечномерном случае, когда оператор $A\colon R^n \to R^n$, для которого матрица $A'(u)$ в некоторой окрестности решения имеет спектр, состоящий из различных неотрицательных собственных значений.
	\begin{block}{\bf Лемма (Васин, 2017)}
		Пусть $n\times n$ матрица $A'(u)$ не имеет кратных собственных значений $\lambda _i$ и числа $\lambda _i$ ($i=1,2,..n$) различны и неотрицательны. Тогда при $\bar\alpha>0$ матрица имеет представление $A'(u)+\bar\alpha I =S(u)\Lambda(u) S^{-1}(u)$ и справедлива оценка
		$$\|(A'(u)+\bar\alpha I)^{-1}\|\le \frac{\mu (S(u))}{\bar\alpha+\lambda_{min}} \le \frac{\mu(S(u))}{\bar\alpha}, \eqno (3.1)$$
		где столбцы матрицы $S(u)$ составлены из собственных векторов матрицы $A'(u)+\bar\alpha I$, $\Lambda$--- диагональная матрица, ее элементы --- собственные значения матрицы $A'(u)+\bar\alpha I$, $$\mu(S(u))=\|S(u)\|\cdot\|S^{-1}(u)\|,$$
		$\mu(S(u))$ --- число обусловленности $S(u)$.
	\end{block}
\end{frame} 
\begin{frame}{3.1. Сходимость регуляризованного метода Ньютона}
	Оператор $A\colon R^n \to R^n$ и $A'(u)$ имеет неотрицательный спектр, удовлетворяющий условиям леммы.
	\begin{block}{\bf Теорема ~3.1.}
		Пусть выполнены условия: \quad $\sup\{\mu(S(u)): u\in S_r(u_\alpha)\}\le\bar S <\infty$, 
		$$\|A(u)-A(v)\|\le N_1\|u-v\|,
		\|A'(u)-A'(v)\|\le N_2\|u-v\|, \quad \forall u, v \in U,$$
		$\|A'(u^0)\| \le N_0\le N_1, \quad \|u^0-u_\alpha\| \le r,$
		
		\smallskip
		$A'(u^0)$---симметричная матрица, $0<\alpha\le\bar\alpha$, $\bar\alpha\ge 4N_0$, $r\le\alpha/8N_2\bar S$.
		
		\smallskip
		Тогда для оператора поправки на итерации
		$$ F(u)=(A'(u)+\bar\alpha I)^{-1}(A(u)+\alpha(u-u^0)-f_\delta) $$
		справедливо неравенство
		$$\|F(u)\|^2\le\mu\langle F(u), u-u_\alpha\rangle, \quad \mu=\frac{4\bar{S}^2(N_1+\alpha)^2}{\alpha\bar{\alpha}}$$
		и для $\nu < 2/\mu$ выполнено свойство сильной фейеровости оператора шага $T$ итераций
	\end{block}
\end{frame}
\begin{frame}
	\begin{block}{\bf Теорема ~3.2.}
		Пусть выполнены условия леммы, а также: \quad $$\sup\{\mu(S(u)): u\in S_r(u_\alpha)\}\le\bar S <\infty,$$ 
		\smallskip
		$$\|A(u)-A(v)\|\le N_1\|u-v\|,\quad
		\|A'(u)-A'(v)\|\le N_2\|u-v\|, \quad \forall u, v \in U,$$
		\smallskip
		$$\|A'(u^0)\| \le N_0\le N_1, \quad \|u^0-u_\alpha\| \le r,$$
		
		\smallskip
		$A'(u^0)$---симметричная матрица, $0<\alpha\le\bar\alpha$, $\bar\alpha\ge 4N_0$, $r\le\alpha/8N_2\bar S$.
		
		\smallskip
		Тогда при
		$\gamma<\frac{\alpha\bar\alpha}{2(N_1+\alpha)^2\bar S^2}$
		оператор шага $T$ процесса (2.1) при установленном $\nu$
		удовлетворяет неравенству (2.7), для итераций $u^k$ справедливо соотношение (2.8) и имеет место сходимость
		$$\lim_{k\to\infty}\|u^k-u_\alpha\|=0.$$
		Если параметр $\gamma$ принимает значение ${\gamma}_{opt}=\frac{\alpha\bar\alpha}{4(N_1+\alpha)^2\bar S^2},$ то справедлива оценка $$\|u^k-u_\alpha\|\le q^k r, \quad q=\sqrt{1-\frac{\alpha ^2}{16(N_1+\alpha)^2\bar S^2}}.$$
	\end{block}
\end{frame}
%\begin{frame}
%	\begin{block}{\bf Замечание~3.1} При доказательстве теоремы достаточно требовать ограниченность величины $$\sup\{\mu(S(u^k)): u^k \in S_r(u_\alpha)\},$$ где $u^k$ --- итерационные точки метода. Причем, при регулярном правиле останова итераций $k(\delta)$, супремум берется по конечному набору номеров $k\le k(\delta)$, что автоматически влечет ограниченность супремума и выполняются оценки аналогичные оценкам в теореме (2.3) при этих значениях $k$. Кроме того, для модифицированного метода Ньютона, в котором производная $A'(u^0)$ вычисляется в фиксированной точке $u^0$, величина $\mu(S(u^0))=\|S(u^0)\|\cdot\|S^{-1}(u^0)\|=\bar S<\infty$.
%	\end{block}
%\end{frame}
\begin{frame}{3.2. Сходимость нелинейных $\alpha$-процессов}
	\begin{block}{\bf Теорема ~3.3.}
		Пусть выполнены условия $\|A(u)-A(v)\|\le N_1\|u-v\|,$ \quad $\|A'(u)-A'(v)\|\le N_2\|u-v\|,$ \quad $ \forall u, v \in U,$
		$	\|A'(u^0)\| \le N_0\le N_1, \quad \|u^0-u_\alpha\| \le r,$ 
		
		\smallskip
		при $u \in S_r(u_\alpha)$ матрица $A'(u)$ имеет спектр, состоящий из неотрицательных различных собственных значений, $A'(u^0)$ --- самосопряженный неотрицательно определенный оператор, параметры удовлетворяют условиям: 
		$$	MMO:\qquad 0<\alpha\le\bar\alpha, \quad r\le\alpha /6\bar SN_2, \quad \bar\alpha \ge N_0 $$
		$$	MHC:\qquad 0<\alpha\le\bar\alpha, \quad r\le\alpha /3N_2,$$
		$$	MMH:\qquad 0<\alpha\le\bar\alpha, \quad r\le\alpha /6N_2.$$
		Тогда для оператора поправки $$F^\varkappa(u)=\frac{\langle(A'(u^k)+\bar\alpha I)^{\varkappa}S_\alpha(u^k), S_\alpha(u^k)\rangle}{\langle(A'(u^k)+\bar\alpha I)^{\varkappa+1}S_\alpha(u^k), S_\alpha(u^k)\rangle}S_\alpha(u^k)$$
		справедливо соотношение
		$$\|F^\varkappa(u)\|^2 \le \mu_\varkappa\langle F^\varkappa(u), u-u_\alpha\rangle, \quad \varkappa=-1,0,1,$$ где
		$$\mu _{-1}=\frac{8\bar S^2(N_1+\alpha)^2}{\alpha\bar\alpha}, \quad 
		\mu _0=\frac{18(N_1+\alpha)^2(N_1+\bar\alpha)}{\alpha\bar\alpha ^2}, \quad $$
		$$\mu _1=\frac{18(N_1+\alpha)^2(N_1+\bar\alpha)^4}{\alpha\bar\alpha ^5}.\eqno (3.2)$$
	\end{block}
\end{frame}
\begin{frame}
	\begin{block}{\bf Теорема ~3.4.}
		Пусть выполнены условия леммы. 
		
		Тогда при $\gamma_\varkappa<2/\mu _\varkappa$, $\varkappa=-1,0,1$, где значения $\mu _k$ определяются соотношениями (3.2), последовательности ${u^k}$, порождаемые процессом (2.3) при $\varkappa=-1,0,1$, сходятся к $u_\alpha$, т.е., $$\lim_{k\to\infty}\|u^k-u_\alpha\|=0,$$ а при $
		\gamma{_\varkappa^{opt}}=\frac{1}{\mu_\varkappa}$
		справедлива оценка $$\|u^{k+1}-u_\alpha\|\le q{_\varkappa^k}r,$$ где
		$$q_{-1}=\sqrt{1-\frac{\alpha^2}{64\bar S^2(N_1+\alpha)^2}}, \quad q_0=\sqrt{1-\frac{\alpha^2\bar\alpha^2}{36(N_1+\alpha)^2(N_1+\bar\alpha)^2}},$$
		$$q_1=\sqrt{1-\frac{\alpha^2\bar\alpha^6}{36(N_1+\alpha)^2(N_1+\bar\alpha)^6}}.$$
	\end{block}
\end{frame}
\begin{frame}{\small\textbf{Случай немонотонного оператора уравнения для модифицированных методов}}
	\begin{block}{Теорема 3.5}
		Пусть выполнены условия $$\|A(u)-A(v)\|\le N_1\|u-v\|,  \quad \forall u, v \in U,$$ 
		$$\|A'(u)-A'(v)\|\le N_2\|u-v\|, \quad \forall u, v \in U,$$
		$$\|A'(u^0)\| \le N_0\le N_1, \quad \|u^0-u_\alpha\| \le r,$$ 
		
		\smallskip
		$A'(u^0)$ --- самосопряженный оператор, спектр которого состоит из неотрицательных различных собственных значений, параметры удовлетворяют условиям:
		$$0\le\alpha\le\bar{\alpha}, \quad r=\alpha/6N_2, \quad \bar{\alpha}\ge N_0.$$
		
		Тогда для оператора поправки на итерации
		$$F^\varkappa(u)=\frac{\langle(A'(u^0)+\bar\alpha I)^{\varkappa}S_\alpha(u^k), S_\alpha(u^k)\rangle}{\langle(A'(u^0)+\bar\alpha I)^{\varkappa+1}S_\alpha(u^k), S_\alpha(u^k)\rangle}S_\alpha(u^k)$$ имеет место неравенство
		$$\|F^\varkappa(u)\|^2\le\frac{8(N_1+\alpha)^2}{3\alpha\bar{\alpha}}\langle F^\varkappa(u), u-u_\alpha\rangle,$$
		где $ \quad \varkappa=-1, 0, 1,$ для модифицированных вариантов ММО, МНС и ММН соответственно.
	\end{block}
\end{frame}
\begin{frame}
	\begin{block}{Теорема 3.6}
		Пусть выполнены условия теоремы 3.5. Тогда при
		$$\gamma _\varkappa <\frac{2}{\mu _\varkappa}\quad (\varkappa=-1,0,1)$$
		для последовательности $\{u^k\}$, порождаемой модифицированным $\alpha$-процессом при соответствующем $\varkappa$, имеет место сходимость $$\lim_{k\to\infty}\|u^k-u_\alpha\|=0, $$ а при 
		$\gamma{_\varkappa^{opt}}=\frac{1}{\mu_\varkappa}$
		справедлива оценка $$\|u^k-u_\alpha\|\le q{_\varkappa^k}r,$$ где
		$$q^\varkappa=\sqrt{1-\frac{9\alpha^2}{64(N_1+\alpha)^2}}$$
	\end{block}
\end{frame}
\begin{frame}
	\begin{block}{\bf Замечание~3.1} Предложенный подход к получению оценок скорости сходимости итерационных процессов полностью переносится на случай, когда спектр матрицы $A'(u^k)$, состоящий из различных вещественных значений, содержит набор малых по абсолютной величине отрицательных собственных значений. Пусть $\lambda _1$ --- отрицательное собственное значение с наименьшим модулем $|\lambda_1|$ и $\bar\alpha -|\lambda _1|=\bar\alpha _1<\alpha^*$. Тогда оценка 
		$$\|(A'(u)+\bar\alpha I)^{-1}\|\le \frac{\mu (S(u))}{\bar\alpha+\lambda_{min}} \le \frac{\mu(S(u))}{\bar\alpha},$$ 
		трансформируется в неравенство
		$$\|(A'(u^k)+\bar\alpha I)^{-1}\|\le\frac{\mu(S(u^k))}{\bar\alpha^*}\le\frac{\bar S}{\bar\alpha^*}\eqno(3.2)$$
		Все теоремы остаются справедливыми при замене $\bar\alpha$ на $\bar\alpha^*$.
	\end{block}
\end{frame}

\begin{frame}{4.1. Оптимизация метода Ньютона для обратных задач гравиметрии и магнитометрии}
	Рассмотрим уравнение гравиметрии в декартовой системе координат с осью $z$, направленной вниз 
	\begin{equation*}
	\begin{aligned}
	A(u)=\gamma\Delta\sigma \bigg\{ &\iint_{D} \frac{1}{[(x-x')^2+(y-y')^2+H^2]^{1/2}}dx'dy' \\
	- &\iint_{D} \frac{1}{[(x-x')^2+(y-y')^2+u^2(x',y')]^{1/2}}dx'dy'\bigg\}=\Delta g(x,y),
	\end{aligned} 
	\end{equation*}
	где $\gamma$ --- гравитационная постоянная, равная $6.67\cdot10^{-8}$ см$^3/$г$\cdot c^2$, $\Delta\sigma=\sigma_2-\sigma_1$ --- скачок плотности на поверхности раздела сред, описываемой функцией $u(x,y)$ и подлежащей определению, $\Delta g(x,y)$ --- аномальное гравитационное поле, вызванное отклонением поверхности от асимптотической плоскости $z=H$ для искомого решения $u(x,y)$.
\end{frame}
\begin{frame}
	Обратная задача магнитометрии при тех же условиях записывается в виде уравнения
	\begin{equation*}
	\begin{aligned}
	A(u)=\Delta J  \bigg\{\&\iint_{D} \frac{H}{[(x-x')^2+(y-y')^2+H^2]^{3/2}}dx'dy' \\
	- \&\iint_{D} \frac{u(x',y')}{[(x-x')^2+(y-y')^2+u^2(x',y')]^{3/2}}dx'dy' \bigg\}=\Delta G(x,y), \eqno(5.2)
	\end{aligned} 
	\end{equation*}
	где $\Delta J$ --- усредненный скачок $z$-компоненты вектора намагниченности, $z=H$ --- асимптотическая плоскость, $\Delta G(x,y)$ --- функция, описывающая аномальное поле, $z=u(x,y)$ --- искомая функция, описывающая поверхность раздела сред с различными свойствами намагниченности.
\end{frame}
\begin{frame}
	Прозводная оператора $A$ в точке $u^0(x,y)$ определяется формулой:
	\begin{itemize}
		\item в задаче гравиметрии
		$$ [A'(u^0)]h=\iint_{D} \frac{u^0(x',y')h(x',y')}{[(x-x')^2+(y-y')^2+(u^0(x',y'))^2]^{3/2}}dx'dy',$$
		\item в задаче магнитометрии
		$$ [A'(u^0)]h=\iint_{D} \frac{(x-x')^2+(y-y')^2-2(u^0(x',y'))^2}{[(x-x')^2+(y-y')^2+(u^0(x',y'))^2]^{5/2}}h(x',y')dx'dy'.$$
	\end{itemize}
	Как видим, элементы матрицы $A'(u^0)$ принимают наибольшие значения при малых значениях $(x-x')$ и $(y-y')$.
	\begin{figure}
		\includegraphics[width=0.4\linewidth, height=0.4\textheight]{Matrix}
	\end{figure}
	\centering Рис. 1. Схема матрицы производной оператора $A$
\end{frame}
\begin{frame}
	\begin{block}{Замечание 4.1.}
		В структурных обратных задачах грави- магнитометрии при решении методом Ньютона без существенной потери точности можно не учитывать значения элементов, отстоящих от диагонали далее, чем на  $\beta$-ю часть  размерности матрицы производной, то есть те значения $a_{ij}$, для которых  $j \in \{i-h(\beta),..i+h(\beta)\} $, где $h(\beta)$ --- полуширина ленты матрицы, $i, j$ --- индекс элемента.
	\end{block}
\end{frame}
\begin{frame}{4.2. Покомпонентные методы для решения задач гравиметрии}
	Запишем исходное операторное уравнение (1.1) в виде:
	$$P(u)=A(u)-f,$$
	где $A(u)=\int_{a}^{b}\int_{c}^{d}K(x,y, x',y',u^k(x,y))dxdy$ --- интегральный оператор задачи гравиметрии.
	
	Итерации в методе Ньютона строятся по схеме
	$$A'(u^k)(\Delta u^k)=-(A(u^k)-f),$$ где $\Delta u^k=u^{k+1}-u^k$.
	Для задачи гравиметрии
	$$f\Delta\sigma\int_{a}^{b}\int_{c}^{d}K'_u(x,y, x',y',u^k(x,y))\Delta u^k dxdy=A(u(x',y'))-f(x',y').\eqno(4.3)$$
	В задаче гравиметиии на изменение гравитационного поля в правой части (4.3) наибольшее значение оказывает отклонение от ассимптотической плоскости в точке $(x',y')$. Тогда можем записать
	$$f\Delta\sigma(\Delta u^k)\int_{a}^{b}\int_{c}^{d}K'_u(x,y, x',y',u^k(x,y)) dxdy=A(u(x',y'))-f(x',y').$$
\end{frame}
\begin{frame}
	Итерации осуществляются по схеме:
	$$u^{k+1}=u^k-\gamma\frac{1}{\varPsi(x',y')}(A(u^k)-f),$$
	где $$\varPsi(x',y')=f\Delta\sigma\int_{a}^{b}\int_{c}^{d}K'_u(x,y, x',y',u^k(x,y)) dxdy.$$
	На основе метода Левенберга Марквардта по предположению о локальности изменения гравитационного поля предложен покомпонентный метод типа Левенберга Марквардта:
	$$	u^{k+1}=u^k-\gamma\frac{1}{\varphi(x',y')}[ A'(u^k)^*(A(u^k)-f)],$$
	где
	\begin{equation*}
	\begin{aligned}
	\varphi(x',y')=\bigg[ f\Delta\sigma\int_{a}^{b}\int_{c}^{d}
	\overline{K'_u(x,y, x',y',u^k(x,y))}dxdy\bigg] \notag \\ \times\bigg[f\Delta\sigma\int_{a}^{b}\int_{c}^{d}K'_u(x,y, x',y',u^k(x,y))dxdy\bigg], 
	\end{aligned}
	\end{equation*}
\end{frame}
\begin{frame}
	Регуляризованные варианты покомпонентных методов:
	\begin{itemize}
		\item Ньютона
		$$u^{k+1}=u^k-\frac{1}{\varPsi(x',y')}(A(u^k)+\alpha u-f_\delta),$$
		для решения регуляризованного по Лаврентьеву уравнения,
		\item Левенберга-Марквардта
		$$u^{k+1}=u^k-\gamma\frac{1}{\varphi(x',y')}[ A'(u^k)^*(A(u^k)-f_\delta)+\alpha u]$$
		для решения уравнения, регуляризованного по Тихонову.
	\end{itemize}
\end{frame}
\begin{frame}{\small\textbf{4.3. Приложения к обратным задачам гравиметрии и магнитометрии}}
	Уравнение гравиметрии в декартовой системе координат с осью $z$, направленной вниз 
	\begin{equation*}
	\begin{aligned}
	A(u)=\gamma\Delta\sigma \bigg\{ &\iint_{D} \frac{1}{[(x-x')^2+(y-y')^2+H^2]^{1/2}}dx'dy' \notag\\
	- &\iint_{D} \frac{1}{[(x-x')^2+(y-y')^2+u^2(x',y')]^{1/2}}dx'dy'\bigg\}=\Delta g(x,y),
	\end{aligned} 
	\end{equation*}
	Уравнение магнитометрии имеет вид
	\begin{equation*}\begin{aligned}
	A(u)=\Delta J  \bigg\{&\iint_{D} \frac{H}{[(x-x')^2+(y-y')^2+H^2]^{3/2}}dx'dy' \notag\\
	- &\iint_{D} \frac{u(x',y')}{[(x-x')^2+(y-y')^2+u^2(x',y')]^{3/2}}dx'dy' \bigg\}=\Delta G(x,y),
	\end{aligned} \end{equation*}
\end{frame}
\begin{frame}
	Точное решение уравнения гравиметрии, определяющее поверхность раздела сред, задается формулой %\textbf{\color{red}(Акимова, Мисилов, Скурыдина, Третьяков, 2015)}
	\begin{equation*}
	\hat{u}(x,y)=5-2e^{-(x/10-3.5)^6-(y/10-2.5)^6}-3e^{-(x/10-5.5)^6-(y/10-4.5)^6},
	\end{equation*}
	\begin{center}
		
		\includegraphics[width=7cm, height=4 cm]{Gravy_exact.png}            %frame~2
	\end{center}
	
	\begin{center}
		Рис.2. Модельная поверхность: $D=\{0\le x\le 100, \,\,0\le y\le 110\}$, \\ $  H=5,\,\,\Delta x=\Delta y=1,\,\,\Delta\sigma=0.21$г/см$^3$.
	\end{center}
\end{frame}
\begin{frame}{\small\textbf{Результаты численных расчетов в задаче гравиметрии}}
	Число обусловленности $\mu(A'_n(u_n^k))\approx 10^{17}$, спектр неотрицательный, собственные значения различны. Правило выхода из процесса итераций каждого из методов
	$$\frac{\|\hat{u}_n-\tilde{u}_n\|_{R^n}}{\|\tilde{u}_n\|_{R^n}}\le 10^{-2},$$
	где $\hat{u}_n$ --- точное решение, а $\tilde{u}_n$ --- восстановленное каждым из четырех итерационных методов. Таким образом, точность численного решения, полученного методом Ньютона, $\alpha$-процессамии их модифицированными аналогами, гарантированно не превышала $\varepsilon=10^{-2}$.
\end{frame}
\begin{frame}
	При значениях параметров $\bar\alpha=\alpha=10^{-3}$, $\gamma=1$ представлены результаты численных расчетов, где
	$$\Delta=\frac{\|A_n(\tilde{u}_n)+\alpha(\tilde{u}_n-u^0)-f_n\|_{R^n}}{\|f_n\|_{R^n}},$$
	$N$ --- число итераций в процессе для достижения точности $10^{-2}$, $T$ --- время реализации метода. В позициях для $\Delta$, $N$, $T$ верхняя строка соответствует основным процессам, а нижняя --- их модифицированным вариантам.
	\begin{table}
		\centering
		{\scriptsize Табл.1. Относительные нормы невязок, итерации и времена счета в задаче гравиметрии}
		\begin{tabular}{|p{0.3\textwidth}|p{0.1\textwidth}|p{0.1\textwidth}|p{0.1\textwidth}|p{0.1\textwidth}|}
			\hline
			\rule{0cm}{0.5cm}
			Методы & ММО & МНС & ММН & РМН \\ \hline
			\rule{0cm}{0.5cm}
			\multirow{$\Delta$} & 0.0048 & 0.0020 & 0.0024 & 0.0023	 \\ \cline{2-5} 
			\rule{0cm}{0.5cm}
			&  0.0094   & 0.0019    &  0.0019   &  0.0021   \\ \hline
			\rule{0cm}{0.5cm}
			\multirow{$N$} & 17  &  21   &   20  &  16    \\ \cline{2-5}
			\rule{0cm}{0.5cm}
			&  22   &   23  &  23   &  16   \\ \hline
			\rule{0cm}{0.5cm}
			\multirow{$T$ (сек)}    &  20   &  11   &  14  & 16    \\ \cline{2-5}
			\rule{0cm}{0.5cm}
			& 328    & 7    &  7   &   7  \\ \hline
		\end{tabular}
	\end{table}
\end{frame}

\begin{frame}
	Точное решение уравнения магнитометрии, определяющее поверхность раздела сред, задается формулой %\textbf{\color{red}(Акимова, Мисилов, Дергачев, 2014)}
	\begin{equation*}
	\hat{u}(x,y)=5-2e^{-(x/10-3.5)^6-(y/10-2.5)^6}-3^{-(x/10-5.5)^6-(y/10-4.5)^6},
	\end{equation*}
	\begin{center}
		\includegraphics[width=7cm, height=4 cm]{Magne_exact.png}            %frame~2
	\end{center}
	
	\begin{center}
		Рис.3. Модельная поверхность: $D=\{0\le x\le 100, \,\,0\le y\le 100\}$, \\ $  H=5,\,\,\Delta x=\Delta y=1,\,\,\Delta J=0.4$.
	\end{center}
\end{frame}
\begin{frame}{\small\textbf{Результаты численных расчетов в задаче магнитометрии}}
	Число обусловленности $\mu(A'_n(u_n^k))\approx 1.8\cdot 10^7$, спектр неотрицательный, состоящий из различных собственных значений, $\bar\alpha=0.01$, $\alpha = 0.0001$, $\gamma=1$, $\varepsilon < 10^{-2}$
	
	\begin{table}
		\centering
		{\scriptsize Табл.2. Относительные нормы невязок, итерации и времена счета в задаче магнитометрии}
		
		\smallskip
		\begin{tabular}{|p{0.2\textwidth}|p{0.1\textwidth}|p{0.1\textwidth}|p{0.1\textwidth}|p{0.1\textwidth}|}
			\hline
			\rule{0cm}{0.5cm}
			Методы & ММО & МНС & ММН & РМН \\ \hline
			\rule{0cm}{0.5cm}
			\multirow{$\Delta$} & 0.0636 & 0.0699 & 0.0802 & 0.0368	 \\ \cline{2-5} 
			\rule{0cm}{0.5cm}
			&  0.0569   & 0.0575    &  0.0595   &  0.0369   \\ \hline
			\rule{0cm}{0.5cm}
			\multirow{$N$} & 4  &  4   &   4  &  5    \\ \cline{2-5}
			\rule{0cm}{0.5cm}
			&  4   &   4  &  4   &  5   \\ \hline
			\rule{0cm}{0.5cm}
			\multirow{$T$ (сек)}    &  10   &  6   &  6  & 22    \\ \cline{2-5}
			\rule{0cm}{0.5cm}
			& 5    & 3    &  3   &   3  \\ \hline
		\end{tabular}
	\end{table}
\end{frame}
\begin{frame}{4.3. Результаты численных экспериментов}
	Рассматривается эксперимент по восстановлению границы раздела в двухслойной среде модифицированным методом Ньютона и его оптимизированным вариантом (c использованием ленточной матрицы производной оператора).
	
	Точное решение уравнения гравиметрии, определяющее поверхность раздела сред, задается формулой
	$$\hat{u}(x,y)=5+4e^{-(x/10-3.5)^4-(y/10-2.5)^4}-3e^{-(x/10-5.5)^4-(y/10-4.5)^4},$$
	\centering
		
	\includegraphics[width=\textwidth, height=0.35\textheight]{gravy_kiev2014.png}
	
	Рис.4. Модельная поверхность (слева) и приближенное решение (справа): $D=\{0\le x\le 270, \,\,0\le y\le 300\}$, \\ $  H=5,\,\,\Delta x=\Delta y=0.3,\,\,\Delta\sigma=0.2$ г/см$^3$.
\end{frame}
\begin{frame}
	Точное решение уравнения магнитометрии, определяющее поверхность раздела сред, задается формулой
	$$\hat{u}(x,y)=5-2e^{-(x/10-3.5)^6-(y/10-2.5)^6}-3e^{-(x/10-5.5)^6-(y/10-4.5)^6},$$
	\centering
	\includegraphics[width=\textwidth, height=0.3\textheight]{magne_kiev2014.png}
	
	Рис.5. Модельная поверхность (слева) и приближенное решение (справа): $D=\{0\le x\le 300, \,\,0\le y\le 300\}$, \\ $  H=5,\,\,\Delta x=\Delta y=0.3,\,\,\Delta\J=0.4$ А/м.
\end{frame}
\begin{frame}
	В таблицах приведены результаты расчетов, критерий останова итераций 
	$$\delta=\frac{\|u_e-u_a\|}{\|u_e\|}\le 0.025,$$ параметр регуляризации $\alpha=\bar{\alpha}=10^{-3}$, полуширина ленты матрицы производной $\beta=1/4$ для задачи гравиметрии и $\beta=1/5$ для задачи магнитометрии.
	\begin{table}[]
		\centering
		{\scriptsize Табл.3. Решение обратной задачи гравиметрии в двухслойной среде}
		\begin{tabular}{|p{0.3\textwidth}|l|l|l|}
			\hline
			\multicolumn{1}{|c|}{Метод}        & \multicolumn{1}{c|}{Число итераций} & \multicolumn{1}{c|}{$T_1$} & \multicolumn{1}{c|}{$T_8$} \\ \hline
			Метод Ньютона                      &  3                                  &       22 мин                  &     2 мин 40 сек                 \\ \hline
			Модифицированный метод Ньютона     &         5                           & 32 мин                  & 4 мин                   \\ \hline
			Метод Ньютона с ленточной матрицей &  4                                   & 24 мин                  & 3 мин                   \\ \hline
		\end{tabular}
	\end{table}
	\begin{table}[]
		\centering
		{\scriptsize Табл.4. Решение обратной задачи магнитометрии в двухслойной среде}
		\begin{tabular}{|p{0.25\textwidth}|p{0.15\textwidth}|l|l|}
			\hline
			\multicolumn{1}{|c|}{Метод}        & \multicolumn{1}{c|}{Число итераций} & \multicolumn{1}{c|}{$T_1$} & \multicolumn{1}{c|}{$T_8$} \\ \hline
			Метод Ньютона                      &   3                               &     9 мин                   &      1 мин 30 сек                 \\ \hline
			Модифицированный метод Ньютона     &              6                     & 15 мин 30 сек                & 2 мин                   \\ \hline
			Метод Ньютона с ленточной матрицей &   5                                  & 9 мин 36 сек                & 1 мин 12 сек                 \\ \hline
		\end{tabular}
	\end{table}
\end{frame}
\begin{frame}
	Рассматривается эксперимент по восстановлению границы раздела в двухслойной среде модифицированным методом Ньютона и покомпонентным методом типа Ньютона.
	
	Точное решение уравнения гравиметрии, определяющее поверхность раздела сред, задается формулой
	\begin{equation*}
	\begin{aligned}
	\hat{u}(x,y)=5-3.21e^{-(x/10.13-6.62)^6-(y/9.59-2.93)^6}-2.78e^{-(x/9.89-4.12)^6-(y/8.63-7.435)^6}\\+3.19e^{-(x/9.89-4.82)^6-(y/8.72-4.335)^6},
	\end{aligned} 
	\end{equation*}
	\centering
	\includegraphics[width=\textwidth, height=0.35\textheight]{gravy_kiev2015.png}
	
	Рис.6. Модельная поверхность (слева) и синтетическое гравитационное поле (справа): $D=\{0\le x\le 300, \,\,0\le y\le 330\}$, \\ $  H=5,\,\,\Delta x=\Delta y=0.33,\,\,\Delta\sigma=0.21$ г/см$^3$.
\end{frame}
\begin{frame}
	\includegraphics[width=\textwidth, height=0.35\textheight]{gravy_kiev2015_methods.png}
	
	\centering
	Рис.7. Решение, полученное модиф. методом Ньютона (слева) и решение, полученное покомпонентным методом (справа).
	
	\flushleft
	1. Параметр регуляризации взят $\alpha=\bar{\alpha}=10^{-3}$,
	
	2. Демпфирующий параметр $\gamma=1.8$ для покомпонентного метода Ньютона.
\end{frame}
\begin{frame}
	Критерий останова итераций $$\varepsilon=\frac{\|u_e-u_a\|}{\|u_e\|}=10^{-2},$$
	
	\begin{table}[]
		\centering
		{\scriptsize Табл.5. Сравнение ММН и ПМН}
		\begin{tabular}{|p{0.25\textwidth}|p{0.15\textwidth}|c|c|c|}
			\hline
			\multicolumn{1}{|c|}{\textbf{Метод}} & \textbf{Число итераций} & \textbf{$T_1$ (100x110)} & \textbf{$T_1$ (300x330)} & \textbf{$T_8$ (300x330)} \\ \hline
			Модифицированный метод Ньютона       & 16                      & 21 сек          & 25 мин          & 3 мин 25 сек    \\ \hline
			Покомпонентный метод типа Ньютона    & 21                      & 13 сек          & 11 мин          & 1 мин 38 сек    \\ \hline
		\end{tabular}
	\end{table}
	{\textcolor{blue}{Замечание.} Размерность матрицы производной в модифицированном методе Ньютона $A'(u^0)$ при сетке $300\times330$ составляет $99000\times99000$}.
\end{frame}

\begin{frame}
	Рассматривается эксперимент по восстановлению границы раздела в двухслойной среде покомпонентным методом типа Ньютона и покомпонентным методом типа Левенберга-Марквардта.
	
	Точное решение уравнения гравиметрии, определяющее поверхность раздела сред, задается формулой
	$$\hat{u}(x,y)=5+4e^{-(x/10-3.5)^4-(y/10-2.5)^4}-3e^{-(x/10-5.5)^4-(y/10-4.5)^4},$$
	\centering
	\includegraphics[width=0.6\textwidth, height=0.35\textheight]{gravyf_nov2014.png}
		
	Рис.8. Cинтетическое гравитационное поле (справа): $D=\{0\le x\le 512, \,\,0\le y\le 512\}$, \\ $  H=5,\,\,\Delta x=0.17 \Delta y=0.19,\,\,\Delta\sigma=0.2$ г/см$^3$.
\end{frame}
\begin{frame}
	\includegraphics[width=\textwidth, height=0.35\textheight]{gravy_nov2014.png}
	
	\centering
	Рис.9. Точное решение (слева) приближенное решение (справа).
	
	\flushleft
	1. Параметр регуляризации взят $\alpha=\bar{\alpha}=10^{-3}$,
	
	2. Демпфирующий параметр $\gamma=1.6$.
	3. Критерий останова --- достижение относительной погрешности 0.025.
	\begin{table}[]
		\centering
		{\scriptsize Табл.6. Сравнение ПМН и ПМЛМ}
		\begin{tabular}{|p{0.25\textwidth}|c|c|c|}
			\hline
			\multicolumn{1}{|c|}{\textbf{Метод}}            & \textbf{Число итераций} & \textbf{$T_1$ (512x512)} & \textbf{$T_8$ (512x512)} \\ \hline
			Покомпонентный метод типа Ньютона               & 3                       & 11 мин 27 сек   & 1 мин 44 сек    \\ \hline
			Покомпонентный метод типа Левенберга-Марквардта & 3                       & 2 ч 2 мин       & 16 мин          \\ \hline
		\end{tabular}
	\end{table}
\end{frame}
\begin{frame}
	Рассматривается эксперимент по восстановлению границ раздела сред в многослойной среде (4 слоя с разной плотностью) в задаче гравиметрии на основе квазиреального аномального поля методами: регуляризованный Левенберга-Марквардта и покомпонентный типа Левенберга-Марквардта.
	\begin{figure}
		\centering
		\includegraphics[height=0.3\textheight]{fields}
	\end{figure}
	\centering\textit{Рис.~10. Суммарное гравитационное поле и поле с шумом 15\% (мГал).}
	\begin{figure}
		\centering
		\includegraphics[height=0.2\textheight]{exact_hor}
	\end{figure}
	\centering\textit{Рис.~11. Точные решения $u_0(x,y)$, $u_1(x,y)$, $u_2(x,y)$.}
	
	$H_1=8$ км, $H_2=15$ км и $H_3=30$ км. Скачки плотности $\Delta\sigma_1=0.2$ г/см$^3$, $\Delta\sigma_2=0.1$ г/см$^3$, $\Delta\sigma_3=0.1$ г/см$^3$. Шаги сетки $\Delta x=2$ км, $\Delta y=3$ км.
\end{frame}
\begin{frame}
	\begin{figure}
		\centering
		\includegraphics[height=0.2\textheight]{levmar}
	\end{figure}
	\centering\textit{Рис.~12. Границы, восстановленные РМЛМ $\tilde{u}_0(x,y)$, $\tilde{u}_1(x,y)$, $\tilde{u}_2(x,y)$.}
	\begin{figure}
		\centering
		\includegraphics[height=0.2\textheight]{clm}
	\end{figure}
	\centering\textit{Рис.~13. Границы, восстановленные РПМЛМ $\hat{u}_0(x,y)$, $\hat{u}_1(x,y)$, $\hat{u}_2(x,y)$.}
\end{frame}
\begin{frame}
	\begin{figure}
		\centering
		\includegraphics[height=0.2\textheight]{lm_noise}
	\end{figure}
	\centering\textit{Рис.~14. Границы, восстановленные РМЛМ для данных с шумом $\tilde{u}_0(x,y)$, $\tilde{u}_1(x,y)$, $\tilde{u}_2(x,y)$.}
	\begin{figure}
		\centering
		\includegraphics[height=0.2\textheight]{clm_noise}
	\end{figure}
	\centering\textit{Рис.~15. Границы, восстановленные РПМЛМ для данных с шумом $\hat{u}_0(x,y)$, $\hat{u}_1(x,y)$, $\hat{u}_2(x,y)$.}
\end{frame}
\begin{frame}
	\begin{itemize}
		\item сетки $100\times100$ и $1000\times1000$,
		\item параметр регуляризации $\alpha=10^{-3}$ и демпфирующий множитель $\gamma=1$,
		\item $\varepsilon=\frac{\|A(u_a)+\alpha u_a-f_\delta\|}{\|f_\delta\|}|=0.1$,
		\item относительные погрешности $\delta_i=\|u_a-u_e\|/\|u_e\|$,
		\item $T_1$ --- Intel Xeon (1 ядро),
		\item $T_2$ --- Intel Xeon (8 ядер),
		\item $T_3$ --- NVIDIA Tesla M2050.
	\end{itemize}
	\begin{table} 
		\centering
		\renewcommand{\arraystretch}{1.5} 
		{\scriptsize Табл.7. Результаты расчетов в задаче гравиметрии в многослойной среде}
		\begin{tabular}{|c|c|c|c|c|c|p{0.1\textwidth}|p{0.1\textwidth}|p{0.1\textwidth}|}
			\hline
			Метод & $N$ & $\varepsilon$ & $\delta_1$ & $\delta_2$ & $\delta_3$ & $T_1$ & $T_2$ & $T_3$ \\ \hline \rule{0cm}{0.4cm}
			РМЛМ  & 60 & 0.238 & 0.052 & 0.026 & 0.051 &   4 мин. 6 сек.   &  2 мин. 15 сек.   &   22 сек.  \\ \cline{7-9} 
			&       &                   &                   &                   &                   &    11 ч. 40 мин.  &  1 ч. 25 мин.  &   35 мин.   \\ \hline \rule{0cm}{0.4cm}
			РПМЛМ & 20 & 0.237 & 0.051 & 0.035 & 0.060 &  33 сек.    &  16 сек.    &  2 сек.    \\ \cline{7-9} 
			&                   &                   &                   &                   &        &   1 ч. 12 мин.   &  10 мин.    &  3 мин.   \\ \hline
		\end{tabular}
	\end{table}
\end{frame}
\begin{frame}{Применение инструментов параллельного программирования и оптимизация вычислений}
	\begin{itemize}
		\item на основе методов и алгоритмов разработаны параллельные алгоритмы для вычислений на многоядерных процессорах и графических ускорителей;
		\item используются инструменты: OpenMP, CUDA;
		\item библиотеки Intel MKL, Cublas;
		\item при больших размерах сеток вычисления производились "на лету": необходимый элемент матрицы вычисляется в момент умножения его на элемент вектора.
	\end{itemize}
\end{frame}

%\begin{frame}{Защищаемые положения}
%	1. Сформулированы и доказаны теоремы, устанавливающие сильную фейеровость оператора шага итераций методов:
%	\begin{itemize}
%		\item 	метод Ньютона;
%		\item	метод минимальной ошибки и его модифицированный вариант;
%		\item	метод наискорейшего спуска и его модифицированный вариант;
%		\item	метод минимальных невязок и его модифицированный вариант.
%	\end{itemize}
%	
%	Доказана сильная фейеровость оператора шага итераций данных методов в случае монотонного оператора задачи и в случае конечномерного оператора с производной, имеющей неотрицательный спектр. Доказывается линейная скорость итерационных процессов. %и устанавливаются регуляризующие свойства двухэтапного метода.
%	
%	2. Предложена вычислительная оптимизация метода Ньютона, которая в задачах гравиметрии и магнитометрии обеспечивает более высокую точность численного решения, а также уменьшает время счета программ.
%	
%\end{frame}
%
%\begin{frame}{Защищаемые положения}
%	3. Предложены покомпонентные методы:
%	\begin{itemize}
%		\item покомпонентный основанный на методе Ньютона для решения нелинейного интегрального уравнения в задаче гравиметрии в двухслойной среде;
%		\item покомпонентный метод типа Левенберга-Марквардта для решения систем нелинейных уравнений  структурных обратных задач гравиметрии в многослойной среде.
%	\end{itemize} Данные методы обладает меньшей вычислительной сложностью в отличие от классических методов Ньютона и Левенберга-Марквардта.
%	
%	 Вычислительные эксперименты показывают, что предложенные метод позволяют существенно уменьшить вычислительную сложность задачи и являются экономичными по потреблению памяти ЭВМ.
%	
%	4. Проведены численные эксперименты для модельных и квазиреальных геофизических данных, разработан комплекс параллельных программ для многоядерных и графических процессоров с использованием технологий OpenMP, CUDA. 
%\end{frame}
	
\begin{frame}{Апробация работы}
	Основные результаты по материалам диссертационной работы докладывались на конференциях:
	
	
	1. XIV и XV Уральская молодежная научная школа по геофизике (Пермь, 2013 г., Екатеринбург 2014 г.);
	
	2. Международная коференция "Параллельные вычислительные технологии" (Ростов-на-Дону, 2014 г., Екатеринбург, 2015 г., Казань, 2017 г.);
	
	3. Международная конференция «Геоинформатика: теоретические и прикладные аспекты» (Киев 2014, 2015, 2016 г.)
	
	4. Международная конференция "Актуальные проблемы вычислительной и прикладной математики" (Новосибирск, 2014 г.)
	
	5. Международный научный семинар по обратным и некорректно поставленным задачам (Москва, 2015 г.)
\end{frame}

\begin{frame}{Публикации в изданиях перечня ВАК}
	\begin{enumerate}
		\item Васин В.В., Акимова Е.Н., Миниахметова А.Ф. Итерационные алгоритмы ньютоновского типа и их приложения к обратной задаче гравиметрии // Вестник ЮУрГУ. Т.6 В.3 (2013), С. 26–37.
		\item Акимова, Е. Н., Мисилов В. Е., Скурыдина А. Ф. Параллельные алгоритмы решения структурной обратной задачи магнитометрии на МВС // Вестник УГАТУ. 2014. Т.18, № 4 (65). C. 206-215.
		\item Akimova E., Miniakhmetova A., Martyshko M. Optimization and parallelization of Newton type methods for solving structurial gravimetry and magnetometry inverse problems // EAGE Geoinformatics 2014 – 13th Intern. Conference on Geoinformatics – Theoretical and Applied Aspects. Kiev, Ukraine. 12–15 May 2014
		%\item Акимова, Е. Н., Мисилов, В. Е., Скурыдина, А. Ф., Третьяков, А. И. Градиентные методы решения структурных обратных задач гравиметрии и магнитометрии на суперкомпьютере “Уран” // Вычислительные методы и программирование, 2015. Т. 16, 155–164.
		\item Akimova E., Skurydina A. A componentwise Newton type method for solving the structural inverse gravity problem // EAGE Geoinformatics 2015 – 14th Intern. Conference on Geoinformatics – Theoretical and Applied Aspects. Kiev, Ukraine. 11–14 May 2015.
		\item Akimova E., Skurydina A. On solving the three-dimensional structural gravity problem for the case of a multilayered medium by the componentwize Levenberg-Marquardt method // EAGE Geoinformatics 2016 – 15th Intern. Conference on Geoinformatics – Theoretical and Applied Aspects. Kiev, Ukraine. 10–13 May 2016.
		\item Васин В.В., Скурыдина, А.Ф. Двухэтапный метод построения регуляризующих алгоритмов для нелинейных некорректных задач // Труды ИММ УрО РАН Т.23 В.1 (2017), С. 57–74.
		
	\end{enumerate}
\end{frame}
	
\begin{frame}{Другие публикации}
	\begin{enumerate}
		\setcounter{enumi}{6}
			\item Мисилов В.Е., Миниахметова А.Ф., Дергачев Е.А. Решение обратной задачи гравиметрии итерационными методами на суперкомпьютере «Уран» // Труды XIV Уральской молодежной научной школы по геофизике. Пермь:
			ГИ УрО РАН.  2013. С. 187–190
			\item Миниахметова А.Ф. Сравнение быстрых методов решения структурной обратной задачи магнитометрии // Труды XV Уральской молодежной научной школы по геофизике. Екатеринбург: ИГФ УрО РАН. 2014. С.~160–162
			\item E. N. Akimova, A. F. Miniakhmetova, V. E. Misilov. Fast stable parallel algorithms for solving gravimetry and magnetometry inverse problems // The International conference "Advanced mathematics, computations and applications – 2014". Institute of Computational Mathematics and Mathematical Geophysics of Siberian Branch of RAS, Novosibirsk, Russia. June 8–11, 2014.
			\item Акимова Е.Н., Мисилов В.Е., Миниахметова А.Ф. Параллельные алгоритмы решения структурной обратной задачи магнитометрии на многопроцессорных вычислительных системах  // Труды межд. конференции «Параллельные вычислительные технологии (ПАВТ’2014)», Ростов-на-Дону, 31 мар. – 4 апр. 2014 г. Челябинск:  ЮУрГУ. 2014. С. 19–29.
	\end{enumerate}
\end{frame}
\begin{frame}{Другие публикации}
	\begin{enumerate}
		\setcounter{enumi}{10}
		\item В.В. Васин, А.Ф. Скурыдина. Регуляризованные модифицированные процессы градиентного типа для нелинейных обратных задач // Международный научный семинар по обратным и некорректно поставленным задачам. Москва, 19 – 21 ноября 2015 г.
		\item Акимова, Е. Н., Мисилов, В. Е., Скурыдина, А. Ф., Третьяков, А. И. Градиентные методы решения структурных обратных задач гравиметрии и магнитометрии на суперкомпьютере “Уран” // Труды межд. конференции «Параллельные вычислительные технологии (ПАВТ’2015)», Екатеринбург, 31 мар. – 2 апр. 2015 г. Челябинск: ЮУрГУ.  2015. С. 8–18.
	\end{enumerate}
\end{frame}

\begin{frame}\label{lastpage}
	\Huge{\centerline{Спасибо за внимание!}}
\end{frame}
\begin{frame}
	\begin{block}{\bf Теорема ~2.1.} 
		Пусть $A$ --- монотонный оператор, для которого выполнены условия 
		$$\|A(u)-A(v)\|\le N_1\|u-v\|, \quad \forall u, v \in U,$$ 
		$$\|A'(u)-A'(v)\|\le N_2\|u-v\|, \quad \forall u, v \in U,$$
		известна оценка для нормы производной в точке $u^0$, т.е.
		$$	\|A'(u^0)\| \le N_0\le N_1, \quad \|u^0-u_\alpha\| \le r \quad
		u^0 \in S_r(u_\alpha),$$ $$r\le \alpha/N_2, \quad 0<\alpha \le \bar\alpha.$$ 
		
		\smallskip
		Тогда для процесса (2.1) c $\gamma=1$ имеет место линейная скорость сходимости метода при аппроксимации единственного решения $u_\alpha$ регуляризованного уравнения (1.2)
		$$\| u^{k}-u_\alpha \| \le q^kr, \quad q=(1-\frac{\alpha}{2\bar\alpha}).$$
	\end{block}
\end{frame}
\begin{frame}
	\begin{block}{\bf Теорема ~2.2.}
		Пусть для монотонного оператора $A$ выполнены условия $$
		\|A(u)-A(v)\|\le N_1\|u-v\|,
		\|A'(u)-A'(v)\|\le N_2\|u-v\|, \quad \forall u, v \in U,$$
		$$\|A'(u^0)\| \le N_0\le N_1, \quad \|u^0-u_\alpha\| \le r, $$
		\smallskip
		$A'(u^0)$ --- самосопряженный оператор, $\|u_\alpha-u^0\|\le r \quad  
		0\le\alpha\le\bar\alpha,\quad\bar\alpha\ge 4N_1,\quad r\le\alpha/8N_2.$
		\vspace{3mm}
		
		Тогда для оператора
		$$ F(u)=(A'(u)+\bar\alpha I)^{-1}(A(u)+\alpha(u-u^0)-f_\delta) $$
		справедлива оценка снизу
		$$<F(u), u-u_\alpha>\ge\frac{\alpha}{4\bar\alpha}{\|u-u_\alpha\|}^2 \quad \forall u \in S_r(u_\alpha).$$
	\end{block}
\end{frame} 
\begin{frame}
	\begin{block}{\bf Теорема ~2.3.}
		Пусть выполнены условия теоремы 2.2. Тогда при
		$\gamma<\frac{\alpha\bar\alpha}{2(N_1+\alpha)^2}$
		оператор шага $T$ процесса (2.1) при
		$$\nu=\frac{\alpha\bar\alpha}{2\gamma(N_1+\alpha)^2}-1$$
		удовлетворяет неравенству (2.3), для итераций $u^k$ справедливо соотношение (2.4) и имеет место сходимость
		$$\lim_{k\to\infty}\|u^k-u_\alpha\|=0.$$
		Если параметр $\gamma$ принимает значение ${\gamma}_{opt}=\frac{\alpha\bar\alpha}{4(N_1+\alpha)^2},$ то справедлива оценка $$\|u^k-u_\alpha\|\le q^k r, \quad q=\sqrt{1-\frac{{\alpha}^2}  {16(N_1+\alpha)^2}}.$$
	\end{block}
\end{frame}
\begin{frame}{3.2. Сходимость нелинейных $\alpha$-процессов}
	\begin{block}{\bf Теорема ~3.1.}
		Пусть для монотонного оператора $A$ выполнены условия $$\|A'(u)-A'(v)\|\le N_2\|u-v\|, \quad \forall u, v \in U,	$$$$\|A'(u^0)\| \le N_0\le N_1, \quad \|u^0-u_\alpha\| \le r,$$ и $A'(u^0)$ --- самосопряженный оператор. 
		
		Кроме того, для ММО параметры $\alpha$, $\bar\alpha$, $r$, $N_2$, $N_0$ удовлетворяют дополнительным соотношениям:
		$$\alpha \le \bar\alpha, \quad r\le \alpha/8N_2, \quad \bar\alpha \ge N_0.$$
		
		Тогда справедливы соотношения
		$$\|F^\varkappa(u)\|^2 \le \mu_\varkappa<F^\varkappa(u), u-u_\alpha>, \quad \varkappa=-1,0,1,$$ где
		$$
		\mu _{-1}=\frac{4(N_1+\alpha)^2}{\alpha\bar\alpha}, \quad \mu _0= \frac{(N_1+\alpha)^2(N_1+\bar\alpha)}{\alpha{\bar\alpha}^2}, $$$$\quad \mu_1= \frac{(N_1+\alpha)^2(N_1+\bar\alpha)^2}{\alpha{\bar\alpha}^3},
		$$
		соответственно для ММО, МНС, ММН.
	\end{block}
\end{frame}
\begin{frame}
	\begin{block}{\bf Теорема ~3.2.}
		Пусть выполнены условия теоремы 3.1. Тогда при
		$$\gamma _\varkappa <\frac{2}{\mu _\varkappa}\quad (\varkappa=-1,0,1)$$
		для последовательности $\{u^k\}$, порождаемой $\alpha$-процессом при соответствующем $\varkappa$, имеет место сходимость $$\lim_{k\to\infty}\|u^k-u_\alpha\|=0, $$ а при 
		$\gamma{_\varkappa^{opt}}=\frac{1}{\mu_\varkappa}$
		справедлива оценка $\|u^k-u_\alpha\|\le q{_\varkappa^k}r,$ где
		$$
		q_{-1}=\sqrt{1-\frac{\alpha^2}{16(N_1+\alpha)^2}}, \quad q_0=\sqrt{1-\frac{\alpha^2\bar\alpha^2}{(N_1+\alpha)^2(N_1+\bar\alpha)^2}}, \quad $$$$q_1=\sqrt{1-\frac{\alpha^2\bar\alpha^4}{(N_1+\bar\alpha)^4}}.
		$$
	\end{block}
\end{frame}
\begin{frame}{\small\textbf{Случай монотонного оператора уравнения}}
	\begin{block}{Теорема 3.5}
		Пусть выполнены условия $$\|A(u)-A(v)\|\le N_1\|u-v\|,  \quad \forall u, v \in U,$$ 
		%$$\|A'(u)-A'(v)\|\le N_2\|u-v\|, \quad \forall u, v \in U,$$
		$$\|A'(u^0)\| \le N_0, \quad \|u^0-u_\alpha\| \le r,$$ 
		
		\smallskip
		Оператор $A$ монотонный, $A'(u^0)$ --- самосопряженный оператор.
		
		\smallskip
		Тогда при $$\gamma=\frac{2\alpha^3}{(N_0+\alpha)(N_1+\alpha)^2}$$ каждая из последовательностей, порождаемых процессами (3.3)-(3.5) сходится к регуляризованному решению $u_\alpha$ и удовлетворяет свойству Фейера.
		Для $$\gamma=\frac{\alpha^3}{(N_0+\alpha)(N_1+\alpha)^2}$$ справедлива оценка
		$$\|u^k-u_\alpha\|\le q^k r,$$ где
		$$q=\sqrt{1-\frac{\alpha^4}{(N_0+\alpha)^2(N_1+\alpha)^2}}.$$
	\end{block}
\end{frame}
\begin{frame}{4. Оценка погрешности двухэтапного метода}
	Согласно \textbf{\color{red}(Tautenhahn, 2002)}, при условии монотонности оператора и истокообразной представимости решения $\hat{u}$ уравнения (1.1)
	$$u^0-\hat{u}=A'(\hat{u})v,\eqno (4.1)$$
	справедлива оценка погрешности регуляризованного решения
	$$\|u_\alpha^{\delta}-\hat{u}\|\le\|u_\alpha^{\delta}-u_\alpha\|+\|u_\alpha-\hat{u}\|\le\frac{\delta}{\alpha}+k_0\alpha,
	\eqno (4.2)$$
	где $k_0=(1+N_2\|v\|/2)\|v\|$, $u_\alpha^{\delta}$, $u_\alpha$ --- решения уравнения (1.2) с возмущенной $f_\delta$ и точной $f$ правой частью уравнения (1.1) соответственно. 
	
	Минимизируя правую часть соотношения (4.2) по $\alpha$, имеем $\alpha(\delta)=\sqrt{\delta /k_0}$, что дает оценку
	$$\|u_{\alpha(\delta)}^{\delta, k}-\hat{u}\|\le 2\sqrt{\delta k_0} \eqno (4.3)$$
	Для итерационных процессов РМН, ММО, МНС, ММН получены оценки вида 
	$$\|u_{\alpha(\delta)}^{\delta, k}-u_{\alpha(\delta)}^{\delta}\|\le q^{k}(\delta)r \eqno (4.4)$$
\end{frame}
\begin{frame}{4.1. Оценка погрешности в случае монотонного оператора}
	Объединяя оценки (4.3), (4.4), приходим к следующему утверждению
	\begin{block}{\bf Теорема~4.1.}
		Пусть для решения $\hat{u}$ уравнения (1.1) с монотонным оператором справедливо условие (4.1) и для метода (3.1) выполнены условия теоремы (3.1). Тогда при выборе числа итераций по правилу
		$$ k(\delta)=\left[\frac{\ln(2\sqrt{k_0\delta}/r)}{\ln q(\delta)}\right]$$
		справедлива оптимальная по порядку оценка погрешности двухэтапного метода
		$$ \|u_{\alpha(\delta)}^{\delta, k}-u_{\alpha(\delta)}^{\delta}\|\le 4\sqrt{k_0 \delta}.$$
	\end{block}
	\let\thefootnote\relax\let\thefootnote\relax\footnotetext{\footnotesize В. В. Васин В.В., Скурыдина, А.Ф. Двухэтапный метод построения регуляризующих алгоритмов для нелинейных некорректных задач // Труды ИММ УрО РАН Т.23 В.1 (2017), С. 57–74.}
\end{frame}
\begin{frame}{4.2. Оценка погрешности в случае оператора c положительным спектром}
	В этой ситуации для двухэтапного алгоритма можно установить оценку для невязки --- основной характеристики точности метода при решении задачи с реальными данными.
	Пусть регуляризованное уравнение разрешимо, тогда для его решения $u_{\alpha(\delta)}^{\delta}$ справедливо соотношение
	$$\|A(u_{\alpha(\delta)}^{\delta})-f_\delta\|=\alpha\|u_{\alpha(\delta)}^{\delta}-u^0\|.\eqno (4.5)$$
	Пусть для некоторой связи $\alpha(\delta)$ $\|u_{\alpha(\delta)}^{\delta}-u^0\|\le m <\infty$, что влечет оценку
	$$\|A(u_{\alpha(\delta)}^{\delta})-f_\delta\|\le\alpha(\delta)m \eqno (4.6)$$
	и сходимость $$\lim_{\delta\to 0}\|A(u_{\alpha(\delta)}^{\delta})-f_\delta\|=0,$$ при $\alpha(\delta)\to 0$, $\delta\to 0$.
	\let\thefootnote\relax\let\thefootnote\relax\footnotetext{\footnotesize В. В. Васин В.В., Скурыдина, А.Ф. Двухэтапный метод построения регуляризующих алгоритмов для нелинейных некорректных задач // Труды ИММ УрО РАН Т.23 В.1 (2017), С. 57–74.}
\end{frame}
\begin{frame}
	Пусть ${u_{\alpha(\delta)}^{\delta, k}}$ --- итерационные точки, полученные одним из методов рассмотренных выше методов. Имеем
	$$\|A(u_{\alpha(\delta)}^{\delta, k})-f_\delta\|\le\|A(u_{\alpha(\delta)}^{\delta, k})-A(u_{\alpha(\delta)}^{\delta})\|+\|A(u_{\alpha(\delta)}^{\delta})-f(\delta)\|\le N_1 r q^{k(\delta)}+\alpha(\delta)m.
	\eqno (4.7)$$
	Выбирая, например, $\alpha(\delta)=\delta^p$ и приравнивая слагаемые в правой части (4.7), получаем правило выбора числа итерации
	$$k(\delta)=\left [\ln(m\delta^p/N)/\ln q(\delta)\right ],$$
	при котором справедлива оценка для невязки двухэтапного метода
	$$\|A(u_{\alpha(\delta)}^{\delta})-f_\delta\|=2m\delta^p.\eqno (4.8)$$
	\begin{block}{Замечание~4.3} Соотношения (4.5)---(4.8) остаются справедливыми для случая, когда матрицы $A'(u^k)$ содержат набор малых отрицательных собственных значений с тем лишь отличием, что в неравенстве (4.7) параметр $q$ во всех методах теперь вычисляется по формулам из раздела 3, в которых параметр $\bar\alpha$ заменен на $\alpha^*$ (см. замечание 3.2).
	\end{block}
	\let\thefootnote\relax\let\thefootnote\relax\footnotetext{\footnotesize В. В. Васин В.В., Скурыдина, А.Ф. Двухэтапный метод построения регуляризующих алгоритмов для нелинейных некорректных задач // Труды ИММ УрО РАН Т.23 В.1 (2017), С. 57–74.}
\end{frame}
\end{document}