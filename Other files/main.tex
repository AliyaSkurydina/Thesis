\documentclass[14pt]{article}
\usepackage[left=25mm, top=20mm, right=10mm, bottom=20mm, nohead, nofoot]{geometry}
\usepackage{setspace}
\linespread{1.3} %\onehalfspacing
\usepackage{ucs}
\usepackage[utf8x]{inputenc}
\usepackage[russian]{babel}
\usepackage{textcomp}
\usepackage{amsfonts,amssymb,amsmath,cite}
\begin{document}
{\center Введение}

\Large
Важнейшими задачами исследования земной коры являются структурные обратные задачи гравиметрии и магнитометрии о восстановлении формы границы раздела между средами с различными плотностями, либо разными значениями вектора намагниченности. Разработка методов решений таких задач представляет интерес как для исследования внутреннего строения Земли, так и для поиска и разведки полезных ископаемых. Они описываются нелинейными операторными уравнениями вида
\\*\begin{math}A(u)=f, \end{math} \\* где $A$ -- нелинейный интегральный оператор Урысона, действующий на паре гильбертовых пространств $U$, $F$.  В данных задачах функция $u$ описывает поверхность раздела сред с разными плотностями, функция $f$ -- гравитационное либо магнитное поле, измеренное на некоторй площади земной поверхности и прошедшее предварительную обработку гравитационных данных. Важно отметить, что найти решение данного уравнения возможно лишь с помощью методов вычислительной математики. Часто при решении геофизических задач приходится работать с зашумленными данными в правой части, в результате которых небольшие изменения в правой части влекут значительные отклонения приближенного решения от точного решения уравнения. Поэтому в данных уравнениях непрерывность оператора $A^{-1}$ в окрестности решения не предполагается, следовательно, эти задачи относятся к классу некорректно поставленных задач. Поэтому для решения исходного уравнения следует использовать итеративно регуляризованные методы. После дискретизации оператора и правой части решение операторного уравнения сводится к системе нелинейных уравнений с большим числом неизвестных, поэтому есть необходимость в параллелизации алгоритмов для многопроцессорных и многоядерных вычислительных систем с целью уменьшения времени счета. 

Ж. Адамар в 1902 г.~\cite{Hadamar1902} впервые определил условия корректности задачи математической физики. Задачи, не отвечающие этим условиям, то есть некорректные,  Ж. Адамар считал лишенными физического смысла. В течение многих лет обратные задачи решались методом подбора, сравнивая вычисленное физическое поле модели с наблюденным. Однако со временем, это мнение претерпело изменения.

Первой работой по теории некорректных задач считается известная работа академика А.Н.Тихонова 1943 г.~\cite{Tikh1943}, в которой он доказал устойчивость некоторых обратных задач при условии принадлежности решения компактному множеству. Также в этой работе он решил одну из актуальных обратных задач разведочной геофизики. В дальнейшем теория некорректных задач оформилось в самостоятельный раздел современной математики. В конце 50-х годов и начале 60-х годов появились работы, посвященные решению некоторых некорректных задач с помощью идей регуляризации, выдающихся отечественных ученых: А.Н.Тихонова, М.М.Лаврентьева, В.К.Иванова. Их исследования в этой области положили начало трем научным школам:  Московской, Сибирской и Уральской.
Началось исследование устойчивых методов решения некорректно-поставленных задач, представляющих собой одно из наиболее актуальных проблем современной математической науки.

В большом цикле работ, выполненных начиная с 1963 года, А.Н. Тихонов сформулировал принцип устойчивого решения некорректно-поставленных задач, ввел понятие регуляризирующего оператора и предложил ряд эффективных методов построения таких операторов, легко реализуемых на ЭВМ ~\cite{Tikh1963_1, Tikh1963_2, TikhGlas1965, TikhArs1986}. Метод, получивший название "метод регуляризации А.Н.Тихонова"\, был применен для решения большого количества как фундаментальных математических, так и актуальных прикладных задач. В частности, тихоновским методом регуляризации были решены задача об отыскании решения интегрального и операторного уравнения первого рода, обратные задачи теории потенциала и теплопроводности. Наряду с Тихоновым, М.М.Лаврентьев изучал методы регуляризации. Ему принадлежит идея замены исходного уравнения близким ему, в некотором смысле, уравнением, для которого задача нахождения решения устойчива к малым изменениям правой части и разрешима для любой правой части ~\cite{Lavr1962}. Были доказаны теоремы сходимости регуляризованного решения к точному ~\cite{Lavr1956}. Основополагающие  результаты  для  интегральных  уравнений  Фредгольма  первого  рода  получены  в  ~\cite{Lavr1959, Lavr1963, LavrVas1966, LavrRomShi1980},  где  для  решения  линейных  интегральных  уравнений  Фредгольма  первого  рода  построены  регуляризирующие  операторы  по  М.М.Лаврентьеву. 

В работах Иванова, выполненных в 1960–1970-е гг., было введено понятие квазирешения ~\cite{Iv1962_2, Iv1963}, были заложены также основы двусторонних оценок регуляризующих алгоритмов ~\cite{Iv1966}, установлены связи между вариационными методами регуляризации, развит единый подход к трактовке линейных некорректных задач в топологических пространствах ~\cite{Iv1967}. 

Однако не все некорректные задачи возможно регуляризовать. Так, российский математик Л.Д.Менихес ~\cite{Menih1978} привел пример интегрального оператора с непрерывным замкнутым ядром, действующего из пространства \( C[0,1] \) в \( L_2[0,1] \), обратная задача для которого нерегуляризуема. Проблемам регуляризуемости также посвящены работы Ю.И.Петунина и А.Н.Пличко ~\cite{PetPlich1980}.

Для построения регуляризующих алгоритмов для решения прикладных задач требуется использовать дополнительную информацию о свойствах искомого решения, заданную в виде равенств и неравенств, характеристик решения, например, свойствами гладкости, естественно вытекающих из физической сущности задачи. Получило развитие построение регуляризующих алгоритмов вариационными методами. А.Б.Бакушинский, Б.Т.Поляк сформулировали общие принципы построения регуляризующих алгоритмов в банаховых пространствах ~\cite{BakPol1974}. Монография А.Б.Бакушинского, А.В.Гончарского ~\cite{BakGon1989} посвящена итеративной регуляризации вариационных неравенств с монотонными операторами, которые единообразно описывают многие постановки задач с априорной информацией. Метод обобщенной невязки был предложен А.В. Гончарским, А.С.Леоновым, А.Г.Яголой ~\cite{GonLeoYag1973}.

Регуляризующие алгоритмы в пространствах функций ограниченной вариации были впервые предложены М.Г.Дмитриевым, В.С.Полещуком ~\cite{DmiPol1972}, И.Ф.Дорофеевым ~\cite{Dor1979}. Далее в работах А.В.Гончарского и В.В.Степанова ~\cite{GonSte1979} А.Л.Агеева ~\cite{Ag1980} была доказана равномерная сходимость приближенных решений. Подход, изложенный в ~\cite{TikhGonSteYag1990} основан на идее двухэтапного алгоритма: построении приближенного решения  исходного операторного уравнения из условия минимизации регуляризованной невязки на априорном множестве, где привлекается информация о неотрицательности, монотонности и выпуклости решения: \\*\begin{math} min\{\| A(u)-f_\delta\| ^2 + \alpha \Omega(u): u\in\Omega, \|f-f_\delta\|\le\delta \}.\end{math} 
\\*На втором этапе для решения корректно поставленной экстремальной задачи применяются методы градиентного типа, линеаризованные методы, или алгоритмы, специально ориентированные на определенный класс априорных ограничений.

В.В.Васиным предложен подход к решению задач с априорной информацией в работах ~\cite{Vas1982, Vas1988} и в монографиях ~\cite{VasAge1993, VasEre2005}, основанный на применении фейеровских отображений для учета априорных ограничений в форме выпуклых неравенств. Термин "фейеровское отображение"\ введен Ереминым в работах ~\cite{Ere1965, Ere1966, Ere1968} в честь венгерского математика Фейера. Отображения, обладающие свойством фейеровости, позволяют строить итерационные процессы с учетом априорных ограничений достаточно общего вида и, в отличие от метрической проекции, допускают эффективную реализацию. На основе класса нелинейных итерационных методов (где оператор шага нелинеен), общее название которых $\alpha$-процессы  ~\cite{KraVaiZab1969}, были предложены регуляризованные методы решения линейных операторных уравнений Фредгольма I рода, возникающих, например, при решении линейных обратных задач гравиметрии. Также Васин доказал сильную сходимость метода Левенберга-Марквардта и его модифицированного варианта для решения регуляризованного по Тихонову нелинейного уравнения. Были приведены численные эксперименты для нелинейной обратной задачи гравиметрии в работах В.В.Васина и Г.Я.Пересторониной~\cite{VasPer2011}, В.В.Васина ~\cite{Vas2012}. Они показали, что основной процесс Левенберга-Марквардта существенно превосходит по точности модифицированный вариант и не требует жестких условий на начальное приближение, но обладает большей вычислительной сложностью, и, следовательно, требует больших затрат машинного времени. 

В данной работе рассматриваются итеративно-регуляризованные методы решения нелинейных обратных задач грави- магнитометрии. Для решения таких задач предложены методы Ньютоновского типа, градиентного типа, являющиеся нелинейными регуляризованными аналогами $\alpha$-процессов и семейство покомпонентных методов типа Ньютона и Левенберга-Марквардта. Регуляризация уравнений производится в форме Лаврентьева. Предложенные методы имеют определенные преимущества. Они обладают хорошей сходимостью, более низкой вычислительной сложностью по сравнению ранее исследованными. Это показано в ряде вычислительных экпериментов.

Цель работы стоит в обосновании сходимости рассматриваемых методов, в реализации соответствующих вычислительных алгоритмов с применением инструментов параллельных вычислений, в проведении численных экспериментов.

Диссертационная работа содержит список обозначений, введение, три главы и список литературы. В работе принята тройная нумерация формул, определений и утверждений: первая цифра обозначает номер главы, вторая -- номер параграфа, третья -- номер объекта в данном параграфе.

Дадим краткое описание содержания диссертации по главам.

В первой главе рассматриваются итеративно-регуляризованные методы Ньютоновского типа для решения нелинейного операторного уравнения первого рода \begin{math}A(u)=f \end{math}. Оператор $A$ действует на паре вещественных гильбертовых пространств $U$, $F$. Обратный оператор $A^{-1}$ в общем случае разрывный. Здесь доказываются теоремы о монотонности оператора шага, устанавливается сильная фейеровость оператора шага в случаях основного метода Ньютона и его модифицированного варианта. Установлено, что в обратных нелинейных задачах гравиметрии и матрица производной оператора $A^{'}(u)$ является почти симметричной матрицей, где элементы, отстоящие все далее от главной диагонали столь малы, что ими можно пренебречь и от плотной матрицы $A^{'}(u)$ перейти к ленточной. Это полезная модификация позволяет экономичнее производить расчеты на ЭВМ с меньшими затратами машинного времени и памяти.
\begin{thebibliography}{2}
\bibitem{Hadamar1902}
Hadamard J. Sur les probl`emes aux derivees partielles et leur signification physique //  Bull. Univ. Princeton. 1902. V. 13.
\bibitem{Tikh1943} Тихонов А.Н. Об устойчивости обратных задач // ДАН СССР. — 1943. — Т. 39, No 5, C. 195–198.
\bibitem{Tikh1963_1} Тихонов А.Н. О решении некорректно поставленных задач.//Докл. АН СССР, 1963, Т. 151, N 3. С. 791-794.
\bibitem{Tikh1963_2} Тихонов А.Н. О регуляризации некорректно поставленных задач // Докл. АН СССР. 1963. Т. 153, № 1. С. 49–52.
\bibitem{TikhGlas1965} Тихонов А.Н., Гласко В.Б. Применение методов регуляризации в нелинейных задачах. // ЖВМ и МФ, 1965, т. 5, N 3. С. 463-473.
\bibitem{TikhArs1986}Тихонов А.Н., Арсенин В.Я. Методы решения некорректных задач // М.: Наука, 1986.
\bibitem{Lavr1956}Лаврентьев М.М. К вопросу об обратной задаче теории потенциала // Доклады Академии наук СССР. - 1956. - Т.106, N 3. - С.389-390.
\bibitem{Lavr1959}Лаврентьев М.М. Об интегральных уравнениях первого рода // Докл. АН СССР. 1959. - Т.127, N 1. - С.31-33.
\bibitem{Lavr1962} Лаврентьев М.М. О некоторых некорректных задачах математической физики. - Новосибирск, 1962. - 92 с.
\bibitem{Lavr1963} Лаврентьев М.М. Об одном классе нелинейных интегральных уравнений // Сибирский математический журнал. - 1963. - Т.4, N 4. - С.837-844.
\bibitem{LavrVas1966} Лаврентьев М.М., В.Г.Васильев. О постановке некоторых некорректных задач математической физики // Сибирский математический журнал. - 1966. - Т.7, N 3. - С.559-576.
\bibitem{LavrRomShi1980}Лаврентьев  М.М.,  Романов  В.Г.,  Шишатский  С.П.  Некорректные  задачи  математической  физики  и  анализа.  М.:  Наука,  1980.
\bibitem{Iv1962_1} Иванов В.К. Интегральные уравнения первого рода и приближенное решение обратной задачи потенциала. // Докл. АН СССР, 1962, т. 142, N 5. С. 998 1000.
\bibitem{Iv1962_2} Иванов В.К. О линейных некорректных задачах. // Докл.АН СССР, 1962, Т. 145. С. 270 272.
\bibitem{Iv1963} Иванов В.К. О некорректно поставленных задачах // Мат. сборник. 1963. T. 61, № 2. С. 211-223.
\bibitem{Iv1966} Иванов В.К. О приближенном решении операторных уравнений первого рода // ЖВМ и МФ, 1966, т. 6, №6. С. 1089–1094 
\bibitem{Iv1967}  Иванов В.К. Об интегральных уравнениях Фредгольма 1 рода. // Дифф. уравн., 1967, т. 3, N 3. С. 410 421.
\\* Иванов В. В., Васин В. В., Танана В. П. Теория линейных некорректных задач и ее приложения // М.: Наука, 1978.

\bibitem{Menih1978} Менихес Л.Д. О регуляризуемости отображений, обратных к интегральным операторам // ДАН СССР. – 1978. – Т. 241. – №2. – С. 625–629.
\bibitem{PetPlich1980}Петунин  Ю. И., Пличко А. Н. Теория характеристик подпространств и ее приложения // Киев Вища шк. 1980. -- 216 с.
\bibitem{BakPol1974} Бакушинский А.Б., Поляк Б.Т. О решении вариационных неравенств // Докл. АН СССР. 1974. - Т.219, N 5. - С.1038-1041.
\bibitem{BakGon1989} Бакушинский А.Б., Гончарский А.В. Итеративные методы решения некорректных задач // М.: Наука, 1978.
\bibitem{GonLeoYag1973} Гончарский А.В., Леонов А.С., Ягола А.Г. Обобщенный принцип невязки // ЖВМ и МФ, 1973, т. 13, N 2. С. 294–302.
\bibitem{DmiPol1972} Дмитриев М.Г., Полещук В.С. О регуляризации одного класса неустойчивых экстремальных задач // // ЖВМ и МФ, 1972,  т. 12, N 5. С. 1316–1318
\bibitem{Dor1979} Дорофеев И.Ф. О решении интегральных уравнений 1 рода в классе функций с ограниченной вариацией // Докл. АН СССР. — 1979. Т. 244, Ш 6. - С. 1303-1311.
\bibitem{GonSte1979}Гончарский А.В., Степанов В.В. О равномерном приближении с ограниченной вариацией некорректно поставленных задач // Докл. АН СССР. 1979. - Т. 248, № 1. - С. 20-22.
\bibitem{Ag1980} Агеев А.Л. Регуляризация нелинейных операторных уравнений на классе функций ограниченной вариации // Журн. вычисл. математики и мат. физики. - 1980. - Т. 20, № 4. - С. 819-826.
\bibitem{AcaVog1994} Acar, R., Vogel, C.R. Analysis of bounded variation penalty methods for ill-posed problems // Inverse Probl. 1994. - Vol. 10, № 6. - P. 1217-1229.
\bibitem{TikhGonSteYag1990} Тихонов А.Н., Гончарский А.В., Степанов В.В., Ягола А.Г. Численные методы решения некорректных задач. М.: Наука, 1990.
\bibitem{Vas1982} Васин В.В. Проксимальный алгоритм с проектированием в задачах выпуклого программирования. –
Свердловск, 1982. – 47 с. Препринт (АН СССР. Уральск. научн. центр, Ин-т матем. и механ.)
\bibitem{Vas1988} Васин В.В. Итерационные методы решения некорректных задач с априорной информацией в гильбертовых пространствах // ЖBМиМФ, 1988. – Т. 28. – №7. – С. 971–980.
\bibitem{VasAge1993} Васин В.В., Агеев А.Л. Некорректные задачи с априорной информацией // Екатеринбург: УИФ Наука, 1993. – 263 с.
\bibitem{VasEre2005} Васин В.В., Еремин И.И. Операторы и итерационные процессы фейеровского типа. Теория и приложения // Москва–Ижевск: Ин-т компьют. исслед. РХД, 2005. – 199 с.
\bibitem{Ere1965} Еремин И.И. Обобщение релаксационного метода Моцкина–Агмона // УМН. – 1965. – Т. 20. – Вып. 2.– С. 183–187.
\bibitem{Ere1966} Еремин И.И. О системах неравенств с выпуклыми функциями в левых частях // Изв. АН СССР. Сер. матем. – 1966. – Т. 30. – Вып. 2. – С. 265–278.
\bibitem{Ere1968} Еремин И.И. Методы фейеровских приближений в выпуклом программировании // Матем. заметки.– 1968. – Т. 3. – Вып. 2. – С. 217–234.
\bibitem{MotSchoe1954} Motzkin T.S., Schoenberg J.J. The relaxation method for linear inequalities // Canad. J. Math. – 1954. – V. 6. – №3. – P. 393–404.
\bibitem{KraVaiZab1969} Красносельский М. А., Вайникко Г. М., Забрейко П. П., Рутицкий Я. Б., Стеценко В. Я. М.: Наука, 1969.
\bibitem{VasPer2011} Васин В.В., Пересторонина Г.Я. Метод Левенберга–Марквардта и его модифицированные варианты для решения нелинейных уравнений с приложением к обратной задаче гравиметрии // Тр. ИММ УрО РАН. 2011. Т. 17. № 2. С. 53–61
\bibitem{Vas2012} Васин В. В. Метод Левенберга–Марквардта для аппроксимации решений нерегулярных операторных уравнений //Автоматика и телемеханика, 2012, № 3, С. 28–38 

\end{thebibliography}
\end{document}