\chapter{Решение уравнений с монотонным оператором}
В первой главе рассматриваются методы решения некорректных задач с нелинейным монотонным оператором. В работе используется двухэтапный подход, где на первом этапе происходит регуляризация по Лаврентьеву, на втором этапе решения задачи применяются итерационные алгоритмы решения регулярной задачи. Первый раздел содержит определения и постановку задачи. Второй раздел главы посвящен вопросам сходимости регуляризованного метода Ньютона. В третьем разделе построены итерационные процессы градиентного типа~--- нелинейные $\alpha$-процессы и доказывается их сходимость к регуляризованному решению. В четвертом разделе приводится оценка погрешности двухэтапного метода. В пятом разделе приводится пример решения модельного нелинейного интегрального уравнения рассмотренными в данной главе итерационными методами.

\newpage
\section{Основные определения и постановка задачи}

Для дальнейшей работы нам понадобятся некоторые определения и обозначения. Пусть $H$ --- гильбертово пространство, нелинейный оператор действует в этом пространстве $A: H \to H$.
\begin{definition}
	Оператор $A$ называется монотонным, если $$\forall u, v\in H\quad\langle A(u)-A(v), u-v\rangle \ge 0.$$
\end{definition}
\begin{definition}
	Оператор $A$ называется равномерно монотонным, если $$\exists \vartheta >0 \quad \forall u, v\in H\quad\langle A(u)-A(v), u-v\rangle \ge \vartheta\|u-v\|^2.$$
\end{definition}
\begin{definition}
	Оператор $A$ называется неотрицательно определенным, если для всех $u\in H$ выполняется неравенство $\langle A(u), u \rangle \ge 0$.
\end{definition}
\begin{definition}
	Оператор $A$ называется дифференцируемым по Фреше в точке $u\in H$, если существует линейный непрерывный оператор \\ $A'(u):H \to H$, такой, что для всех $h\in H$ справедливо
	$$A(u+h)=A(u)+A'(u)h+\omega(u,h),$$
	где $\frac{\|\omega(u,h)\|}{\|h\|}\to 0$ при $\|h\|\to 0$.
	Оператор $A'(u):H \to H$ называется производной Фреше оператора $A$ в точке $u$.
\end{definition}
Рассмотрим линейное уравнение 
\begin{equation}\label{lin_equ}
Bx=y
\end{equation}
с ограниченным самосопряженным положительно определенным оператором $B:H\to H$.
\begin{definition}
	При некотором фиксированном вещественном $\alpha \in [-1, \infty)$ назовем нелинейные итерационные методы решения уравнения $\eqref{lin_equ}$
		$$x^{k+1}=x^k-\frac{\langle B^\alpha\Delta^k, \Delta^k \rangle}{\langle B^{\alpha +1}\Delta^k, \Delta^k\rangle}\Delta^k,\quad \Delta^k=Bx^k-y.$$
		%где $\Delta^k=Ax^k-y$.
		$\alpha$-процессами \cite{KraVayZab1969}.
\end{definition}
При $\alpha=1$ получаем метод минимальных невязок (Красносельский и др., 1969), при $\alpha=0$ получаем метод наискорейшего спуска (Канторович, Акилов, 1959), при $\alpha=-1$ получаем метод минимальной ошибки.

Пусть задан нелинейный оператор $T: H \to H$, $Fix(T)$ --- множество неподвижных точек оператора $T$.
\begin{definition}
	Усиленное свойство Фейера \cite{VasEre2009} для оператора $T$ означает, что существует такое $\nu>0$, при котором выполнено соотношение
	\begin{equation}\label{fejer_prop_uni}
	\forall u\in H \quad \forall z\in Fix(T) \quad{\|T(u)-z\|}^2\le{\|u-z\|}^2-\nu{\|u-T(u)\|}^2.
	\end{equation}
%	где $\forall u\in H$ $\forall z\in Fix(T)$, т.е. $z=T(z)$. 
	Операторы, для которых выполняется усиленное свойство Фейера, называются сильно фейеровскими.
\end{definition}
Рассмотрим итерационный процесс $u^{k+1}=T(u^k)$. Если оператор $T$ является сильно фейеровским, то очевидно, что для всех $k\in\mathbb{N}$ и для всех $z\in Fix(T)$ будет выполнено неравенство
	\begin{equation}\label{fejer_prop_it}
	{\|u^{k+1}-z\|}^2\le{\|u^k-z\|}^2-\nu{\|u^k-u^{k+1}\|}^2.
	\end{equation}

Пусть $\{x^k\}$ --- сходящаяся последовательность приближений некоторого итерационного метода с оператором шага $T: H \to H$ с начальным приближением $x^0\in H$ к некоторому $z\in Fix(T)$.
\begin{definition} 
	Будем говорить, что итерационный метод обладает линейной скоростью сходимости, если $$\exists q\in (0,1)\quad\exists k_0\in\mathbb{N} \quad \forall k\ge k_0: \|x^{k+1}-z\|<q\|x^k-z\|.$$ 
\end{definition}

Перейдем к постановке задачи этой главы.

Рассматривается нелинейное уравнение
\begin{equation}\label{equ1}A(u)=f,\end{equation}
в гильбертовом пространстве $H$ с монотонным непрерывно дифференцируемым по Фреше оператором $A$, для которого обратные операторы $A'(u)^{-1}$, $A^{-1}$ разрывны, что влечет некорректность задачи $\eqref{equ1}$. Правая часть $f$ задана с погрешностью $\delta$: $\|f-f_\delta\|\le\delta$. Требуется построить приближенное решение уравнения $\eqref{equ1}$, устойчивое к погрешности входных данных, то есть регуляризующий алгоритм. 

Используется двухэтапный метод, в котором на первом этапе проводится регуляризация по схеме Лаврентьева
\begin{equation}\label{equ2}A(u)+\alpha(u-u^0)-f_\delta=0,\end{equation}
где $u^0$ --- начальное приближение к решению; а на втором этапе для аппроксимации регуляризованного решения $u_\alpha$ уравнения $\eqref{equ2}$ применяется либо метод Ньютона c параметром шага $\gamma$ и параметрами регуляризации $\bar\alpha$, $\alpha$ (РМН)
\begin{equation}\label{equ_rmn}
u^{k+1}=u^k-\gamma(A'(u^k)+\bar\alpha I)^{-1}(A(u^k)+\alpha(u^k-u^0)-f_\delta)\equiv{T(u^k)},
\end{equation}
%Здесь $\alpha, \bar\alpha$ --- положительные параметры регуляризации, $\gamma>0$ --- демпфирующий множитель (параметр регулировки шага). 
либо нелинейные аналоги $\alpha$-процессов
\begin{equation}\label{equ_alphaproc}
u^{k+1}=u^k-\gamma\frac{\langle (A'(u^k)+\bar\alpha I)^{\varkappa}S_\alpha(u^k), S_\alpha(u^k)\rangle }{\langle(A'(u^k)+\bar\alpha I)^{\varkappa+1}S_\alpha(u^k), S_\alpha(u^k)\rangle }S_\alpha(u^k)\equiv{T(u^k)},
\end{equation}
где $S_\alpha(u^k)=A(u^k)+\alpha(u^k-u^0)-f_\delta$. Здесь $\bar\alpha \ge \alpha >0$ --- параметры регуляризации, $\gamma>0$~--- параметр регулировки шага. При $\varkappa=-1,0$ получаем методы минимальной ошибки (ММО) и наискорейшего спуска (МНС), а при $\varkappa=1$ и самосопряженности оператора $A'(u^k)$ --- метод минимальных невязок (ММН).    
%регуляризованный метод минимальной ошибки (ММО): $$u^{k+1} =u^k - \gamma\frac{\langle B^{-1}(u^k)S_\alpha(u^k), S_\alpha (u^k)\rangle}{\|S_\alpha(u^k)\|^2}S_\alpha(u^k),$$
%регуляризованный метод наискорейшего спуска (МНС):
%$$ u^{k+1} =u^k - \gamma\frac{\langle S_\alpha(u^k), S_\alpha (u^k)\rangle}{\langle B(u^k)S_\alpha(u^k), S_\alpha(u^k)\rangle}(A(u^k)+\alpha(u^k-u^0)-f_\delta)$$
В случае, когда оператор $A'(u^k)$ не является самосопряженным, метод минимальных невязок имеет вид:
\begin{equation}\label{equ_mmn}
u^{k+1} =u^k - \gamma\frac{\langle [A'(u^k)+\bar{\alpha}I]S_\alpha(u^k), S_\alpha (u^k)\rangle}{\|[A'(u^k)+\bar{\alpha}I]S_\alpha(u^k)\|^2}S_\alpha (u^k)\equiv{T(u^k)}.
\end{equation}
Далее в тексте диссертации обозначения при $\varkappa=1$ относятся к $\eqref{equ_mmn}$.

Так как оператор $A$ --- монотонный, то его производная $A'(u)$ --- неотрицательно определенный оператор. Следовательно, операторы $(A'(u^k)+\bar\alpha I)^{-1}$ существуют и ограничены \cite{Bak1976}, процессы $\eqref{equ_rmn}$-$\eqref{equ_alphaproc}$ определены корректно.

\newpage
\section{Метод Ньютона}
Ранее РМН исследовался в работах А. Б. Бакушинского \cite{Bak1976,Bak1992} и совместной работе с А. В. Гончарским \cite{BakGon1989} при значениях параметров $\gamma=1$, $\alpha=\bar{\alpha}=\alpha_k$. Подход основан на том, что априори выбирается подходящим образом последовательность параметров $\alpha_k$ и при некоторых условиях, в том числе на вторую производную оператора $A$ доказывается сходимость итераций к решению уравнения $\eqref{equ1}$ без анализа фейеровости процесса и оценки погрешности. 

В данной работе используется двухэтапный метод, где на первом этапе производится регуляризация методом Лаврентьева, а на втором этапе --- аппроксимация регуляризованного решения итерационными процессами, где вводятся два параметра регуляризации и параметр, отвечающий за регулировку шага. Это позволяет при согласовании параметров регуляризации и регулировки шага получить оценку погрешности двухэтапного метода, оптимальную по порядку. В отличие от подхода А. Б. Бакушинского, выбор управляемых параметров выбирается другим способом, в явном виде. 

Ранее в рамках двухэтапного подхода в работах В.В. Васина \cite{VasAkiMin2013, Vasin2014} исследовался модифицированный метод Ньютона, когда вместо $A'(u^k)$ в $\eqref{equ_rmn}$ используется производная в начальной точке $A'(u^0)$ в ходе всего итерационного процесса, где $A'(u^0)$ --- самосопряженный неотрицательно определенный оператор  
$$
u^{k+1}=u^k-\gamma(A'(u^0)+\bar\alpha I)^{-1}(A(u^k)+\alpha(u^k-u^0)-f_\delta)\equiv{T(u^k)}.
$$
Перейдем к исследованию процесса $\eqref{equ_rmn}$. Пусть выполнены следующие условия:
\begin{equation}\label{cond1.1}
\forall u, v \in S(u^0; R) \quad \|A(u)-A(v)\|\le N_1\|u-v\|,
\end{equation}
\begin{equation}\label{cond1.2}
\forall u, v \in S(u^0; R) \quad \|A'(u)-A'(v)\|\le N_2\|u-v\|,
\end{equation}
где шар $S(u^0; R)$ содержит решения уравнений $\eqref{equ1}$, $\eqref{equ2}$ и известна оценка для нормы производной в точке $u^0$ (начальном приближении), т.е.
\begin{equation}\label{cond1.3}
\|A'(u^0)\|\le N_1.
\end{equation} 
\begin{remark}
	Начальное приближение в неравенстве $\eqref{cond1.3}$ в общем случае не обязано совпадать с $u^0$ в схеме $\eqref{equ2}$. Однако, для простоты изложения, будем считать, что это один и тот же элемент. Кроме того, для монотонного оператора $A$ оператор $A+\alpha I$ --- равномерно монотонный, поэтому при выполнении условия $\ref{cond1.1}$ согласно (\cite{KufFuch1988},~теорема 43.7), регуляризованное уравнение (\ref{equ2}) имеет единственное решение.
\end{remark}

\begin{theorem}\label{teo2.1} Пусть $A$ --- монотонный оператор, для которого выполнены условия $\eqref{cond1.1}$, $\eqref{cond1.2}$, $\|u^0-u_\alpha\| \le r$, $r\le \alpha/N_2$, $0<\alpha \le \bar\alpha$. Тогда для процесса $\eqref{equ_rmn}$ c $\gamma=1$ имеет место линейная скорость сходимости метода при аппроксимации единственного решения $u_\alpha$ регуляризованного уравнения $\eqref{equ2}$
	\begin{equation}\label{nwt_conv}
	\| u^k-u_\alpha \| \le q^kr, \quad q=(1-\frac{\alpha}{2\bar\alpha}).
	\end{equation}
\end{theorem}
\begin{proof} 
Учитывая, что для монотонного оператора $A$ имеет место оценка $\| (A'(u)+\bar\alpha I)^{-1} \| \le 1/\bar\alpha$, а из $\eqref{cond1.2}$ следует справедливость разложения
$$
A(u_\alpha)=A(u^k)+A'(u^k)(u_\alpha-u^k)+\xi, \quad \|\xi\|\le \frac{N_2}{2}\|u_\alpha-u^k\|^2,
$$
где оценка для $\|\xi\|$ следует из леммы 2 \cite{Tre1993} (стр. 339), приходим к соотношению 
$$
u^{k+1}-u_\alpha=u^k-u_\alpha-(A'(u^k)+\bar\alpha I)^{-1}(A(u^k)-A(u_\alpha)+\alpha(u^k-u_\alpha))=u^k- u_\alpha$$ $$-(A'(u^k)+\bar\alpha I)^{-1}(A'(u^k)(u^k-u_\alpha)+\bar\alpha(u^k-u_\alpha)-\xi+(\alpha-\bar\alpha)(u^k-u_\alpha)). $$
Из полученного соотношения вытекает оценка
$$
\|u^{k+1}-u_\alpha\|\le\frac{1}{\bar\alpha}\left(\frac{N_2{\|u^{k}-u_\alpha\|}^2}{2}+(\bar\alpha-\alpha)\|u^k-u_\alpha\|\right)$$
$$\le\left(1-\frac{\alpha}{\bar\alpha}+\frac{N_2}{2\bar\alpha}\|u^k-u_\alpha\|\right)\|u^k-u_\alpha\|.$$
Имея $\|u^0-u_\alpha\|\le r \le \alpha/N_2$ и предполагая $\| u^{k}-u_\alpha \|\le q^kr$, по индукции приходим к оценке $\eqref{nwt_conv}$.
\end{proof}

Важным свойством множества сильно фейеровских операторов является замкнутость относительно операций произведения и взятия выпуклой суммы~\cite{Vas1988}. Располагая итерационными процессами с сильно фейеровским оператором шага и общим множеством неподвижных точек, можно конструировать разнообразные гибридные методы (то есть оператор шага $T$ представляет собой суперпозицию таких операторов шага, при этом сохраняет свойство сильной фейеровости), а также учитывать в итерационном алгоритме априорные ограничения на решение в виде системы линейных или выпуклых неравенств.

Установим усиленное свойство Фейера для оператора шага $T$ в методе $\eqref{equ_rmn}$. Для начала докажем следующую теорему.
\begin{theorem}\label{teo2.2} Пусть для монотонного оператора $A$ выполнены условия $\eqref{cond1.1}$ -- $\eqref{cond1.3}$, $A'(u^0)$ --- самосопряженный оператор, для параметров $\alpha$, $\bar{\alpha}$, $r$ справедливы соотношения 
	\begin{equation}\label{cond2.7}
	0<\alpha\le\bar\alpha,\quad\bar\alpha\ge 4N_1,\quad
	\|u^0-u_\alpha\| \le r, \quad r\le\alpha/8N_2.
	\end{equation}
	Тогда для оператора
	$$ F(u)=(A'(u)+\bar\alpha I)^{-1}(A(u)+\alpha(u-u^0)-f_\delta) $$
	справедлива оценка снизу
	\begin{equation}\label{est2.8}
	\langle F(u), u-u_\alpha\rangle\ge\frac{\alpha}{4\bar\alpha}{\|u-u_\alpha\|}^2 \quad \forall u \in S(u_\alpha;r).
	\end{equation}
\end{theorem}
\begin{proof} Введем обозначение $B(u)=A'(u)+\bar\alpha I$. Принимая во внимание, что $u_\alpha$ --- решение уравнения $\eqref{equ2}$, имеем
$$
\langle F(u), u-u_\alpha\rangle=\langle F(u)-F(u_\alpha), u-u_\alpha\rangle=\alpha\langle B^{-1}(u)(u-u_\alpha), u-u_\alpha\rangle$$ \begin{equation}\label{ineq2.9}+\langle B^{-1}(u)(A(u)-A(u_\alpha)), u-u_\alpha\rangle.
\end{equation}
Учитывая, что $A'(u^0)$ --- самосопряженный и, ввиду монотонности $A$, неотрицательно определенный оператор, для первого слагаемого в правой части равенства $\eqref{ineq2.9}$, получаем
$$\alpha\langle B^{-1}(u)(u-u_\alpha), u-u_\alpha\rangle=\alpha\langle B^{-1}(u^0)(u-u_\alpha), u-u_\alpha\rangle$$ $$+\alpha\langle (B^{-1}(u)-B^{-1}(u^0))(u-u_\alpha), u-u_\alpha\rangle \ge \frac{\alpha}{\bar\alpha+N_1}{\|u-u_\alpha\|}^2$$
$$ - \alpha|\langle B^{-1}(u)(B(u^0)-B(u))B^{-1}(u^0)(u-u_\alpha), u-u_\alpha\rangle| $$
$$\ge \left( \frac{\alpha}{\bar\alpha+N_1}-\frac{\alpha N_2\|u-u^0\|}{{\bar\alpha}^2}\right)\|u-u_\alpha\|^2$$
\begin{equation}\label{ineq2.10}
\ge\left(\frac{\alpha}{\bar\alpha+N_1}-\frac{2\alpha N_2r}{{\bar\alpha}^2}\right)\|u-u_\alpha\|^2,
\end{equation} где использовано неравенство $\|u-u^0\|\le\|u-u_\alpha\|+\|u_\alpha-u^0\|\le 2r$.
Для второго слагаемого в правой части $\eqref{ineq2.9}$ запишем, используя формулу Лагранжа \cite{Tre1993}
$$ \langle B^{-1}(u)(A(u)-A(u_\alpha)), u-u_\alpha\rangle= \langle B^{-1}(u^0)(A(u)-A(u_\alpha)), u-u_\alpha\rangle$$
$$+\langle (B^{-1}(u)-B^{-1}(u^0))(A(u)-A(u_\alpha)), u-u_\alpha\rangle$$
$$=\langle B^{-1}(u^0)\int\limits_0^1 (A'(u_\alpha+\theta(u-u_\alpha))-A'(u^0))d\theta (u-u_\alpha), u-u_\alpha\rangle$$
$$+ \langle B^{-1}(u^0)A'(u^0)(u-u_\alpha), u-u_\alpha\rangle $$
$$+\langle (B^{-1}(u)-B^{-1}(u^0))(A(u)-A(u_\alpha)), u-u_\alpha\rangle$$
$$\ge-\frac{N_2}{\bar\alpha}\int\limits_0^1\|u_\alpha+\theta(u-u_\alpha)-u^0\|d\theta {\|u-u_\alpha\|}^2 $$
$$-\frac{1}{{\bar\alpha}^2}\left ( \|A'(u)-A'(u^0)\|\|A(u)-A(u_\alpha)\|\|(u-u_\alpha)\|\right ) $$$$ \ge - \frac{N_2}{2\bar\alpha} \left ( \|u_\alpha-u^0\|+\|u-u^0\|\right ){\|u-u_\alpha\|}^2 - \frac{N_1 N_2}{{\bar\alpha}^2}\|u-u^0\| {\|u-u_\alpha\|}^2 $$
\begin{equation}\label{ineq2.11}
\ge -\frac{3N_2r}{2\bar\alpha}{\|u-u_\alpha\|}^2-\frac{2rN_1 N_2}{{\bar\alpha}^2}{\|u-u_\alpha\|}^2.\end{equation}

Объединяя $\eqref{ineq2.10}$,$\eqref{ineq2.11}$, приходим к неравенству $$
\langle F(u), u-u_\alpha\rangle\ge\left (\frac{\alpha}{\bar\alpha+N_1}-\frac{2N_2 r \alpha}{{\bar\alpha}^2}-\frac{3N_2r}{2\bar\alpha}-\frac{2rN_1 N_2}{{\bar\alpha}^2}\right ){\|u-u_\alpha\|}^2, $$ откуда с учетом условий $\eqref{cond2.7}$ на параметры $\alpha$, $\bar\alpha$, $r$ приходим к оценке $\eqref{est2.8}$.
\end{proof}
\begin{theorem} \label{teo2.3} 
	Пусть выполнены условия теоремы $\ref{teo2.2}$. Тогда, если
	\begin{equation}\label{ineq2.12}
	\gamma<\frac{\alpha\bar\alpha}{2(N_1+\alpha)^2}
	\end{equation}
	и \begin{equation}\label{eq2.13}
	\nu=\frac{\alpha\bar\alpha}{2\gamma(N_1+\alpha)^2}-1,
	\end{equation}
	то оператор шага $T$ процесса $\eqref{equ_rmn}$ удовлетворяет неравенству $\eqref{fejer_prop_uni}$, для итераций $u^k$ справедливо соотношение $\eqref{fejer_prop_it}$ и имеет место сходимость
	\begin{equation}\label{eq2.14}
	\lim_{k\to\infty}\|u^k-u_\alpha\|=0.
	\end{equation}
	Если параметр $\gamma$ принимает значение \begin{equation}\label{eq2.15}
	\gamma^{\textnormal{opt}}=\frac{\alpha\bar\alpha}{4(N_1+\alpha)^2},
	\end{equation} то справедлива оценка \begin{equation}\label{est2.16}
	\|u^k-u_\alpha\|\le q^k r, \quad q=\sqrt{1-\frac{{\alpha}^2}  {16(N_1+\alpha)^2}}.\end{equation}
\end{theorem}
\begin{proof}
В условиях теоремы справедливо неравенство
\begin{equation}\label{eq2.17}
\|F(u)\|^2\le\|B^{-1}(u)\|^2\|A(u)-A(u_\alpha)+\alpha(u-u_\alpha)\|^2 \le \frac{(N_1+\alpha)^2}{\bar\alpha^2}{\|u-u_\alpha)\|}^2,
\end{equation}
которое вместе с $\eqref{est2.8}$ влечет соотношение
\begin{equation}\label{ineq2.18}
{\|F(u)\|}^2 \le \frac{4(N_1+\alpha)^2}{\alpha\bar\alpha}\langle F(u), u-u_\alpha\rangle.
\end{equation}
Условие $\eqref{fejer_prop_uni}$ на оператор шага $T$ эквивалентно 
\begin{equation}\label{ineq2.19}
{\|F(u)\|}^2 \le \frac{2}{\gamma(1+\nu)}\langle F(u), u-u_\alpha\rangle.
\end{equation}
Сравнивая неравенства $\eqref{ineq2.18}$ и $\eqref{ineq2.19}$, получаем условие $\eqref{ineq2.12}$ для $\gamma$ и выражение $\eqref{eq2.13}$ для $\nu$.

При $u=u^k$ из неравенства $\eqref{fejer_prop_uni}$ вытекает $\eqref{fejer_prop_it}$ и соотношение
$$ \|u^k-T(u^k)\|=\gamma\|F(u^k)\|\to 0, \quad k\to\infty,$$ что вместе с $\eqref{est2.8}$ влечет сходимость $\eqref{eq2.14}$.
Принимая во внимание $\eqref{est2.8}$, $\eqref{eq2.17}$, получим неравенство
$$ {\|u^{k+1}-u_\alpha\|}^2={\|u^k-u_\alpha\|}^2-2\gamma\langle F(u^k), u^k-u_\alpha\rangle+{\gamma}^2\|F(u^k)\|^2 $$
\begin{equation}\label{ineq2.20}
\le \left (1-\gamma\frac{\alpha}{2\bar\alpha}+{\gamma}^2\frac{(N_1+\alpha)^2}{{\bar\alpha}^2}\right )\|u^k-u_\alpha\|^2
\end{equation}
При значениях $\gamma=\gamma^{\textnormal{opt}}$ из $\eqref{eq2.15}$ выражение в круглых скобках неравенства $\eqref{ineq2.20}$ достигает минимума и при $\gamma^{\textnormal{opt}}$ параметр $q$ вычисляется по формуле, представленной в $\eqref{est2.16}$.
\end{proof}

\newpage
\section{Нелинейные аналоги альфа-процессов}

В работе \cite{Vasin2016} для монотонных нелинейных операторных уравнений были предложены регуляризованные модифицированные процессы с самосопряженным оператором $A'(u^0)$ ($-\infty \le \varkappa <\infty$)
$$
u^{k+1}=u^k-\gamma\frac{\langle (A'(u^0)+\bar\alpha I)^{\varkappa}S_\alpha(u^k), S_\alpha(u^k)\rangle }{\langle(A'(u^0)+\bar\alpha I)^{\varkappa+1}S_\alpha(u^k), S_\alpha(u^k)\rangle }S_\alpha(u^k),
$$
которые можно считать нелинейными аналогами классических $\alpha$-процессов для линейных уравнений с самосопряженным оператором.

В диссертационной работе для решения уравнения $\eqref{equ2}$ с монотонным оператором предлагается новый класс итерационных методов --- немодифицированные $\alpha$-процессы $\eqref{equ_alphaproc}$, $\eqref{equ_mmn}$.

Сначала опишем экстремальные принципы, которые используются при их  построении для нелинейного монотонного оператора $A$. Используя разложение Тейлора в точке $u^k$ до второго порядка, получим линейное уравнение
\begin{equation*}
A(u^k)+A'(u^k)(u-u^k)=f_{\delta}.
\end{equation*}
Зададим итерационный процесс в следующем виде
\begin{equation*}
u^{k+1}=u^k-\beta(A(u^k)-f_{\delta})
\end{equation*}
и найдем параметр $\beta$ из условия
\begin{equation}\label{cond3.1}
\min_{\beta}{\|u^k-\beta(A(u^k)-f_{\delta})-z\|^2},
\end{equation}
где $z$ --- решение уравнения $A'(u^k)z=F^k$, $F^k=f_{\delta}+A'(u^k)u^k-A(u^k)$. Заменяя теперь оператор $A(u)$ на $A(u)+\alpha(u-u^0)$, а $A'(u^k)$ на $A'(u^k)+\bar\alpha I$, получаем процесс $\eqref{equ_alphaproc}$ при $\varkappa=-1$ и $\gamma=1$, т.е. нелинейный регуляризованный вариант ММО. Если теперь вместо $\eqref{cond3.1}$ использовать экстремальные принципы
$$\min_{\beta}\{\langle A'(u^k)u^{k+1},u^{k+1}\rangle-2\langle u^{k+1},F(u^k)\rangle\}$$
либо 
\begin{equation}\label{cond3.2}
\min_{\beta}\{\|A'(u^k)(u^k-\beta(A(u^k)-f_{\delta})-F(u^k)\|^2\},
\end{equation}
то получаем после тех же замен нелинейный регуляризованный аналог МНС, т.е. $\eqref{equ_alphaproc}$ при $\varkappa=0$ и $\gamma=1$, либо ММН $\eqref{equ_mmn}$, $\gamma=1$.
%, т.е. $\eqref{equ_alphaproc}$ при $\varkappa=1$, $\gamma=1$ с учетом следующего замечания.

%\begin{remark}
%	Формула $\eqref{equ_alphaproc}$ при $\varkappa=1$ справедлива лишь для самосопряженного оператора $A'(u)$. В общем же случае, знаменатель дроби при $\varkappa=1$ следует заменить на $\|(A'(u)+\alpha I)S_\alpha (u)\|^2$, как это следует из условия минимума задачи $\eqref{cond3.2}$. Это обстоятельство будет учтено во всех выкладках в главах 1, 2.
%\end{remark}

Установим сходимость процессов $\eqref{equ_alphaproc}$, $\eqref{equ_mmn}$ при $\varkappa=-1,0,1$ к решению уравнения $\eqref{equ2}$. Как и прежде, используем следующие обозначения: 
$$B(u)=A'(u)+\bar\alpha I, \quad S_\alpha (u)=A(u)+\alpha(u-u^0)-f_\delta,
$$
а также введем новый оператор
$$F_\varkappa(u)=\beta_\varkappa(u) S_\alpha(u), \quad \beta _\varkappa(u) =\frac{\langle B^\varkappa(u)S_\alpha(u), S_\alpha (u)\rangle}{\langle B^{\varkappa +1}(u)S_\alpha(u), S_\alpha(u)\rangle}$$ 
где при $\varkappa=1$ в $\beta_\varkappa(u)$ следует заменить знаменатель на $\|B(u)S_\alpha(u)\|^2$.% (см. замечание 1.2).
\begin{theorem}\label{teo3.1}
	Пусть для монотонного оператора $A$ выполнены условия $\eqref{cond1.1}$ --- $\eqref{cond1.3}$ и $A'(u^0)$ --- самосопряженный оператор. Кроме того, для ММО параметры $\alpha$, $\bar\alpha$, $r$ удовлетворяют дополнительным соотношениям:
	\begin{equation}\label{cond3.3}
	\alpha \le \bar\alpha, \quad \bar\alpha \ge N_1,\quad \|u^0-u_\alpha\|\le r, \quad r\le \alpha/8N_2,.
	\end{equation}
	Тогда справедливы соотношения
	\begin{equation}\label{ineq3.4}
	\|F_\varkappa(u)\|^2 \le \mu_\varkappa\langle F_\varkappa(u), u-u_\alpha\rangle, \quad \varkappa=-1,0,1,
	\end{equation} где
	\begin{equation}\label{cond3.5}
	\mu _{-1}=\frac{4(N_1+\alpha)^2}{\alpha\bar\alpha}, \quad \mu _0= \frac{(N_1+\alpha)^2(N_1+\bar\alpha)}{\alpha{\bar\alpha}^2}, \quad \mu_1= \frac{(N_1+\alpha)^2(N_1+\bar\alpha)^2}{\alpha{\bar\alpha}^3},
	\end{equation}
	соответственно для ММО, МНС, ММН.
\end{theorem}
\begin{proof} Рассмотрим  ММО, т.е. $\eqref{equ_alphaproc}$ при $\varkappa=-1$. Принимая во внимание монотонность оператора $A$, самосопряженность и неотрицательность $A'(u^0)$ и условия на параметры $\eqref{cond3.3}$, получим (ниже $F_{-1}(u)$, $B^{-1}(u)$, означает $F_\varkappa(u)$, $B^\varkappa(u)$ при $\varkappa=-1$)
$$ \langle F_{-1}(u), u-u_\alpha\rangle=\beta_{-1}(u)\langle A(u)-A(u_\alpha)+ \alpha(u-u_\alpha),u-u_\alpha\rangle\ge\alpha\beta_{-1}(u)\|u-u_\alpha\|^2
$$ $$\ge\alpha\left(\frac{\langle B^{-1}(u^0)S_\alpha(u), S_\alpha(u)\rangle}{\|S_\alpha (u)\|^2}-\frac{\mid\langle (B^{-1}(u)-B^{-1}(u^0))S_\alpha(u), S_\alpha (u)\rangle \mid}{\|S_\alpha (u)\|^2}\right)$$
$$\times \|u-u_\alpha\|^2 \ge\left(\frac{\alpha}{N_1+\bar\alpha}-\alpha\|B^{-1}(u)\|\|B^{-1}(u^0)\|\|A'(u)-A'(u^0)\|\right)\|u-u_\alpha\|^2$$ 
\begin{equation}\label{ineq3.6}
\ge\left(\frac{\alpha}{N_1+\bar\alpha}-\frac{2\alpha N_2r}{\bar\alpha ^2}\right)\|u-u_\alpha\|^2\ge\frac{\alpha}{4\bar\alpha}\|u-u_\alpha\|^2,\end{equation}
где учтено, что $\|u-u^0\|\le\|u-u_\alpha\|+\|u_\alpha-u^0\|\le 2r.$ Кроме того, выполнены неравенства $$\|F_{-1}(u)\|^2=\mid\beta_{-1}(u)\mid^2\|A(u)-A(u_\alpha)+\alpha(u-u_\alpha)\|^2\le(N_1+\alpha)^2\|B^{-1}(u)\|^2\|u-u_\alpha\|^2 $$
\begin{equation}\label{ineq3.7}
\le\frac{(N_1+\alpha)^2}{\bar\alpha^2}\|u-u_\alpha\|^2 .
\end{equation}
Объединяя $\eqref{ineq3.6}$ и $\eqref{ineq3.7}$, получаем
\begin{equation}\label{ineq3.8}
\|F_{-1}(u)\|^2\le\frac{4(N_1+\alpha)^2}{\alpha\bar\alpha}\langle F_{-1}(u), u-u_\alpha\rangle.
\end{equation}

Перейдем к оценке МНС ($\varkappa=0$). Из соотношений
$$\langle F_0(u), u-u_\alpha\rangle=\beta_0(u)\langle A(u)-A(u_\alpha)+\alpha(u-u_\alpha), u-u_\alpha\rangle\ge\alpha\beta_0(u)\|u-u_\alpha\|^2$$ 
\begin{equation}\label{ineq3.9}
=\alpha\frac{\|S_\alpha (u)\|^2}{\langle A'(u)S_\alpha (u),S_\alpha (u)\rangle +\bar\alpha\|S_\alpha (u)\|^2}|u-u_\alpha\|^2\ge \frac{\alpha}{N_1+\bar\alpha}\|u-u_\alpha\|^2,\end{equation}
$$
\|F_0(u)\|^2=\|\beta_0(u)\|^2\|S_\alpha(u)-S_\alpha(u_\alpha)\|^2 $$
\begin{equation}\label{ineq3.10}\le\frac{(N_1+\alpha)^2\|S_\alpha (u)\|^4\|u-u_\alpha\|^2}{\mid\langle A'(u)S_\alpha (u),S_\alpha (u)\rangle +\bar\alpha\|S_\alpha (u)\|^2\mid ^2}\le\frac{(N_1+\alpha)^2}{\bar\alpha^2}\|u-u_\alpha\|^2,
\end{equation}
\vskip 10pt
приходим к неравенству
$$\|F_0(u)\|^2\le\frac{(N_1+\alpha)^2(N_1+\bar\alpha)}{\alpha\bar\alpha^2}\langle F_0(u), u-u_\alpha\rangle.$$

Обратимся теперь к ММН $\eqref{equ_mmn}$. Получим неравенства:
$$\langle F_1(u), u-u_\alpha\rangle\ge\alpha\beta_1(u)\|u-u_\alpha\|^2= \alpha\frac{\langle (A'(u)+\bar\alpha I)S_\alpha (u),S_\alpha (u)\rangle }{\|B(u)S_\alpha (u)\|^2}\|u-u_\alpha\|^2$$
\begin{equation}\label{ineq3.11}
\ge\frac{\alpha\bar\alpha}{\|B(u)\|^2}\|u-u_\alpha\|^2\ge\frac{\alpha\bar\alpha}{(N_1+\bar\alpha)^2}\|u-u_\alpha\|^2,
\end{equation}
$$\|F_1(u)\|\le\frac{(N_1+\alpha)\langle B(u)S_\alpha (u),S_\alpha (u)\rangle }{\|B(u)S_\alpha (u)\|^2}\|u-u_\alpha\|$$
$$\le\frac{(N_1+\alpha)\langle B(u)S_\alpha (u),S_\alpha (u)\rangle }{\langle B(u)S_\alpha (u), B(u)S_\alpha (u)\rangle }\|u-u_\alpha\|$$ 
$$=\frac{(N_1+\alpha)\langle B(u)S_\alpha (u),S_\alpha (u)\rangle \|u-u_\alpha\|}{\|A'(u)S_\alpha (u)\|^2+\alpha\langle A'(u)S_\alpha (u),S_\alpha (u)\rangle +\bar\alpha\langle B(u)S_\alpha (u), S_\alpha (u)\rangle }$$
\begin{equation}\label{ineq3.12}
\le\frac{N_1+\alpha}{\bar\alpha}\|u-u_\alpha\|
\end{equation}
из которых вытекает оценка
$$\|F_1(u)\|^2\le\frac{(N_1+\alpha)^2(N_1+\bar\alpha)^2}{\alpha\bar\alpha^3}\langle F_1(u), u-u_\alpha\rangle.$$

Таким образом, доказана справедливость неравенства $\eqref{ineq3.4}$ при значениях $\mu_\varkappa$ из $\eqref{cond3.5}$.
\end{proof}

\begin{theorem}\label{teo3.2}
	Пусть выполнены условия теоремы $(\ref{teo3.1})$ и $\mu_\varkappa$ принимает значения $\eqref{cond3.5}$. Тогда при всех
	\begin{equation}\label{ineq3.13}
	\gamma <\frac{2}{\mu _\varkappa}\quad (\varkappa=-1,0,1)
	\end{equation}
	для последовательности $\{u^k\}$, порождаемой процессами $\eqref{equ_alphaproc}$, $\eqref{equ_mmn}$ при соответствующем $\varkappa$, имеет место сходимость $$\lim_{k\to\infty}\|u^k-u_\alpha\|=0,$$ а при 
	\begin{equation}\label{eq3.14}
	\gamma^{\textnormal{opt}}=\frac{1}{\mu_\varkappa}
	\end{equation}
	справедлива оценка $$\|u^{k+1}-u_\alpha\|\le q{_\varkappa^k}r,$$ где
	$$q_{-1}=\sqrt{1-\frac{\alpha^2}{16(N_1+\alpha)^2}}, \quad q_0=\sqrt{1-\frac{\alpha^2\bar\alpha^2}{(N_1+\alpha)^2(N_1+\bar\alpha)^2}},$$
	\begin{equation}\label{eq3.15}
	q_1=\sqrt{1-\frac{\alpha^2\bar\alpha^4}{(N_1+\bar\alpha)^4}}.
	\end{equation}
\end{theorem}
\begin{proof} Сопоставляя неравенство $\eqref{ineq2.19}$ при $F(u)=F_\varkappa(u) \quad (\varkappa=-1,0,1)$ с соотношением $\eqref{ineq3.4}$, находим, что при $\gamma$, удовлетворяющем  $\eqref{ineq3.13}$, условие фейеровости выполняется для всех трех процессов. Поэтому сходимость итераций при выполнении условия $\eqref{ineq3.13}$ устанавливается аналогично теореме $(\ref{teo2.3})$, касающейся метода Ньютона. Подставляя в $\eqref{ineq2.20}$ $F_\varkappa(u^k)$ и используя оценки $\eqref{ineq3.7}$, $\eqref{ineq3.8}$ (при $\varkappa=-1$), $\eqref{ineq3.9}$, $\eqref{ineq3.10}$ (при $\varkappa=0$), $\eqref{ineq3.11}$, $\eqref{ineq3.12}$ (при $\varkappa=1$), вычисляем выражение в круглых скобках в правой части неравенства $\eqref{ineq2.20}$ для каждого метода. Минимизируя это выражение по $\gamma$, получаем значение $\gamma^{\textnormal{opt}}$, определяемое формулой $\eqref{eq3.14}$ и вычисляем коэффициенты $q_\varkappa$, которые принимают вид из $\eqref{eq3.15}$.
\end{proof}

\begin{remark}
	Из теорем $\eqref{teo3.1}$, $\eqref{teo3.2}$ следует, что МНС и ММН не требуют выбора близкого к $u_\alpha$ начального приближения $u^0$, то есть в этих случаях имеет место глобальная сходимость итерационных процессов к регуляризованному решению.
\end{remark}

\newpage
\section{Оценка погрешности двухэтапного метода}

Полученные в главе 1 оценки скорости сходимости для итерационных процессов $\eqref{equ_rmn}$--$\eqref{equ_alphaproc}$ и результаты по аппроксимации точного решения уравнения $\eqref{equ1}$ семейством регуляризованных решений $u_\alpha$ позволяют получить оценку погрешности двухэтапного метода.

По теоремам 3.1 и 3.2 из работы \cite{Tau2002} при условии монотонности оператора и истокообразной представимости решения $\hat{u}$ уравнения $\eqref{equ1}$
\begin{equation}\label{cond5.1}
u^0-\hat{u}=A'(\hat{u})v
\end{equation}
справедлива оценка погрешности регуляризованного решения
\begin{equation}\label{est5.2}
\|u_\alpha^{\delta}-\hat{u}\|\le\|u_\alpha^{\delta}-u_\alpha\|+\|u_\alpha-\hat{u}\|\le\frac{\delta}{\alpha}+k_0\alpha,
\end{equation}
где $k_0=(1+N_2\|v\|/2)\|v\|$, $u_\alpha^{\delta}$, $u_\alpha$ -- решения уравнения $\eqref{equ2}$ с возмущенной $f_\delta$ и точной $f$ правой частью уравнения $\eqref{equ1}$ соответственно. Минимизируя правую часть соотношения $\eqref{est5.2}$ по $\alpha$, имеем 
\begin{equation}\label{est5.21}
	\alpha(\delta)=\sqrt{\delta /k_0},
\end{equation}
что дает оценку
\begin{equation}\label{est5.3}
\|u_{\alpha(\delta)}^{\delta}-\hat{u}\|\le 2\sqrt{\delta k_0}
\end{equation}
В данной главе для итерационных процессов $\eqref{equ_rmn}$--$\eqref{equ_alphaproc}$ получены оценки вида 
\begin{equation}\label{est5.4}
\|u_{\alpha(\delta)}^{\delta, k}-u_{\alpha(\delta)}^{\delta}\|\le q^k(\delta)r.
\end{equation}
\newpage
Введем переобозначения. Вместо используемого ранее элемента $u_\alpha$ обозначим через $u_\alpha^{\delta}$ решение уравнения $\eqref{equ2}$ с возмущенной правой частью $f_\delta$, через $u_\alpha$ --- решение уравнения $\eqref{equ2}$ с точной правой частью. Кроме того, вместо $u^k$ используется $u_{\alpha(\delta)}^{\delta, k}$, чтобы подчеркнуть зависимость от параметров $\delta$, $\alpha$.

Объединяя $\eqref{est5.3}$, $\eqref{est5.4}$, приходим к следующему утверждению.
\begin{theorem}\label{teo5.1}
	Пусть для решения $\hat{u}$ уравнения $\eqref{equ1}$ с монотонным оператором справедливо условие $\eqref{cond5.1}$ и для метода $\eqref{equ_alphaproc}$ выполнены условия теоремы~$\ref{teo3.1}$. Тогда при выборе числа итераций по правилу
	\begin{equation}\label{equ5.5}
	k(\delta)=\left[\frac{\ln(2\sqrt{k_0\delta}/r)}{\ln q(\delta)}\right]
	\end{equation}
	справедлива оптимальная по порядку оценка погрешности двухэтапного метода
	\begin{equation}\label{est5.6}
	\|u_{\alpha(\delta)}^{\delta, k}-\hat{u}\|\le 4\sqrt{k_0 \delta},
	\end{equation}
	где $\alpha(\delta)$ определяется из $\eqref{est5.21}$.
\end{theorem}
\proof Объединяя $\eqref{est5.3}$, $\eqref{est5.4}$, получаем
\begin{equation}\label{est5.7}
\|u_{\alpha(\delta)}^{\delta, k}-\hat{u}\|\le\|u_{\alpha(\delta)}^{\delta, k}-u_{\alpha(\delta)}^{\delta}\|+\|u_{\alpha(\delta)}^{\delta}-\hat{u}\|\le rq^k(\delta)+ 2\sqrt{k_0\delta}.
\end{equation}
Приравнивая слагаемые в правой части $\eqref{est5.7}$, получаем выражение для числа итераций $\eqref{equ5.5}$ и оценку $\eqref{est5.6}$. Оптимальность по порядку оценки $\eqref{est5.6}$ устанавливается аналогично \cite{Vasin2015} с использованием методологии оценивания погрешности метода через модуль непрерывности обратного линейного оператора~\cite{Ivanov1974, IvaVasTan2002}.

\newpage
\section{Численные эксперименты}

Устанавливается, что для интегрального уравнения выполнено условие Липшица для оператора задачи и его производной, которое фигурирует в теоремах сходимости. Кроме того, на основе анализа выполненных численных экспериментов показывается работоспособность исследуемых итерационных процессов. Постановка задачи взята из работы \cite{Tau2002}.

Рассматривается ДУ с $x(t)$, $y(t)$, $t\in[0, 1]$ с заданной константой $c_0>0$
%Обратная задача определения закона распределенного роста $x(t)$, $t\in[0, 1]$ с заданной константой $c_0>0$
\begin{equation}\label{taut_1}
\frac{dy}{dt}=x(t)y(t), \quad y(0)=c_0,
\end{equation}
где $x(t), y(t)\in L^2[0,1]$. Интегрируя $\eqref{taut_1}$, приходим к нелинейному операторному уравнению
\begin{equation}\label{taut_2}
F(x)=y,
\end{equation}
где $$[F(x)](t)=c_0 e^{\int_{0}^{t}x(\tau)d\tau}$$
действует из $L^2[0,1]$ в $L^2[0,1]$. В случае, когда правая часть задана с шумом $y^\delta(t)=y(t)e^{\frac{\delta}{5} sin(t/{\delta}^2)}$, при $y^\delta\to y$ в $L^2[0,1]$, величина \\ $\|x-x^\delta\|=\|\frac{1}{5 \delta}cos(t/{\delta}^2)\|\to\infty$ при $\delta\to 0$. Это показывает, что задача $\eqref{taut_2}$ поставлена некорректно. Запишем производную оператора $F$
\begin{equation}\label{taut_3}
[F'(x)h](t)=[F(x)](t)\int_{0}^{t}h(\tau)d\tau.
\end{equation}
Так как в силу $\eqref{taut_1}$, $[F(x)](t)\ge 0$ и $\langle\int_{0}^{t}h(\tau)d\tau, h\rangle\ge 0$, то производная оператора неотрицательно определена $\langle F'(x)h, h\rangle\ge 0$. Оператор $F$ монотонен. Для~проверки условий Липшица для операторов $F$, $F'(x)$ в шаре $S(u^0; R)$ используем оценки: $$\|\int_{0}^{1}h(\tau)d\tau \|\le\|h\|,\quad |e^\lambda-e^\mu|\le|\lambda-\mu|max\{e^\lambda, e^\mu\},$$
$$\|F(u)-F(v)\|\le c_0\|e^{\int_{0}^{1}u(\tau)d\tau}-e^{\int_{0}^{1}v(\tau)d\tau}\|
\le c_0 max\{e^{\|u\|},e^{\|v\|}\}\|u-v\|,$$
$$\|(F'(u)-F'(v))h\|\le\|F(u)-F(v)\|\|h\|\le c_0 max\{e^{\|u\|},e^{\|v\|}\}\|h\|\|u-v\|.$$
Имеем оценку нормы производной оператора в начальном приближении\\ $\|F'(x^0)h\|\le c_0 e^{\|x^0\|}\|h\|,$ т.е. $\|F'(x^0)\|\le c_0 e^{\|x^0\|}$.%, $\|x^0-u_\alpha\|\le r$.

%{\bfseries 1.3.1. Эксперимент без использования шума} 
\subsection*{1.5.1. Эксперимент без использования шума}
 Точное решение $z(t)=t^2$, по точному решению построили правую часть $y(t)$. Начальное приближение $x^0(t)=t^3$, $\bar\alpha=\alpha=10^{-2}$, критерий останова $\frac{\|x^k-z\|}{\|z\|}\le\varepsilon=10^{-2}$, где $x^k$ --- приближение на $k$-й итерации. Выбор начального приближения, близкого к точному решению, обусловлен условиями теорем 1.1, 1.5 для сходимости к регуляризованному решению немодифицированных вариантов методов $\eqref{equ_rmn}$, $\eqref{equ_alphaproc}$, $\eqref{equ_mmn}$. 
 На рис.~\ref{fig:mmn_ch1} изображено восстановленное решение ММН. Точное решение отмечено сплошной линией, начальное приближение отмечено штрихпунктирной линией, решение, полученное методом ММН обозначено пунктирной линией. 
\begin{figure}[h]
	\centering
	\includegraphics[height=12.0cm]{mmn2_ch1}
	\caption{Восстановленное ММН решение.}
	\label{fig:mmn_ch1}
\end{figure}
Ниже в таблице \ref{table1.1} показаны результаты расчетов методами $\eqref{equ_rmn}$, $\eqref{equ_alphaproc}$, $\Delta=\frac{\|F(x^k)+\alpha(x^k-x^0)-y\|}{\|y\|}$ --- относительная норма невязки. 
\begin{table}[H]
	\centering
	%\renewcommand{\arraystretch}{1.5}
	\caption{Результаты для правой части без шума}
	\label{table1.1}
	\begin{tabular}{|l|c|c|c|}
		\hline
	\textbf{Метод}                   & \multicolumn{1}{l|}{\textbf{Параметр шага, $\gamma$}} & \multicolumn{1}{l|}{\textbf{$\Delta$}} & \multicolumn{1}{l|}{\textbf{Число итераций, N}} \\ \hline
		ММО                              & \begin{tabular}[c]{@{}c@{}}0.5\end{tabular}                                       & 0.003                                   & 25                                     \\ \hline
		\multicolumn{1}{|r|}{ММО модиф.} & 0.5                                                                                                           & 0.003                                   & 22                                     \\ \hline
		МНС                              & 0.001                                                                                                         & 0.003                                   & 283                                    \\ \hline
		МНС модиф.                       & \begin{tabular}[c]{@{}c@{}}0.02 (c 1-й итер.), \\ 0.005 (c 30-й итер.),\\   0.002 (c 32-й итер.)\end{tabular} & 0.003                                   & 32                                     \\ \hline
		ММН                              & 1                                                                                                             & 0.003                                   & 32                                     \\ \hline
		ММН модиф.                       & 1                                                                                                             & 0.003                                   & 27                                     \\ \hline
		РМН                              & 1                                                                                                             & 0.003                                   & 26                                     \\ \hline
		РМН модиф.                       & \begin{tabular}[c]{@{}c@{}}0.75\end{tabular}                                       & 0.003                                   & 6                                      \\ \hline
	\end{tabular}
\end{table}

\subsection*{1.5.2. Эксперимент для задачи без использования шума с начальным приближением, далеким от точного решения} 

Точное решение и правая часть такие же, как в эксперименте 1.5.1.  Начальное приближение $x^0(t)=0$, $\bar\alpha=\alpha=10^{-2}$, критерий останова $\frac{\|x^k-z\|}{\|z\|}\le\varepsilon=10^{-1}$, где $x^k$ --- приближение на $k$-й итерации. Выбор начального приближения обусловлен фактом, установленным в статье \cite{Vasin2016}, где для модифицированных вариантов методов $\eqref{equ_alphaproc}$ доказывается глобальная сходимость итерационных процессов. %%На рис.~\ref{fig:mmn_ch1_const} изображено восстановленное решение методом ММН. Точное решение отмечено голубым цветом, начальное приближение --- малиновым, ММН --- зеленым. 
%\begin{figure}
%	\centering
%	\includegraphics[height=6.0cm]{mmn_ch1_const}
%	\caption{Восстановленное ММН решение с нулевым начальным приближением.}
%	\label{fig:mmn_ch1_const}
%\end{figure}
Ниже в таблице \ref{table1.2} показаны результаты расчетов методами $\eqref{equ_rmn}$, $\eqref{equ_alphaproc}$, $\Delta$ --- относительная норма невязки. 
\begin{table}[H]
	\centering
	%\renewcommand{\arraystretch}{1.5}
	\caption{Результаты для правой части без шума, с начальным приближением, равным константе}
	\label{table1.2}
	\begin{tabular}{|l|c|c|c|}
		\hline
		\textbf{Метод}                   & \multicolumn{1}{l|}{\textbf{Параметр шага, $\gamma$}} & \multicolumn{1}{l|}{\textbf{$\Delta$}} & \multicolumn{1}{l|}{\textbf{Число итераций, N}} \\ \hline
		ММО                              & 1                                                     & 0.015                                  & 25                                              \\ \hline
		\multicolumn{1}{|r|}{ММО модиф.} & 0.1                                                   & 0.015                                  & 20                                              \\ \hline
		МНС                              & 0.025                                                 & 0.021                                  & 27                                              \\ \hline
		МНС модиф.                       & 0.025                                                 & 0.024                                  & 24                                              \\ \hline
		ММН                              & 1                                                     & 0.019                                  & 12                                              \\ \hline
		ММН модиф.                       & 1                                                     & 0.016                                  & 8                                               \\ \hline
		РМН                              & 1                                                     & 0.016                                  & 19                                              \\ \hline
		РМН модиф.                       & 0.75                                                  & 0.016                                  & 8                                               \\ \hline
	\end{tabular}
\end{table}

\subsection*{1.5.3. Эксперимент для задачи с возмущенной правой частью с начальным приближением, далеким от точного решения} 

Точное решение такое же, как в эксперименте 1.5.1. Правая часть $y^\delta(t)=y(t)e^{\frac{\delta}{5} sin(t/{\delta}^2)}$, $\delta=0.1$, $\|y-y^{\delta}\|=0.07<\delta$. Начальное приближение $x^0(t)=0$, $\gamma$, $\bar\alpha=1$, $\alpha=10^{-3}$, критерий останова $\frac{\|x^k-z\|}{\|z\|}\le\varepsilon=0.25$, где $x^k$ --- приближение на $k$-й итерации.

{\bfseries\large Вывод.} В эксперименте 1.5.1. относительная погрешность методов ММО, МНС, модифицированных ММО, МНС и РМН в рамках эксперимента была достигнута при выборе $\gamma<1$ в силу теоремы 1.5, тогда как для метода Ньютона немодифицированного варианта сходимость при $\gamma=1$ доказана теоремой~1.1. Для достижения необходимой точности решения модифицированным МНС параметр $\gamma$ потребовалось уменьшать на 30-й итерации $\gamma=0.005$ и на 32-й итерации $\gamma=0.002$.
%Локальная сходимость модифицированного метода Ньютона оговаривается в статье \cite{VasAkiMin2013}, но в данном случае была достигнута требуемая точность, как и для немодифицированных методов, рассматриваемых в данной главе. 
В эксперименте 1.3.2., согласно теореме 1.5, при выборе $\gamma<1$, достигнута относительная погрешность $\varepsilon$ итераций $\eqref{equ_rmn}$, $\eqref{equ_alphaproc}$, $\eqref{equ_mmn}$. 
В последнем эксперименте  достигнута точность $\varepsilon$ за 8--9 итераций, $\Delta\approx 0.04$.

%В статье \cite{VasSkur2017} приводятся оценки погрешности двухэтапного метода для $\|u^{\delta}-\hat{u}\|$ сверху ($u^{\delta}$ --- решение уравнения с возмущенной правой частью, $\hat{u}$ --- решение уравнения $\eqref{equ1}$), устанавливается сходимость $$\lim_{\delta\to 0}\|A(u_{\alpha(\delta)}^{\delta, k})-f_\delta\|=0,$$ при $\alpha(\delta)\to 0$, $\delta\to 0$. 
%
%{\bfseries\large Вывод.} Для задачи с возмущенной правой частью удалось достигнуть точности $\varepsilon$, не превыщающую оценку для $\|u^{\delta}-\hat{u}\|$, относительная норма невязки $\Delta$ уменьшается с каждой итерацией. 

%\section{Выводы к первой главе}