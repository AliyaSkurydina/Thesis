% Общие поля титульного листа диссертации и автореферата
\institution{Институт математики и механики УрО РАН им. Н.Н. Красовского}

\topic{Регуляризованные алгоритмы на основе схем Ньютона, Левенберга -- Марквардта и нелинейных аналогов $\alpha$-процессов для решения нелинейных операторных уравнений}

\author{Скурыдина Алия Фиргатовна}

\specnum{01.01.07}
\spec{Вычислительная математика}
%\specsndnum{01.04.07}
%\specsnd{Физика конденсированного состояния}

\sa{Акимова Елена Николаевна}
\sastatus{доктор физико-математических наук}
%\sasnd{ФИО второго руководителя}
%\sasndstatus{к.~ф.-м.~н., проф.}

%\scon{ФИО консультанта}
%\sconstatus{д.~ф.-м.~н., проф.}
%\sconsnd{ФИО второго консультанта}
%\sconsndstatus{д.~ф.-м.~н., проф.}

\city{Екатеринбург}
\date{\number\year}

% Общие разделы автореферата и диссертации
\mkcommonsect{actuality}{Актуальность темы исследования.}{%
Построение итеративно регуляризованных алгоритмов востребовано для решения широкого круга прикладных некорректно поставленных задач. Так, решение структурных обратных задач гравиметрии и магнитометрии сводится к решению нелинейных интегральных уравнений Урысона первого рода. После дискретизации операторное уравнение сводится к системе нелинейных уравнений с большим числом неизвестных, поэтому есть необходимость в параллелизации алгоритмов для многопроцессорных и многоядерных вычислительных систем с целью уменьшения времени счета. 
}

\mkcommonsect{development}{Степень разработанности темы исследования.}{
Ж. Адамар в 1902~г.~\cite{Hadamar1902} впервые определил условия корректности задачи математической физики. Задачи, не отвечающие этим условиям, то есть некорректные,  Ж. Адамар считал лишенными физического смысла. В течение многих лет обратные задачи решались методом подбора, например, в геофизике, сравнивая вычисленное физическое поле модели с наблюденным. Однако со временем, это мнение претерпело изменения.

Первой работой по теории некорректных задач считается работа академика А.Н. Тихонова 1943 г.~\cite{Tikh1943}, в которой он доказал устойчивость некоторых обратных задач при условии принадлежности решения компактному множеству. Также в этой работе он решил одну из актуальных обратных задач разведочной геофизики. В дальнейшем теория некорректных задач оформилась в самостоятельный раздел современной математики. В конце 50-х годов и начале 60-х годов появились работы, посвященные решению некоторых некорректных задач с помощью идей регуляризации, выдающихся отечественных ученых: А.Н. Тихонова, М.М. Лаврентьева, В.К. Иванова. Их исследования в этой области положили начало трем научным школам:  московской, сибирской и уральской.
Началось исследование устойчивых методов решения некорректно-поставленных задач, представляющих собой одно из наиболее актуальных проблем современной математической науки.

В большом цикле работ, выполненных начиная с 1963 года, А.Н. Тихонов сформулировал принцип устойчивого решения некорректно поставленных задач, ввел понятие регуляризирующего оператора и предложил ряд эффективных методов построения таких операторов, легко реализуемых на ЭВМ ~\cite{Tikh1963_1, Tikh1963_2, TikhGlas1965, TikhArs1986}. Метод, получивший название <<метод регуляризации А.Н. Тихонова>>, был применен для решения большого количества как фундаментальных математических, так и актуальных прикладных задач. В частности, тихоновским методом регуляризации были решены задача об отыскании решения интегрального и операторного уравнения первого рода, обратные задачи теории потенциала и теплопроводности. 

Наряду с Тихоновым, М.М. Лаврентьев изучал методы регуляризации. Ему принадлежит идея замены исходного уравнения близким ему в некотором смысле уравнением, для которого задача нахождения решения устойчива к малым изменениям правой части и разрешима для любой правой части ~\cite{Lavr1962}. Были доказаны теоремы сходимости регуляризованного решения к точному ~\cite{Lavr1956}. Основополагающие  результаты  для  интегральных  уравнений  Фредгольма  первого  рода  получены  в  работах~\cite{Lavr1959, Lavr1963, LavrVas1966, LavrRomShi1980},  где  для  решения  линейных  интегральных  уравнений  Фредгольма  первого  рода  построены  регуляризирующие  операторы  по  М.М. Лаврентьеву. 

В работах Иванова, выполненных в 1960--1970-е годы, было введено понятие квазирешения ~\cite{Iv1962_2, Iv1963}, были заложены также основы двусторонних оценок регуляризующих алгоритмов ~\cite{Iv1966}, установлены связи между вариационными методами регуляризации, развит единый подход к трактовке линейных некорректных задач в топологических пространствах ~\cite{Iv1967}. 

Однако не все некорректные задачи возможно регуляризовать. Так, российский математик Л.Д. Менихес ~\cite{Menih1978} привел пример интегрального оператора с непрерывным замкнутым ядром, действующего из пространства \( C[0,1] \) в \( L_2[0,1] \), обратная задача для которого нерегуляризуема. Проблемам регуляризуемости также посвящены работы Ю.И. Петунина и А.Н. Пличко ~\cite{PetPlich1980}.

Для построения регуляризующих алгоритмов для решения прикладных задач требуется использовать дополнительную информацию о свойствах искомого решения, заданную в виде равенств и неравенств, характеристик решения, например, свойствами гладкости, естественно вытекающих из физической сущности задачи. Получило развитие построение регуляризующих алгоритмов вариационными методами. А.Б. Бакушинский, Б.Т. Поляк сформулировали общие принципы построения регуляризующих алгоритмов в банаховых пространствах \cite{BakPol1974}. Метод обобщенной невязки был предложен А.В. Гончарским, А.С. Леоновым, А.Г. Яголой \cite{GonLeoYag1973}.  Монография А.Б. Бакушинского, А.В. Гончарского \cite{BakGon1989} посвящена итеративной регуляризации вариационных неравенств с монотонными операторами, которые единообразно описывают многие постановки задач с априорной информацией. В работе \cite{Bak1992} А.Б. Бакушинский предложил модификацию метода Гаусса -- Ньютона в духе итеративной регуляризации и исследовал его на сходимость. Метод
Гаусса -- Ньютона был также исследован в работах B. Blaschke, A. Neubauer, O. Scherzer, B. Kaltenbacher, A.G.Ramm \cite{BlaNeuSch1997,KalNeuRam2002}.

Методам решения операторных уравнений первого рода посвящены работы В.П. Тананы~\cite{Tan1977, Tan1997} и монография \cite{Tan1981}. Был предложен метод $L$-регуляризации, представляющий собой разновидность метода Тихонова, расширивший класс регуляризуемых задач \cite{Tan2003_1,Tan2003_2}.

Регуляризующие алгоритмы в пространствах функций ограниченной вариации были впервые предложены М.Г. Дмитриевым, В.С. Полещуком \cite{DmiPol1972}, И.Ф. Дорофеевым \cite{Dor1979}. Далее в работах А.В. Гончарского и В.В. Степанова \cite{GonSte1979} А.Л. Агеева \cite{Ag1980} была доказана равномерная сходимость приближенных решений. Подход, изложенный в \cite{TikhGonSteYag1990}, основан на идее двухэтапного алгоритма: построении приближенного решения  исходного операторного уравнения из условия минимизации регуляризованной невязки на априорном множестве, где привлекается информация о неотрицательности, монотонности и выпуклости решения. %$$min\{\| A(u)-f_\delta\| ^2 + \alpha \Omega(u): u\in\Omega, \|f-f_\delta\|\le\delta \},$$ 
%где $A$ --- оператор задачи, $f$ --- правая часть без шума, $f_\delta$ --- возмущенная правая часть, $\delta$ --- уровень шума, $\alpha$ --- параметр регуляризации. 

На втором этапе для решения корректно поставленной экстремальной задачи применяются методы градиентного типа, линеаризованные методы, или алгоритмы, специально ориентированные на определенный класс априорных ограничений.

Для решения систем нелинейных уравнений предложены методы в работах Л.В. Канторовича \cite{Kan1947}, Б.Т. Поляка \cite{Pol1969}, J. M. Ortega и W. C. Rheinboldt \cite{OrtRhe1970}, A. Neubauer, O. Scherzer \cite{NeuSch1995_1, Sch1995}, M.J.D. Powell \cite{Pow1970}, J.E.Dennis, R.B. Schnabel, P.D. Frank \cite{DenSchn1996}, C.T. Kelley \cite{Kel1995}, R.B. Schnabel и P.D. Frank \cite{SchnFra1983} для решения систем уравнений с сингулярной или плохо обусловленной матрицей Якоби, J.C. Gilbert, J. Nocedal, S.J. Wright \cite{GilNoc1991, NocWri2006}. Термин <<$\alpha$-процессы>>, характеризующий класс нелинейных итерационных методов (где оператор шага нелинеен) для решения линейного уравнения с ограниченным самосопряженным положительно полуопределенным оператором, был введен в монографии М.А. Красносельского, Г.М. Вайникко, П.П. Забрейко \cite{KraVayZab1969}. Сходимость и устойчивость методов наискорейшего спуска и минимальной ошибки исследовалась авторами A. Neubauer, O. Scherzer в работе\cite{NeuSch1995_2}.

L. Landweber в статье \cite{Lan1951} 1951 г. предложил метод для решения линейных интегральных уравнений Фредгольма I рода. В дальнейшем авторы M. Hanke, A. Neubauer и O. Scherzer \cite{HanNeuSch1995,Neu2000,NeuSch1995_2} применили метод Ландвебера для решения нелинейных нерегулярных уравнений, доказали теоремы о сходимости и исследовали скорость сходимости метода. Градиентные методы с применением метода Ландвебера исследовались М.Ю. Кокуриным в работах \cite{Kok2010_1,Kok2010_2}.

В работах \cite{Han1997,Han2010} M. Hanke предложил новую схему метода Левенберга -- Марквардта для решения некорректных задач на примере задачи фильтрации.

В.В. Васиным предложен подход к решению задач с априорной информацией в работах ~\cite{Vas1982, Vas1988} и в монографиях ~\cite{VasAge1993, VasEre2009}, основанный на применении фейеровских отображений для учета априорных ограничений в форме выпуклых неравенств. Термин <<фейеровское отображение>> введен Ереминым в работах ~\cite{Ere1965, Ere1966, Ere1968} на основе идей венгерского математика Фейера. Отображения, обладающие свойством фейеровости, позволяют строить итерационные процессы с учетом априорных ограничений достаточно общего вида и, в отличие от метрической проекции, допускают эффективную реализацию. На основе $\alpha$-процессов были предложены регуляризованные методы решения линейных операторных уравнений Фредгольма $I$ рода, возникающих, например, при решении линейных обратных задач гравиметрии. Также Васин доказал сильную сходимость метода Левенберга -- Марквардта и его модифицированного варианта для решения регуляризованного по Тихонову нелинейного уравнения. Были приведены численные эксперименты для нелинейной обратной задачи гравиметрии в работах В.В. Васина и Г.Я. Пересторониной~\cite{VasPer_2011}, В.В. Васина ~\cite{Vasin_2012}. Они показали, что основной процесс Левенберга -- Марквардта существенно превосходит по точности модифицированный вариант и не требует жестких условий на начальное приближение, но обладает большей вычислительной сложностью, и, следовательно, требует больших затрат машинного времени.

При исследовании методов решения некорректных задач важное место занимает оценка погрешности регуляризованного решения по отношению к точному решению. Для уравнения с монотонным оператором исследовался метод Лаврентьева U. Tautenhahn \cite{Tau2002,Tau2004}, стратегия выбора параметра регуляризации по Тихонову исследовался авторами Q. Jin Zong-Yi Hou, O. Scherzer, H. W. Engl и K. Kunisch \cite{JinZon1997,JinZon1999,SchEngKun1993}. В.П. Тананой была доказана сходимость решения $L$-регуляризованной вариационной задачи к решению исходного операторного уравнения первого рода, продемонстрировав на примере двумерной обратной задачи гравиметрии \cite{Tan2003_2}.

Обратные задачи гравиметрии и магнитометрии в случае одной контактной поверхности в виде интегральных уравнений были поставлены Б.В. Нумеровым и Н.Р. Малкиным \cite{Num1930, Mal1931}. Вопросами единственности решения обратных задач теории потенциала занимались в разное время авторы П.С. Новиков \cite{Nov1938}, М.М. Лаврентьев, В.К. Иванов, А.И. Прилепко \cite{Pri1965}, А.В. Цирульский, Н. В. Федорова, В. В. Кормильцев \cite{FedTsi1976,TsiKor1990}.

Методы решения структурных обратных задач гравиметрии и магнитометрии предложены в работах В.Б. Гласко и др.\cite{GlaOstFil1970}, В.Н. Страхова \cite{Str1967,Str1969,Str1974_1,Str1974_2,Str1976}, П.С. Мартышко \cite{MarPru1982,MarTsi2008}. Метод локальных поправок, использующий свойства изменения гравитационного поля в отдельной точке предложен И.Л. Пруткиным \cite{Pru1983,Pru1986,Prutdiss1998}, на этой же идее основывается метод В.Е. Мисилова \cite{Mis2014,Mis2015}.

Вопросы параллельных вычислений рассмотрены в монографиях В.В. Воеводина и Вл.В. Воеводина \cite{VoeVoe2002}, Дж. Ортеги \cite{Ort1991}, Д.К. Фаддеева, В.Н. Фаддеевой \cite{FadFad1977}.
Параллельные алгоритмы для решения задач
гравиметрии и магнитометрии исследуются в работах Е.Н. Акимовой \cite{Aki1994,Aki2009,Akidiss2009,AkiBel2011,AkiMisKos2015}.
}

\mkcommonsect{objective}{Цели и задачи диссертационной работы:}{%
построить новые методы решения нелинейных операторных уравнений первого рода в гильбертовом пространстве, исследовать их сходимость. Предложить методы решения обратной задачи гравиметрии, использующие особенности физической модели.
}

\mkcommonsect{novelty}{Научная новизна.}{%
Результаты, полученные в диссертационной работе, являются новыми и состоят в следующем:

	в рамках двухэтапного метода построения регуляризующего алгоритма обоснованы сходимость метод Ньютона и нелинейные аналоги альфа-процессов: метод минимальной ошибки (ММО), метод наискорейшего спуска (МНС) и метод минимальных невязок (ММН). Также установлена сходимость модифицированных вариантов методов ММО, МНС, ММН, когда производная оператора вычисляется в начальной точке итераций. Рассмотрены два случая: оператор задачи является монотонным, либо оператор является конечномерным и его производная имеет неотрицательный спектр.
	
	Для решения систем нелинейных интегральных уравнений  с ядром оператора структурной обратной задачи гравиметрии для модели  двуслойной среды предложен экономичный покомпонентный метод, основанный на методе Ньютона. 
	
	Для решения систем нелинейных уравнений  структурных обратных задач гравиметрии в многослойной среде предложен метод на основе метода Левенберга -- Марквардта --- покомпонентный метод типа Левенберга -- Марквардта.
	
	Предложена вычислительная оптимизация метода Ньютона и его модифицированного варианта в виде перехода от плотно заполненной матрицы производной оператора к ленточной в силу особенности строения ядер интегральных операторов задач грави- магнитометрии.

}

\mkcommonsect{value}{Теоретическая и практическая значимость.}{%
Результаты, изложенные в диссертации, могут быть использованы для решения нелинейных операторных уравнений. В частности, на практике можно применять для обратных задач теории потенциала, для различных обратных задач фильтрации.
}

\mkcommonsect{results}{Положения, выносимые на защиту:}{%
1. Сформулированы и доказаны теоремы, устанавливающие сильную фейеровость оператора шага итераций методов:
	\begin{itemize}
		\item 	метод Ньютона;
		\item	метод минимальной ошибки и его модифицированный вариант;
		\item	метод наискорейшего спуска и его модифицированный вариант;
		\item	метод минимальных невязок и его модифицированный вариант.
	\end{itemize}
	
	Доказана сильная фейеровость оператора шага итераций данных методов в случае монотонного оператора задачи и в случае конечномерного оператора с производной, имеющей неотрицательный спектр. Доказывается линейная скорость сходимости итерационных процессов.
	
	2. Предложена вычислительная оптимизация метода Ньютона в приложении к задачам гравиметрии и магнитометрии для уменьшения объема вычислений. %обеспечивает более высокую точность численного решения, а также уменьшает время счета программ.
	Предложены покомпонентные методы:
		\begin{itemize}
			\item покомпонентный основанный на методе Ньютона для решения нелинейного интегрального уравнения в задаче гравиметрии в двухслойной среде;
			\item покомпонентный метод типа Левенберга -- Марквардта для решения систем нелинейных уравнений  структурных обратных задач гравиметрии в многослойной среде.
		\end{itemize} %Данные методы обладают меньшей вычислительной сложностью в отличие от классических методов Ньютона и Левенберга-Марквардта.
		
%Вычислительные эксперименты показывают, что предложенные метод позволяют существенно уменьшить вычислительную сложность задачи и являются экономичными по потреблению памяти ЭВМ.
		
		3. Разработан комплекс параллельных программ для многоядерных и графических процессоров с использованием технологий OpenMP, CUDA для решения зада с большим объемом вычилений. Комплекс протестирован на модельных и квазиреальных задачах.  
}

\mkcommonsect{approbation}{Степень достоверности и апробация результатов.}{%
Результаты, полученные в работе над диссертацией, полностью подтверждаются численными экспериментами. Основные результаты по материалам диссертационной работы докладывались на конференциях:

1. XIV и XV Уральская молодежная научная школа по геофизике (Пермь, 2013 г., Екатеринбург 2014 г.);

2. Международная коференция <<Параллельные вычислительные технологии>> (Ростов-на-Дону, 2014 г., Екатеринбург, 2015 г., Казань, 2017 г.);

3. Международная конференция <<Геоинформатика: теоретические и прикладные аспекты>> (Киев 2014, 2015, 2016 г.)

4. Международная конференция <<Актуальные проблемы вычислительной и прикладной математики>> (Новосибирск, 2014 г.)

5. Международный научный семинар по обратным и некорректно поставленным задачам (Москва, 2015 г.)
}

\mkcommonsect{pub}{Публикации.}{%
Материалы диссертации опубликованы в $11$ печатных работах, из них $3$
статей в рецензируемых научных изданиях \cite{VasAkiMin2013, VasSkur2017,Skur2017_2}, $3$ рекомендованных ВАК и проиндексированных Scopus \cite{AkSkur2014,AkSkur2015,AkSkur2016}, $3$ статей в
сборниках трудов конференций и $2$ тезисов докладов.
}

\mkcommonsect{contrib}{Личный вклад автора.}{%
Подготовка к публикации работ проводилась совместно с соавторами. Все результаты, представленные в данной работе, получены автором лично. Защищаемые положения отражают вклад автора в опубликованных работах. В работе \cite{VasSkur2017} автору диссертации принадлежат построение методов для решения нелинейных уравнений на основе $\alpha$-процессов, доказательства сходимости и сильной фейеровости регуляризованного метода Ньютона, сильной фейеровости нелинейных $\alpha$-процессов для монотонного оператора и оператора, производная которого имеет неотрицательный спектр, результаты численного моделирования. В работах \cite{VasAkiMin2013,MisMinDer2013,AkiMisMin2014} автором проведено численное моделирование для методов ньютоновского типа с разработкой параллельных программ для метода Ньютона и его модифицированного варианта. В статье \cite{AkMisSkTr2015} автор реализовал параллельный алгоритм линеаризованного метода минимальной ошибки. В работе \cite{AkSkur2014} автором предложена вычислительная оптимизация метода Ньютона и поставлен вычислительный эксперимент, разработаны параллельная программы. В работах \cite{AkiMinMis2014,AkSkur2015,AkSkur2016} автором предложены методы покомпонентного типа Ньютона и Левенберга -- Марквардта, проведены численные эксперименты, написаны параллельные программы для задач с большими сетками. В работе \cite{VasSkur2015} автору принадлежат доказательства сходимости модифицированных методов на основе $\alpha$-процессов в случае монотонного оператора задачи, а также результаты расчетов на ЭВМ.
}

\mkcommonsect{struct}{Структура и объем диссертации.}{%
Диссертация состоит из введения, обзора литературы, $3$ глав, заключения и библиографии.
Общий объем диссертации $\pageref{LastPage}$ страниц, включая $18$ рисунков, $14$ таблиц.
Библиография включает $112$ наименований, в том числе $12$ публикаций автора.
}
