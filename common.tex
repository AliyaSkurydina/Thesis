% Общие поля титульного листа диссертации и автореферата
\institution{Название организации}

\topic{Тема диссертации}

\author{Скурыдина Алия Фиргатовна}

\specnum{01.01.07}
\spec{Вычислительная математика}
%\specsndnum{01.04.07}
%\specsnd{Физика конденсированного состояния}

\sa{Акимова Елена Николаевна}
\sastatus{д.~ф.-м.~н., доц.}
%\sasnd{ФИО второго руководителя}
%\sasndstatus{к.~ф.-м.~н., проф.}

%\scon{ФИО консультанта}
%\sconstatus{д.~ф.-м.~н., проф.}
%\sconsnd{ФИО второго консультанта}
%\sconsndstatus{д.~ф.-м.~н., проф.}

\city{Екатеринбург}
\date{\number\year}

% Общие разделы автореферата и диссертации
\mkcommonsect{actuality}{Актуальность темы исследования.}{%
Построение итеративно регуляризованных алгоритмов востребовано для решения широкого круга прикладных задач. Так, решение структурных обратных задач гравиметрии и магнитометрии сводится к решению нелинейных интегральных уравнений Урысона первого рода.
}

\mkcommonsect{development}{Степень разработанности темы исследования.}{
Текст о степени разработанности темы.
}

\mkcommonsect{objective}{Цели и задачи диссертационной работы:}{%
построить новые методы решения нелинейных операторных уравнений, исследовать их сходимость.

Для достижения поставленных целей были решены следующие задачи:
\begin{itemize}
	\item для нелинейного уравнения с монотонным оператором доказаны теоремы сходимости для регуляризованного метода Гаусса---Ньютона, доказана сильная фейеровость итерационных процессов;
	\item построены регуляризованные методы градиентного типа, названные нелинейными аналогами $\alpha$-процессов, для нелинейного уравнения с монотонным оператором доказаны теоремы сходимости для них, доказана сильная фейеровость итерационных процессов;
	\item для задачи с немонотонным оператором с производной, имеющей неотрицательный спектр, доказаны теоремы сходимости методов Ньютона, нелинейных $\alpha$-процессов и их модифицированных вариантов;
	\item предложена вычислительная оптимизация метода Ньютона и его модифицированного варианта при решении задач с матрицей производной, близкой к ленточной;
	\item lля решения систем нелинейных интегральных уравнений  с ядром оператора структурной обратной задачи гравиметрии в двуслойной среде предложен покомпонентный метод, основанный на методе Ньютона;
	\item для решения систем нелинейных уравнений  структурных обратных задач гравиметрии в многослойной среде предложен подход на основе метода Левенберга-Марквардта – покомпонентный метод типа Левенберга-Марквардта;
	\item проведены численные эксперименты, интерпретированы результаты.
\end{itemize}
}

\mkcommonsect{novelty}{Научная новизна.}{%
Результаты, полученные в диссертационной работе, являются новыми и состоят в следующем:

	в рамках двухэтапного метода построения регуляризующего алгоритма обоснованы сходимость метод Ньютона и нелинейные аналоги альфа-процессов: метод минимальной ошибки (ММО), метод наискорейшего спуска (МНС) и метод минимальных невязок (ММН). Также установлена сходимость модифицированных вариантов методов ММО, МНС, ММН, когда производная оператора вычисляется в начальной точке итераций. Рассмотрены два случая: оператор задачи является монотонным, либо оператор является конечномерным и его производная имеет неотрицательный спектр.
	
	Для решения систем нелинейных интегральных уравнений  с ядром оператора структурной обратной задачи гравиметрии в двуслойной среде предложен покомпонентный метод, основанный на методе Ньютона. 
	Предложена вычислительная оптимизация метода Ньютона и его модифицированного варианта в виде перехода от плотно заполненной матрицы производной оператора к ленточной в силу особенности строения ядер интегральных операторов задач грави- магнитометрии.
	Для решения систем нелинейных уравнений  структурных обратных задач гравиметрии в многослойной среде предложен подход на основе метода Левенберга-Марквардта – покомпонентный метод типа Левенберга-Марквардта.

}

\mkcommonsect{value}{Теоретическая и практическая значимость.}{%
Результаты, изложенные в диссертации, могут быть использованы для решения нелинейных операторных уравнений, в частности, задач гравиметрии и магнитометрии.
}

\mkcommonsect{methods}{Методология и методы исследования.}{%
Текст о методах исследования.
}

\mkcommonsect{results}{Положения, выносимые на защиту:}{%
1. Сформулированы и доказаны теоремы, устанавливающие сильную фейеровость оператора шага итераций методов:
	\begin{itemize}
		\item 	метод Ньютона;
		\item	метод минимальной ошибки и его модифицированный вариант;
		\item	метод наискорейшего спуска и его модифицированный вариант;
		\item	метод минимальных невязок и его модифицированный вариант.
	\end{itemize}
	
	Доказана сильная фейеровость оператора шага итераций данных методов в случае монотонного оператора задачи и в случае конечномерного оператора с производной, имеющей неотрицательный спектр. Доказывается линейная скорость сходимости итерационных процессов.
	
	2. Предложена вычислительная оптимизация метода Ньютона, которая в задачах гравиметрии и магнитометрии обеспечивает более высокую точность численного решения, а также уменьшает время счета программ.
	
	3. Предложены покомпонентные методы:
		\begin{itemize}
			\item покомпонентный основанный на методе Ньютона для решения нелинейного интегрального уравнения в задаче гравиметрии в двухслойной среде;
			\item покомпонентный метод типа Левенберга-Марквардта для решения систем нелинейных уравнений  структурных обратных задач гравиметрии в многослойной среде.
		\end{itemize} Данные методы обладает меньшей вычислительной сложностью в отличие от классических методов Ньютона и Левенберга-Марквардта.
		
Вычислительные эксперименты показывают, что предложенные метод позволяют существенно уменьшить вычислительную сложность задачи и являются экономичными по потреблению памяти ЭВМ.
		
		4. Проведены численные эксперименты для модельных и квазиреальных геофизических данных, разработан комплекс параллельных программ для многоядерных и графических процессоров с использованием технологий OpenMP, CUDA. 
}

\mkcommonsect{approbation}{Степень достоверности и апробация результатов.}{%
Основные результаты по материалам диссертационной работы докладывались на конференциях:

1. XIV и XV Уральская молодежная научная школа по геофизике (Пермь, 2013 г., Екатеринбург 2014 г.);

2. Международная коференция "Параллельные вычислительные технологии" (Ростов-на-Дону, 2014 г., Екатеринбург, 2015 г., Казань, 2017 г.);

3. Международная конференция «Геоинформатика: теоретические и прикладные аспекты» (Киев 2014, 2015, 2016 г.)

4. Международная конференция "Актуальные проблемы вычислительной и прикладной математики" (Новосибирск, 2014 г.)

5. Международный научный семинар по обратным и некорректно поставленным задачам (Москва, 2015 г.)
}

\mkcommonsect{pub}{Публикации.}{%
Материалы диссертации опубликованы в $N$ печатных работах, из них $n_1$
статей в рецензируемых журналах~\cite{Ivanov_1999_Journal_17_173,
Petrov_2001_Journal_23_12321,Sidorov_2002_Journal_32_1531}, $n_2$ статей в
сборниках трудов конференций и $n_3$ тезисов докладов.
}

\mkcommonsect{contrib}{Личный вклад автора.}{%
Содержание диссертации и основные положения, выносимые на защиту, отражают персональный вклад автора в опубликованные работы.
Подготовка к публикации полученных результатов проводилась совместно с соавторами, причем вклад диссертанта был определяющим. Все представленные в диссертации результаты получены лично автором.
}

\mkcommonsect{struct}{Структура и объем диссертации.}{%
Диссертация состоит из введения, обзора литературы, $n$ глав, заключения и библиографии.
Общий объем диссертации $P$ страниц, из них $p_1$ страницы текста, включая $f$ рисунков.
Библиография включает $B$ наименований на $p_2$ страницах.
}
